\chapter{\cernvmreleasetesting Server Platform Setup}
\label{sct:serversetup}

\section{Introduction}
This section provides complete step by step instructions on how to setup and configure the \tapper server which are part of a basic working
\releasetesting environment by outlining the procedure for setting up the server, hence why this is called a \emph{walkthrough} document. 
This guide is intended for users familiar enough with computers and desktop environments to enter basic commands in a terminal and install 
various operating systems. 

\section{Red Hat Based Server Setup}
\subsection{Installing the system}
\subsection{Configuring the system}
\subsection{Installing the Tapper Server}
\subsection{Setting up Tapper Web Interface and Database}

\section{Debian Based Server Setup}
\subsection{Installing the system}
\subsection{Configuring the system}
\flushleft
\begin{enumerate}
\item For installing Debian, follow the instructions outlined in the sections ``Installing the system'' and ``Configuring the system'' 
for installation and configuration instructions with the only exception being that the hostname should be something unique such as 
\emph{cernvm-debian60 server}, to indicate that it is running the tapper server, {\bf again keep the hostname convention consistent}.
\end{enumerate}

\subsection{Installing the Tapper Server}
\begin{enumerate}
\item Next, execute the following commands to install necessary dependencies, \emph{from now on all commands require root privileges}.
\lstset{caption= Install Dependencies}
\begin{lstlisting}
$ apt-get update
$ apt-get install make
$ apt-get install subversion
\end{lstlisting}

\item Now, download the latest copy of the Tapper-Starerkit, which is an installer for Tapper from the \cernvmreleasetesting 
Google Code Project page
\lstset{caption= Download Tapper-Starterkit}
\begin{lstlisting}
$ svn checkout http://cernvm-release-testing.googlecode.com/svn/trunk/ cernvm-release-testing-read-only
\end{lstlisting}

\item Now edit the Makefile in the Tapper-Starterkit installer folder and configure variable TAPPER\_SERVER which 
is the hostname of the machine that is currently installing the starter-kit. For now disregard the TESTMACHINE 
variables, also \emph{make sure the variable TAPPER\_SERVER? is changed to TAPPER\_SERVER} you should have something
similar to this in the Makefile.
\lstset{caption= Makefile Configuration}
\begin{lstlisting}
# initial machine names
TAPPER_SERVER=cernvm-server
TESTMACHINE1=johnconnor
TESTMACHINE2=sarahconnor
TESTMACHINE3=bullock
\end{lstlisting}

\item After you have configured the Makefile in the installer folder, install the Tapper-Starterkit by executing the
following command, for any prompts during the installation leave them as default and press enter. {\bf During
the installation, you will be prompted for the mysql password, DO NOT ENTER a password here UNLESS you already have an
existing MySQL installation/database with a password set for the "root" account}
\lstset{caption= Install Tapper-Starterkit}
\begin{lstlisting}
$ make localsetup
\end{lstlisting}
\end{enumerate}

\subsection{Setting up Tapper Web Interface and Database}