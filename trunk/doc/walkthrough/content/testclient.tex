\chapter{\cernvmreleasetesting Test Client Platform Setup}
\label{sct:testclientsetup}

\section{Introduction}
The intent of this document is to provided a step-by-step guide on setting up an entire \cernvmreleasetesting\ infrastructure, including instructions
on how to set up and configure test clients, the main server running the web interface and database, as well as writing and executing tests. If you are
new to release testing and want a document to guide you through the entire process of setting up a working \cernvmreleasetesting\ infrastructure,
then this guide for you.

This section provides complete step by step instructions on how to setup and configure the test clients which are part of a basic working
\releasetesting environment by outlining the procedure for setting up test clients on numerous platforms, hence why this is called a 
\emph{walkthrough} document. This guide is intended for users familiar enough with computers and desktop environments to enter basic commands
in a terminal and install various operating systems. 

As this guide is directed towards users who are new to \cernvmreleasetesting and \tapper and
are interested in quickly getting a \cernvm testing infrastructure quickly set up, many assumptions regarding the requirements necessary are made.
As is the case, these instructions are provided for a generalized audience based on our own experience and the requirements that we feel most users
will have, \emph{so feel free to deviate from the instructions}. 

\section{Windows Test Client Setup}
\subsection{Installing the system}
\subsection{Configuring the system}
\subsection{Installing and configuring the hypervisors}
\subsection{Configuring the CernVM Image}
\subsection{Setting up the Tapper Test Suite}

\section{Red Hat Based Test Client Setup}
\subsection{Installing the system}
\subsection{Configuring the system}
\subsection{Installing and configuring the hypervisors}
\subsection{Configuring the CernVM Image}
\subsection{Setting up the Tapper Test Suite}

\section{OS X Test Client Setup}
\subsection{Installing the system}
\subsection{Configuring the system}
\subsection{Installing and configuring the hypervisors}
\subsection{Configuring the CernVM Image}
\subsection{Setting up the Tapper Test Suite}

\section{Debian Based Test Client Setup}
\subsection{Installing the system}
\subsection{Configuring the system}
\subsection{Installing and configuring the hypervisors}
\subsection{Configuring the CernVM Image}
\subsection{Setting up the Tapper Test Suite}