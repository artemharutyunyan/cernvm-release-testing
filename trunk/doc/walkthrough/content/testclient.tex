\chapter{\cernvmreleasetesting Test Client Platform Setup}
\label{sct:testclientsetup}

\section{Introduction}
The intent of this document is to provided a step-by-step guide on setting up an entire \cernvmreleasetesting\ infrastructure, including instructions
on how to set up and configure test clients, the main server running the web interface and database, as well as writing and executing tests. If you are
new to release testing and want a document to guide you through the entire process of setting up a working \cernvmreleasetesting\ infrastructure,
then this guide for you.

This section provides complete step by step instructions on how to setup and configure the test clients which are part of a basic working
\releasetesting environment by outlining the procedure for setting up test clients on numerous platforms, hence why this is called a 
\emph{walkthrough} document. This guide is intended for users familiar enough with computers and desktop environments to enter basic commands
in a terminal and install various operating systems. 

As this guide is directed towards users who are new to \cernvmreleasetesting and \tapper and
are interested in quickly getting a \cernvm testing infrastructure quickly set up, many assumptions regarding the requirements necessary are made.
As is the case, these instructions are provided for a generalized audience based on our own experience and the requirements that we feel most users
will have, \emph{so feel free to deviate from the instructions}. 

\section{Windows Test Client Setup}
\subsection{Installing the system}
\subsection{Configuring the system}
\subsection{Installing and configuring the hypervisors}
\subsection{Configuring the CernVM Image}
\subsection{Setting up the Tapper Test Suite}

\section{Red Hat Based Test Client Setup}
\subsection{Installing the system}
\subsection{Configuring the system}
\subsection{Installing and configuring the hypervisors}
\subsection{Configuring the CernVM Image}
\subsection{Setting up the Tapper Test Suite}

\section{OS X Test Client Setup}
\subsection{Installing the system}
\subsection{Configuring the system}
\subsection{Installing and configuring the hypervisors}
\subsection{Configuring the CernVM Image}
\subsection{Setting up the Tapper Test Suite}

\section{Debian Based Test Client Setup}
\subsection{Installing the system}
\flushleft
\begin{enumerate}
\item First, begin by loading the Debian Squeeze CD and select the `default install' at the initial boot menu.

\item For the first few installation options, such as default language and keyboard layout simply leave the default values and select
`next' unless you have different requirements than the default options.

\item When prompted to enter the hostname for the computer, enter a relevant hostname for the machine based on the hardware or operating
system it is running; the hostname should be relevant and unique as \tapper will use this hostname to idenify the system. A good naming
convention should refer to the hardware or operating system and call it a host to differentiate from the virtual machine that will be
running as a guest\footnote{This will make more sense later when you have to create and configure CernVM images}, for example a hostname such
as \emph{cernvm-debian6-host} could be used, {\bf whatever conventions you use make sure it is consistent}.

\item Next, when prompted to enter the password for the root password, set the root password to something you will not forget and is fairly
difficult with numbers and letters.

\item Next, when prompted to enter a username for the new user to be created on the system, set the username as something simple and relevant
such as \emph{cernvm} and the password to something you will not forget and is fairly complex with numbers and letters. But, most importantly 
{\bf keep this username and password consistent across all systems created as part of the infrastructure} as it makes administration and everything
else much easier.

\item When prompted for the timezone simply configure the timezone according to your local time zone, such as CEST

\item Now, when prompted by the installer to configure the partitioning layout, if there are other operating systems installed on the system 
select the `Automatic Partitioning' option, and if available select the option `Use Remaining Free Space'. Otherwise, if there are no other 
operating system installed on the hard drive select the `Manual' option, \emph{beware that doing so will erase everything on the hard drive}.
Using the manual option, create two primary partitions, with the first taking up the size of the hard drive minus twice the size of the amount
of RAM installed, and the second primary partition as a SWAP file using the remaining free space. The following is an example of what the partiton
layout would look like for a 40.0 GB hard drive with 2GB of ram.

\flushleft
\lstset{caption=Manual Partition Layout Example}
\begin{lstlisting}
#1	PRIMARY	36.0 GB		Bf	EXT4	/
#2	PRIMARY	4.0  GB		 f	SWAP	SWAP
\end{lstlisting}

\item When prompted, select yes to use a network mirror and select a mirror near your location and in a country with a reputable 
connection/internet infrastructure such as (USA, Netherlands, Germany, France).

\item When prompted to choose software to install, select the following
\begin{itemize}
\item 	Graphical desktop environment
\item 	SSH Server
\item	Standard system utilities
\end{itemize}
\end{enumerate}

\item After that the system should soon complete the installation, after the installation has finished ensure that you reboot the system and remove
the CD so that the system does not load the CD again when it starts.

\subsection{Configuring the system}
\begin{enumerate}
\item After the system has booted remove the follow unnecessary startup applications by selecting from the menu  
\verb|System -> Preferences -> Startup Applications|
\begin{itemize}
\item	bluetooth
\item	evolution alarm
\item	Gnome Login Sound
\item	visual assistance/aid
\item	screensaver
\item	update notifier
\item	volume control
\item	print queue
\item	any others you think are unnecessary based on your own discretion
\end{itemize}

\item Remove the follow unnecessary services by selecting from the menu \verb|System -> Administration -> Services|
\begin{itemize}
\item	alsa utils
\item	bluetooth
\item	CUPS
\item	exim4
\item	any others you think are unnecessary based on your own discretion
\end{itemize}

\item Next enable and configure remoted desktop from the menu \verb|System -> Preferences -> Remote Desktop| and ensure
that the following options are configured
\begin{itemize}
\item	Enable the option "Allow others to view your desktop"
\item	Enable the option "Allow other users to control your desktop"
\item	Disable the option "You must confirm access to this machine"
\end{itemize}

\item Next configure the system to login automatically at boot from the menu select \verb|System -> Administration -> Login Screen|
and then set it to login to the user account you created previously (such as cernvm) automatically.

\item Next, remove cd-rom support from sources.list, which is used by Debian for updates \footnote{And is a nuisance for any new user
as it forces you to find the CD and put it in the computer for the update to continue}, execute the following command with root priviledges
and comment out any lines that start with "deb cdrom" by using a \#
\lstset{caption=Removing CD-ROM Requirement for Updates}
\begin{lstlisting}
$ su -c "gedit /etc/apt/sources.list"
\end{lstlisting}

\item Again, continue to edit /etc/apt/sources.list still with root priviledges and ensure that each line ends with ``main contrib non-free'',
then save the file and do the following command with root priviledges
\lstset{caption=Updating the System}
\begin{lstlisting}
$ su -c "apt-get update"
\end{lstlisting}

\item Next, configure the screen saver from the menu \verb|System -> Preferences -> Screensaver| and ensure that the following options are 
configured
\begin{itemize}
\item	Disable the option ``Lock screen when screensaver active''
\end{itemize}

\item The following instructions involve enabling headless support so that you can remote desktop to the machine without having a monitor
connected to the computer
\begin{itemize}
\item[a.] Edit the xorg.conf file and put the following in it
\lstset{caption=Configuring Xorg for Headless Support}
\begin{lstlisting}
$ su -c "gedit /etc/X11/xorg.conf"

Section "Device"
Identifier "VNC Device"
Driver "vesa"
EndSection

Section "Screen"
Identifier "VNC Screen"
Device "VNC Device"
Monitor "VNC Monitor"
SubSection "Display"
Modes "1280x1024"
EndSubSection
EndSection

Section "Monitor"
Identifier "VNC Monitor"
HorizSync 30-70
VertRefresh 50-75
EndSection

\end{lstlisting}
	
\item[b.] Then edit grub and set the option ``nomodeset'', and proceed to update grub and reboot
\lstset{caption=Configuring Grub for Headless Support}
\begin{lstlisting}
$ su -c "gedit /etc/default/grub"

GRUB_CMDLINE_LINUX="nomodeset"

$ su -c "update-grub"
\end{lstlisting}
\end{itemize}
\end{enumerate}

\subsection{Installing and configuring the hypervisors}
\begin{enumerate}
\item The fiest step is to install the hypervisor, start by installing kvm and virsh using the following command with root priviledges
\lstset{caption=Installing KVM and Virsh Support}
\begin{lstlisting}
su 	-c "apt-get install qemu-kvm libvirt-bin"
\end{lstlisting}

\item Now, proceed to download and extract the desired kvm virtual machine image onto the system from the CernVM download portal\footnote{The images can be obtained here: \url{http://cernvm.cern.ch/portal/downloads} it is recommended that you download the KVM Basic image for your architecture}, for this guide the
basic image will be downloaded as it is the most practical image for the majority of users.

\lstset{caption=Download and Extract CernVM KVM Basic Image}
\begin{lstlisting}
$ wget http://rbuilder.cern.ch/downloadImage?fileId=1719
$ gunzip cernvm*.gz
\end{lstlisting}

\item Now, to ensure that KVM is properly configured and installed, follow this guide provided on the CernVM website
\url{http://cernvm.cern.ch/portal/kvm} {\bf except create the following virtual machine definition file using gedit}
\emph{You will need to change the following XML tags in the configuration file accordingly}
\begin{itemize}
\item \verb|<uuid>| by generating a uuid using the uuid command
\item \verb|<source file>| according to wherever you extract the cernvm kvm image
\item \verb|<mac address>| not a necessity, but change it to something slightly different
\end{itemize}

\lstset{caption=Create CernVM KVM Definition File}
\begin{lstlisting}
# Install uuid tool and generate uuid
$ su -c "apt-get install uuid"
$ uuid

# Create the cernvm.xml definition file and set the XML tags accordingly
$ gedit cernvm.xml

<domain type='kvm'>
  <name>cernvm</name>
  <uuid>b32147e7-9b89-dda9-b15d-53ba5f54f590</uuid>
  <memory>524288</memory>
  <currentMemory>524288</currentMemory>
  <vcpu>1</vcpu>
  <os>
    <type arch='x86_64' machine='pc-0.12'>hvm</type>
    <boot dev='hd'/>
  </os>
  <features>
    <acpi/>
    <apic/>
    <pae/>
  </features>
  <clock offset='utc'/>
  <on_reboot>restart</on_reboot>
  <on_crash>restart</on_crash>
  <devices>
    <emulator>/usr/bin/kvm</emulator>
    <disk type='file' device='disk'>
      <driver name='qemu' type='raw'/>
      <source file='/home/cernvm/image/cernvm-2.3.0-x86_64.hdd'/>
      <target dev='hda' bus='ide'/>
      <address type='drive' controller='0' bus='0' unit='0'/>
    </disk>
    <controller type='ide' index='0'>
      <address type='pci' domain='0x0000' bus='0x00' slot='0x01' 
      function='0x1'/>
    </controller>
    <interface type='network'>
      <mac address='52:54:00:ca:d5:d3'/>
      <source network='default'/>
      <target dev='vnet0'/>
      <address type='pci' domain='0x0000' bus='0x00' slot='0x03' 
      function='0x0'/>
    </interface>
    <serial type='pty'>
      <target port='0'/>
    </serial>
    <console type='pty'>
      <target type='serial' port='0'/>
    </console>
    <input type='mouse' bus='ps2'/>
    <graphics type='vnc' port='-1' autoport='yes'/>
    <video>
      <model type='cirrus' vram='9216' heads='1'/>
      <address type='pci' domain='0x0000' bus='0x00' slot='0x02'
      function='0x0'/>
    </video>
    <memballoon model='virtio'>
      <address type='pci' domain='0x0000' bus='0x00' slot='0x04'
      function='0x0'/>
    </memballoon>
  </devices>
</domain>
\end{lstlisting}

\item Finally, configure the the virtual machine network to automatically start when the system boots, then create the virual machine
and ensure that it can be started and stopped using virsh, \emph{all virsh commands require root privileges}.
\lstset{caption= Configure the Virtual Machine and Verify it Works}
\begin{lstlisting}
$ virsh net-autostart default
$ virsh define /home/cernvm/image/cernvm.xml
$ virsh start cernvm
# Wait about 2 minutes for the system to boot
$ virsh stop cernvm
\end{lstlisting}
\end{enumerate}

\subsection{Configuring the CernVM Image}
\begin{enumerate}
\item The next steps involve configuring the CernVM image to integrate with virsh as well as the test suite, first
start the virtual machine, execute the command to get the IP address of the CernVM image. Then follow this guide
provided on the CernVM website on how to configure the CernVM image and create a new user using the web interface \url{http://cernvm.cern.ch/portal/cvmconfiguration} and reboot the system, \emph{all virsh commands require root privileges}.
\begin{lstlisting}
$ virsh start cernvm
# Wait about 2 minutes for the system to boot 
# Then get IP Address using the following command
$ arp -an
\end{lstlisting}

\item Now that the system has booted, login in to the system using ssh for the user you created using the CernVM web interface
\begin{lstlisting}
$ ssh <user you created>@cernvm-image-ipaddress
\end{lstlisting}

\item Now that you are logged into the system through SSH, execute the following instructions to enable virsh console access
\lstset{caption= Enable Virsh Console Access}
\begin{lstlisting}
# Type the following command, and enter a root password you won't forget
$ passwd

# Change to the root account and enable console for root so that you can 
# login using virsh console
$ su

# Enable root login on tty
$ echo “ttyS0″ >> /etc/securetty

# Then add console=ttyS0 to the kernel parameter line in /etc/grub.conf
$ vi /etc/grub.conf

# Add getty to /etc/inittab file after all the other "tty" lines, 
# add the following line to /etc/inittab
s0:2345:respawn:/sbin/agetty -L 38400 ttyS0 vt100

# Reboot the system and the console should now work
$ virsh console cernvm
\end{lstlisting}

\item Login to the system using virsh console and then simply type "root" and enter the password to login as the 
root account, \emph{there will most likely not be any console display from the virtual machine until you press 
enter after entering the password}, then enable root login through SSH using the following commands.
\lstset{caption= Enable SSH Root Login}
\begin{lstlisting}
# Login as root using password you set with the passwd command
$ virsh console cernvm

# edit the file /etc/ssh/sshd_config and uncomment the line 
# PermitRootLogin yes	in order to enable root login
$ vi /etc/ssh/sshd_config
\end{lstlisting}

\item Next from the host machine \emph{ie. the machine you're currently using} enable automatic login as root 
through ssh on the KVM guest, first ensure that the guest machine has been started.
\lstset{caption= Enable Automatic SSH Root Login}
\begin{lstlisting}
$ virsh start cernvm

# Generate a public key, when prompted press enter for everything
$ ssh-keygen -t rsa

# Get the ip address of the running cernvm guest, as done previously
$ arp -an

# Next, run the following command to setup automatic login for ssh
# without having to type the password. When prompted for the password
# enter the password you just set previously
$ ssh-copy-id root@cernvm-image-ipaddress
		
# Now, try and login to the machine using ssh and ensure that you can
# login as root without having to type the password
$ ssh root@cernvm-image-ipaddress
\end{lstlisting}

\item If everying has worked so far, the test client and CernVM image have been configured properly, if you are
having issues solve them before proceeding further, or go the the section on ``Server Platform Setup'' as that
done not require virtual machine creation and configuration.
\end{enumerate}

\subsection{Setting up the Tapper Test Suite}
\item {\bf Before proceeding any further ensure that you have all other test clients set up this far, and then proceed
to follow the instructions for setting up and configuring the \tapper server in the section ``Server Platform Setup''}.
