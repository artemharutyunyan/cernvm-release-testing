\section{Debian Based Server Setup}
\subsection{Installing ad Configuring the system}
\flushleft
\begin{enumerate}
\item 	For installing a Debian based server, follow the instructions outlined in the sections ``Installing the 
		system''~\ref{sec:debianinstall} and ``Configuring the system''~\ref{sec:debianconfig} for installation and configuration 
		instructions with the only exception being that the hostname should be something unique such as \emph{cernvm-debian6-server}, to 
		indicate that it is running the \tapper server, {\bf again keep the hostname convention consistent}.
\end{enumerate}


\newpage
\subsection{Installing the Tapper Server}
\begin{enumerate}
\item 	Next, execute the following commands to install necessary dependencies, \emph{from now on all commands require root privileges}.

\lstset{language=bash,caption= Install Dependencies}
\begin{lstlisting}
$ apt-get update
$ apt-get install make
$ apt-get install subversion
\end{lstlisting}

\item 	Now, download the latest copy of the Tapper-Deployment, which is an installer for Tapper from the \cernvmreleasetesting 
		Google Code Project page

\lstset{language=bash,caption= Download Tapper-Deployment}
\begin{lstlisting}
$ svn checkout http://cernvm-release-testing.googlecode.com/svn/trunk/\
installer/tapper-deployment
\end{lstlisting}

\item 	Now edit the Makefile in the Tapper-Deployment installer folder and configure variable TAPPER\_SERVER which 
		is the hostname of the machine that is currently installing the starter-kit. For now disregard the TESTMACHINE 
		variables, you should have something similar to this in the Makefile.

\lstset{language=bash,caption= Makefile Configuration}
\begin{lstlisting}
# initial machine names
TAPPER_SERVER=cernvm-server
TESTMACHINE1=johnconnor
TESTMACHINE2=sarahconnor
TESTMACHINE3=bullock
\end{lstlisting}

\item	After you have configured the Makefile in the installer folder, install Tapper-Deployment by executing the
		following command, for any prompts during the installation leave them as default and press enter. {\bf During
		the installation, you will be prompted for the mysql password, DO NOT ENTER a password here UNLESS you already have an
		existing MySQL installation/database with a password set for the ``root'' account}. Finally, if you have any errors or
		other issues during the installation, please contact us with a summary of the problem and send us a copy of the installation 
		log ``install.log''.

\lstset{language=bash,caption= Install Tapper-Deployment}
\begin{lstlisting}
$ cd installer/
$ make localsetup 2>&1 | tee install.log
\end{lstlisting}
\end{enumerate}


\newpage
\subsection{Setting up Tapper Web Interface and Database}
\begin{enumerate}
\item 	Next you need to set a password for the root account of the mysql database\footnote{This will eventually be implemented in the 
		makefile}

\lstset{language=bash,caption= Set MySQL Root Password}
\begin{lstlisting}
# Example: mysqladmin -u root password abc123
$ mysqladmin -u root password <newpassword>
\end{lstlisting}

\item 	Now that the installion has completed and the security issue has been dealt with, ensure that you can access the tapper web
		interface and that it is working by viewing it in your browser using the url, \url{http://localhost/tapper}\footnote{This can be accessed
		locally and remotely from other systems using the server hostname or IP address}

\item 	Next, install PHPMyAdmin so that it's easy to administrate and configure the Tapper databases, when prompted to select the
		``Web server to reconfigure automatically'' select apache2 by pressing the space bar and press enter. \emph{If you are 
		prompted to ``configure databases for phpmyadmin with dbconfig-common'' select {\bf NO} }.

\lstset{language=bash,caption= Install PHPMyAdmin}
\begin{lstlisting}
$ apt-get update

# When prompted for the server to reconfigure automatically select apache2
# when prompted to configure the database with dbconfig-common select NO
$ apt-get install phpmyadmin
\end{lstlisting}

\item	Verify that PHPMyAdmin has been installed and configured correctly and that you can access the tapper web
		interface by viewing it in your browser using the url, \url{http://localhost/phpmyadmin}. Login to the PHPMyAdmin
		web interface using the \emph{username root and the MySQL root password you set earlier using mysqladmin}.

\item 	Now, add all of the configured test machines created in the ``Test Client Platform Setup'' section to the
		\tapper database and set the test clients as active, then add the hardware specifications for each test client
		to the database. This example is just using a single generic test machine, you will have to repeat these commands
		for each test client and change the hostname \emph{cernvm-host} and the values for mem, core,vendor, and has\_ecc 
		as needed; the vendor can be AMD or Intel.
		
\lstset{language=bash,caption= Adding Test Clients to Tapper Database}
\begin{lstlisting}
# Add the hostname of the test client to database
$ tapper-testrun newhost --name cernvm-host --active

# Add the hardware specifications for the test client
$ mysql testrundb -utapper -ptapper
$ insert into host_feature(host_id, entry, value)  values \
((select id from host where name = 'cernvm-host'),     'mem',  4096);
$ insert into host_feature(host_id, entry, value)  values \
((select id from host where name = 'cernvm-host'),   'cores',     4);
$ insert into host_feature(host_id, entry, value)  values \
((select id from host where name = 'cernvm-host'),  'vendor', 'AMD');
$ insert into host_feature(host_id, entry, value)  values \
((select id from host where name = 'cernvm-host'), 'has_ecc',     0);
\end{lstlisting}


\item 	Next, send a sample test report to the tapper server, to ensure that the web interface, MCP, database, and reports
		framework are all working by viewing the tapper reports in your browser at the following url, 
		\url{http://localhost/tapper/reports} You should now see a report from whatever the ``Tapper-Machine-Name'' in 
		demo\_report.tap was set as. \emph{For the example demo\_report.tap provided below it would be cernvm-server}.

\lstset{language=bash,caption= Send a Report from the \tapper~Server to Itself}
\begin{lstlisting}
# Save the following in a file named demo_report.tap
$ vi demo_report.tap

	1..2
	# Tapper-Suite-Name: Tapper-Deployment
	# Tapper-Suite-Version: 1.001
	# Tapper-Machine-Name: cernvm-server
	ok - Hello test world
	ok - Just another description

# Send the report to the tapper server using netcat
$ cat demo_report.tap | netcat -q7 -w1 cernvm-server 7357
\end{lstlisting}

\item 	Finally, ssh login to one of the test machine that was set up earlier, \emph{in our examples, cernvm-host} and send another 
		sample test report to the tapper server, to ensure that the web interface, MCP, database, and reports framework are all working 
		by viewing the tapper reports in your browser at the following url: \url{http://localhost/tapper/reports}. You should now see a 
		report from whatever the ``Tapper-Machine-Name'' in demo\_report.tap was set as. \emph{For the example demo\_report.tap provided 
		below it would be cernvm-testclient}.

\lstset{language=bash,caption= Send a Report to the \tapper~Server from a Test Client}
\begin{lstlisting}
# Save the following in a file named demo_report.tap
$ vi demo_report.tap

	1..2
	# Tapper-Suite-Name: Tapper-Deployment
	# Tapper-Suite-Version: 1.001
	# Tapper-Machine-Name: cernvm-testclient
	ok - Hello test world
	ok - Just another description

# Send the report to the tapper server using netcat
$ cat demo_report.tap | netcat -q7 -w1 cernvm-server 7357
\end{lstlisting}

\item	Now that it has been verified that the tapper server, including the web interface, MCP, database, and reports framework 
		are all working; return to the sections titled ``Setting up the Tapper Test Suite''	for each of the test client, as 
		there are unique instructions for each operating system.
\end{enumerate}