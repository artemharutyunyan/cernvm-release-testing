\section{OS X Test Client Setup}
\subsection{Installing the system}
\label{sec:osxinstall}

In most cases your Apple computer should have already came with the OS X operating system installed, but in case you need to
install the operating system manually, or perhaps you wish to install the latest operating system, such as Snow Leopard or Lion
the following instructions are provided. The following instructions outline the procedure for installing OS X on your Apple computer, 
if your Apple computer already came installed with one of the newer versions of OS X, then simply procede to the next section,
``Configuring the system''~\ref{sec:osxconfig}.

\begin{enumerate}
\item	First, begin by booting from the OS X installation CD and simply wait until it boots into the installation environment.

\item	Now that the installation CD has booted, the next procedure is to prepare the hard drive you with to install OS X on 
		for installation. From the menu at the top of the screen, select \verb|Utilties -> Disk Utility...| then from the left 
		pane, select the drive that you wish to install OS X on. Now, it is a bit unintuitive, but in order to format and parition 
		the drive so that it can be accessed by the OS X installer, you must click on the ``Erase'' tab, \emph{It is pretty self 
		explanatory, but this will erase everything on the hard drive}. Now, for the option ``Volume Format:'' from the drop down menu
		select ``Mac OS Extended (Journaled)'' \footnote{This is the optimal format to use as journaled file systems are the least likely
		filesystem to become corrupted}, then for the option ``Name:'' enter a name for the hard drive, \emph{this is not a name for the
		computer}, so simply enter a name in all capital letters such as \emph{CERNVM}. Finally, click the ``Erase...'' button to format
		the drive and prepare it for OS X installation.
		
\item	Now, go back to the OS X installer and read through and agree to the End User License Agreement, \emph{make sure you read through
		the EULA and agree to the terms}, then click the ``Agree'' button to continue the installation.
		
\item	Next, at the installation stage titled, ``Select a Destination'', click on the hard drive you formatted previously, as well 
		the name you entered for the drive when it was created should be listed below the hard drive icon, then click ``Continue''.
		
\item	At the next stage of the intallation titled, ``Install Summary'', if you wish to make customizations to the packages installed
		as part of the OS X system, click the ``Customize'' button, otherwise simply click the ``Install'' button to begin the
		installation of OS X.
		
\item	After the system reboots, you will be presented with a welcome screen, select your country from the list and then click
		``Continue''.
		
\item 	At the next installation stage, titled ``Select Your Keyboard'', chose a keyboard layout, it is probably best to just leave
		it at the default ``U.S.'' keyboard layout.
		
\item	At the next installation stage, titled ``Do You Already Own a Mac?'' simply select the option ``Do not transfer my information
		now'' and click ``Continue''.
		
\item	Next, for ``Enter Your Apple ID''\footnote{This is your iTunes and Apple store account, since this is a test client we won't
		be using iTunes} leave everything blank and simply click ``Continue''.
		
\item	Next for the ``Registration Information'', again just click ``Continue'' and if you are prompted by a message about ``Some
		of your registration information is missing'', just click ``Continue'' again.
		
\item	Now, at the next installation stage titled ``Create Your Account'', for the options ``Full Name'' and ``Account Name:''
		enter a name and account name that is simple and relevant such as \emph{cernvm} and set the password to something you will
		not forget and is fairly complex with numbers and letters. But, most importantly {\bf keep the username and password entered
		consistent across all systems created as part of the infrastructure} as it makes administration and everything else much easier. 
		
\item	Next for ``Select Time Zone'' click on your location on the map of the world to change the time zone, and then for the option
		``Closest City:'' select a city near your location.
		
\item	Finally, at the last stage of the installation click ``Done''.	
\end{enumerate}




\subsection{Configuring the system}
\label{sec:osxconfig}

\begin{enumerate}
\item	After the system has booted, the first thing to configure are the power management settings and other preferences, as this system
		will be running as a test client, sleep and other automatic energy saving features must be disabled. Begin by navigating to the
		power options, \verb|Apple logo -> System Preferences -> Energy Saver| for the option ``Computer sleep'' slide the bar to the
		far right so that it is set to ``Never'' and ensure that the following options are all disabled.
\begin{itemize}
\item	Put the hard disk(s) to sleep when possible

\item	Wake for Ethernet network access

\item 	Allow power button to put the computer to sleep
\end{itemize}

\item	Next, set a hostname for the system from the menu \verb|Apple logo -> System Preferences -> Sharing| beside the ``Computer Name:''
		option at the top, click the ``Edit...'' button. Then enter a relevant hostname for the system based on the hardware or operating
		system it is running; the hostname should be relevant and unique to better identify the system. A good naming convention should 
		refer to the hardware or operating system and call it a host to differentiate from the virtual machine that will be running as 
		a guest, for example a hostname such as \emph{cernvm-osx-host} could be used, {\bf whatever convention you use make sure it is 
		consistent}.

\item	Next, enable SSH access to the system by navigating to \verb|Apple logo -> System Preferences -> Sharing| and from the list of 
		services that can be shared, enable ``Remote Login'', which is SSH.
		
\item	Now, to enable VNC access to the system, select from the same list of services that can be shared, ``Remote Management'' and
		for local access options window that appears, enable all of the options listed such as ``Observe'' and ``Change settings''.
		Then to enable VNC compatibilty so that the OS X system can be accessed by other non-Apple computers, click the ``Computer
		Settings...'' button and enable the following options and set a password for the ``VNC viewers...'' option.
\begin{itemize}
\item	Show Remote Management status in menu bar
\item	Anyone may request permission to control screen
\item	VNC viewers may control screen with password
\end{itemize}

\item	Now, to ensure that your user logs in automatically, navigate to \verb|Apple logo -> System Preferences -> Accounts| and click
		``Login Options'', you may have to click on the lock icon and enter your password in order to make changes to the login options. 
		Then for the option ``Automatic login:'' select your user from the list of accounts to enable automatic login.
		
\item 	Finally, to ensure that the settings were configured properly, reboot the machine and ensure that the following work.
\begin{itemize}
\item	It automatically boots up into the full desktop environment without having to login
\item	You have access to the machine using SSH and can login
\item	You have VNC access to the machine and can control the system using VNC	
\end{itemize}
\end{enumerate}




\subsection{Installing libvirt and virsh}
\label{sec:osxvirsh}
\begin{enumerate}
\item	The virtualization API libvirt and the command line tool virsh~\cite{libvirt} are the essential components required 
		for setting up a test client and must be installed and properly configured before any testing can begin. Ensure that
		you follow the proceeding directions carefully and validate that virsh is working properly before proceeding to 
		install and configure the various hypervisors.

\item	First, to install libvirt, begin by installing Homebrew\footnote{Homebrew is a package manager for OS X, which is similar to apt, it will
                be used to install dependencies instead of manually building from source} using the following command, for a more detailed
                installation guide refer to the Homebrew wiki  \url{https://github.com/mxcl/homebrew/wiki/ Installation}.

\lstset{language=bash, caption=Install Homebrew}
\begin{lstlisting}
# Install Homebrew using the following command
/usr/bin/ruby -e "$(curl -fsSL https://raw.github.com/gist/323731)"
\end{lstlisting}

\item        Next, download and install Xcode, which is freely available at \url{http://developer.apple.com/xcode/}. At the moment Xcode 3 can
                be downloaded for free, whereas Xcode 4 costs money and must be purchased from the Apple App store\footnote{There is still one
                catch, in order to download Xcode 3, which is free, you must first make an Apple developer account}, this guide uses the freely
                available Xcode 3, which at the time of writing is version 3.2.6.

\item        Next, to install libvirt simply use the following command, please note, in the unlikely event that there has not been an updated
                "formula" to install a newer version of libvirt, then you can attempt to either create one or update the current libvirt formula and 
                benefit yourself and others who may install libvirt using Homebrew.  The following are instruction for creating a formula 
                \url{https://github.com/mxcl/homebrew/wiki/Formula- Cookbook}.
                
\lstset{language=bash, caption=Install libvirt}
\begin{lstlisting}
# Install libvirt using the following command
brew install libvirt
\end{lstlisting}

\item 	Finally, ensure that virsh installed correctly and is running by connecting to the test hypervisor and ensuring 
		that the test virtual machine, named ``test'' is running.

\lstset{language=bash,caption=Verify virsh was Installed Properly}
\begin{lstlisting}
# Verify virsh is working, test should be running
$ virsh -c test:///default list --all
\end{lstlisting}
\end{enumerate}




\subsection{Installing and configuring VirtualBox}
\label{sec:osxvbox}
\begin{enumerate}
\item	First, begin by downloading and installing a version of VirtualBox supported by libvirt for OS X from the VirtualBox download 
		page, it is best to download the latest version within the series \emph{that has been available for at least a month prior to the release
		of the version of libvirt installed}. VirtualBox can be downloaded from the following location,  
		\url{http://www.virtualbox.org/wiki/Downloads} ensure that you select the appropriate architecture for your system. The following
		 instructions for this section of the guide uses VirtualBox 4.0.10 for AMD64.
		
\item	Next, to verify that VirtualBox has been installed properly and that virsh can connect to the VirtualBox hypervisor, 
		verify that the following VirtualBox kernel extensions are loaded.
		
\begin{itemize}
\item	org.virtualbox.kext.VBoxDrv
\item	org.virtualbox.kext.VBoxUSB
\item	org.virtualbox.kext.VBoxNetFlt
\item	org.virtualbox.kext.VBoxNetAdp
\end{itemize}

\lstset{language=bash,caption=Verify that VirtualBox Kernel Exentsions are Loaded}
\begin{lstlisting}
# Verify that the kernel extentsions are loaded
$ kextstat | grep -i virtualbox
\end{lstlisting}

\item        Finally, verify that VirtualBox has been configured properly and that virsh supports the current version of VirtualBox installed
                by connecting to the virsh console for the VirtualBox hypervisor.

\lstset{language=bash,caption=Verify VirtualBox Works with Virsh}
\begin{lstlisting}
# Verify that virsh can connect to virtualbox
$ virsh --connect vbox:///session
\end{lstlisting}
\end{enumerate}




\subsection{Installing and configuring VMware}
\label{sec:osxvmware}
\begin{enumerate}
\item	First, begin by downloading and installing the latest version of VMware Fusion for OS X from the VMware product 
		page, \url{http://www.vmware.com/products/}, VMware Fusion requires a license, so you will have to purchase it
		in order to continue.
		
\item	Next, to verify that VMware Fusion has been installed properly, verify that the following VMware kernel extensions 
		are loaded, currently virsh has support to connect to the VMware hypervisor and has only recently supported connecting
		to VMware Fusion since I added support in version 0.9.5 of libvirt.
		
\begin{itemize}
\item	com.vmware.kext.vmx86
\item	com.vmware.kext.vmci
\item	com.vmware.kext.vmioplug
\item	com.vmware.kext.vmnet
\end{itemize}

\lstset{language=bash,caption=Verify VMware Kernel Extensions Loaded}
\begin{lstlisting}
# Verify that the kernel extentsions are loaded
$ kextstat | grep -i vmware
\end{lstlisting}

\item	Next, add VMware Fusion to the system PATH variable permanently so that VMware Fusion can be executed from the 
			command line as well as accessed by virsh.

\lstset{language=bash,caption=Add VMware Fusion to PATH}
\begin{lstlisting}
# Add VMware Fusion to PATH
$ echo 'export PATH=/Library/Application\ Support/VMware\ Fusion/:$PATH' >> ~/.profile
$ echo 'export PATH=/Applications/VMware\ Fusion.app/Contents/MacOS:$PATH' >> ~/.profile

# Source the file, set PATH variable
$ . ~/.profile
\end{lstlisting}

\item	Finally, verify that VMware Fusion has been configured properly and that virsh supports the current version of VMware Fusion
            installed by connecting to the virsh console for the VMware Fusion hypervisor.

\lstset{language=bash,caption=Verify VMware Fusion Works with Virsh}
\begin{lstlisting}
# Verify that virsh can connect to vmware fusion
$ virsh --connect vmwarews:///session
\end{lstlisting}
\end{enumerate}




\subsection{Setting up the Tapper Test Suite}
\label{sec:osxtestsuite}
\begin{enumerate}
\item 	{\bf Before proceeding any further ensure that you have all other test clients set up this far, and then proceed
		to follow the instructions for setting up and configuring the \tapper~server in the section ``Server Platform Setup''}		
		~\ref{sec:serversetup}.
		
\item 	Now that the \tapper~server has been installed and configured and the \tapper web interface and database have proven
		to be working, the next step is to verify that the test client can actually send a report to the \tapper~server in
		the form of a TAP file. After sending the TAP report to the server, ensure that the test client is working by viewing 
		the tapper reports in your browser at the following url: \url{http:/<tapper\_server>/tapper/reports}. You should now see a 
		report from the test client, there should be a report from a system named whatever the ``Tapper-Machine-Name'' in 
		demo\_report.tap was set as. \emph{For the example demo\_report.tap provided below it would be cernvm-osxtestclient}.
		\footnote{This is why a consistent hostname convention was emphasized earlier, as reports are often sorted and organized 
		based on hostnames}.
		
\lstset{language=bash,caption=Send a Basic Report to the \tapper~Server}
\begin{lstlisting}
# Save the following in a file named demo_report.tap

	1..2
	# Tapper-Suite-Name: Tapper-Deployment
	# Tapper-Suite-Version: 1.001
	# Tapper-Machine-Name: cernvm-osxtestclient
	ok - Hello World
	ok - Just another description

# Send the report to the tapper server using netcat
$ cat demo_report.tap | nc -w10 cernvm-server 7357
\end{lstlisting}

\item 	Next, download a copy of the CernVM Test Suite and the CernVM Test Cases from the Google Code svn repository
		\cite{GCreleasetesting} and install the the following dependencies.
			
\lstset{language=bash,caption=Install CernVM Test Suite}
\begin{lstlisting}
# Install subversion, required to checkout auto-tapper
$ brew install subversion

# Checkout a copy of auto-tapper and cernvm testcases
$ svn checkout http://cernvm-release-testing.googlecode.com/svn/\
trunk/tapper/tapper-autoreport/ cernvm-testsuite
\end{lstlisting}

\item        Now, install the following dependencies using Homebrew\footnote{This section is extremely short in comparison to the previous
                instructions for this section which required building the dependencies from source}.
                
\lstset{language=bash,caption=Install Dependencies}
\begin{lstlisting}
# Update Homebrew
$ brew update

# Install the following dependencies
$ brew install bash
$ brew install bash-completion
$ brew install ssh-copy-id
$ brew install wget
$ brew install ossp-uuid
$ brew install spawn-fcgi
$ brew install md5sha1sum
$ brew install curl
\end{lstlisting}

\item	Next, set the default terminal as the newer version of bash that was installed with Homebrew and then verify that
			the new version of bash has been properly configured as the default terminal.

\lstset{language=bash,caption=Configure Bash}
\begin{lstlisting}
# Change to root account, enter password if prompted
$ sudo su

# Add the new bash shell to the list of acceptable shells
$ echo '/usr/local/bin/bash' >> /etc/shells

# For any account that the script will be executed on 
# set the default shell as the new bash shell installed
$ chsh -s /usr/local/bin/bash

# Verify the new version of bash has been configured, check 
# the version of bash installed matches "brew info bash" output
$ echo $BASH_VERSION
\end{lstlisting}

\item	Now that cernvm-testsuite has been installed, configure the variables in the configuration file for the
			hypervisors you want to test and according to your \tapper infrastructure setup. Sample configuration files are provided
			in the {\bf config} folder, all of the settings provided in the configuration files by default are the {\bf mandatory, minimum 
			configuration options} and in most cases the	defaults should be sufficient for testing, the only mandatory settings that are
			not provided by default {\bf and must be configured manually are CVM\_TS\_REPORT\_SERVER and 
			CVM\_VM\_IMAGE\_VERSION }. Please refer to the developer manual for a more complete detailed list of the mandatory and 
			optional configuration settings.

\begin{description}
\item[CVM\_TS\_SUITENAME]	Must ALWAYS be set, the default suite name in the configuration file should be suitable

	  		  
\item[CVM\_TS\_SUITEVERSION] 	Must ALWAYS be set, reflects the release version number of the test suite framework, the  default
								version given in the configuration should only be changed if you customize/update the scripts

\item[CVM\_TS\_REPORT\_SERVER]	Must ALWAYS be set, this is the ip address or hostname of the Tapper report server which the reports
      							from the test results are sent to
		
\item[CVM\_TS\_DOWNLOAD\_PAGE]	Must ALWAYS be set, normally the default url provided in the configuration file is accurate, but in the
								event that the internal \cernvm image release page is relocated then this url must be changed.

\item[CVM\_VM\_HYPERVISOR]	Must ALWAYS be set, should not have to change the defaul hypervisor in the configuration files, valid values 
							(case sensitive) are {\bf kvm,vbox,vmware}

\item[CVM\_VM\_TEMPLATE]		Must ALWAYS be set, normally the default template provided in the configuration file should not be changed,
								only change this to use a custom template file. The custom template file \emph{must be placed within the
								templates folder}

\item[CVM\_VM\_NET\_TEMPLATE]	Must ALWAYS be set, normally the default network template provided in the configuration file should not be
	 							changed, only change this to use a custom network template file for the \cernvm image,  {\bf only applies to kvm
	 							and virtualbox}. The custom network template file \emph{must be placed within the templates folder}

\item[CVM\_VM\_IMAGE\_VERSION]	Must ALWAYS be set, specifies the version of the CernVM image to use from the release page
	
\item[CVM\_VM\_IMAGE\_TYPE]		Must ALWAYS be set,  specifies the type of CernVM image, valid image types supported, (case sensitive) are 
								{\bf basic and desktop}

\item[CVM\_VM\_ARCH]		Must ALWAYS be set, specifies the architecture of the \cernvm image, valid architectures (case sensitive) are {\bf 
							x86 and x86\_64}
\end{description}	

\item	Finally, now that cernvm-testsuite has been installed and configured on the test client and the test client and \tapper~Server
			have proven to be working, the next step is to verify that cernvm-testsuite works correctly and can actually send a report to 	
			the \tapper~server in the form of a TAP file. To execute the \cernvm Test Cases script, ``cernvm-tests.sh'' simply source
			the configuration file you wish to use and execute the script. Once the script has completed and sent a TAP report to the server, 
			ensure that the test client is working by viewing the tapper reports in your browser at the  following 
			url: \url{http:/<tapper\_server>/tapper/reports}. You should now see a new report from the test client, there should be a 
			report from a system with the same hostname.\footnote{This is why a consistent hostname convention was emphasized earlier, as 
			reports are often sorted and organized based on hostnames}.

\lstset{language=bash,caption=Execute CernVM Test Cases Script}
\begin{lstlisting}
# Simply source the configuration file, and execute the script
$ . ./config/<configuration_file>
$ ./cernvm-tests.sh
\end{lstlisting}
\end{enumerate}