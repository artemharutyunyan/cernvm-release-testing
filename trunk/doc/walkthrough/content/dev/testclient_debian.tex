\section{Debian Based Test Client Setup}
\subsection{Installing the system}
\label{sec:debianinstall}
\flushleft
\begin{enumerate}
\item 	Install the system as you would for any other Linux distribution, except pay attention to the following instructions on
		configuring Debian to use ext4 (if available) instead of ext3 as the performance gains are noticeable.


\item	Now, when prompted by the installer to configure the partitioning layout, if there are other operating systems installed on the 
		system select the ``Guided - use entire disk'' option, and if available select the option ``Use Remaining Free Space''. 
		Otherwise, if there are no other operating system installed on the hard drive select the ``Manual'' option, \emph{beware that 
		doing so will risk erasing everything on the hard drive if you create a new partition table}. Using the manual option, create 
		two primary partitions, with the first taking up the size of the hard drive minus twice the size of the amount of RAM installed, 
		and the second primary partition as a SWAP file using the remaining free space. The following is an example of what the partiton 
		layout would look like for a 40.0 GB hard drive with 2GB of ram.

\lstset{language=bash,caption=Manual Partition Layout Example}
\begin{lstlisting}
#1	PRIMARY	36.0 GB		B f	 EXT4	/
#2	PRIMARY	4.0  GB		  f	 SWAP	SWAP
\end{lstlisting}

\item 	Finally, the last important installation setting, when prompted to choose software to install, select the following
\begin{itemize}
\item 	Graphical desktop environment
\item 	SSH Server
\item	Standard system utilities
\end{itemize}
\end{enumerate}




\subsection{Configuring the system}
\label{sec:debianconfig}
\begin{enumerate}
\item	After the system has booted remove the follow unnecessary startup applications by selecting from the menu  
		\verb|System -> Preferences -> Startup Applications|
\begin{itemize}
\item	bluetooth
\item	evolution alarm
\item	Gnome Login Sound
\item	print queue
\item	screensaver
\item	update notifier
\item	visual assistance/aid
\item	volume control
\item	any others you think are unnecessary based on your own discretion
\end{itemize}

\item Remove the follow unnecessary services by selecting from the menu \verb|System -> Administration -> Services|
\begin{itemize}
\item	alsa utils
\item	bluetooth
\item	CUPS
\item	exim4
\item	any others you think are unnecessary based on your own discretion
\end{itemize}

\item Next enable and configure remote desktop from the menu \verb|System -> Preferences -> Remote Desktop| and ensure
that the following options are configured
\begin{itemize}
\item	Enable the option ``Allow others to view your desktop''
\item	Enable the option ``Allow other users to control your desktop''
\item	Disable the option ``You must confirm access to this machine''
\end{itemize}

\item Next configure the system to login automatically at boot from the menu select \verb|System -> Administration -> Login Screen|
and then set it to login to the user account you created previously (such as cernvm) automatically.

\item Next, remove cd-rom support from sources.list, which is used by Debian for updates \footnote{And is a nuisance for any new user
as it forces you to find the CD and put it in the computer for the update to continue}, execute the following command with root privileges
and comment out any lines that start with ``deb cdrom'' by using a \#

\lstset{language=bash,caption=Removing CD-ROM Requirement for Updates}
\begin{lstlisting}
$ su -c "gedit /etc/apt/sources.list"
\end{lstlisting}

\item	Again, continue to edit /etc/apt/sources.list still with root privileges and ensure that each line ends with ``main 
		contrib non-free'', then save the file and do the following command with root privileges.

\lstset{language=bash,caption=Updating the System}
\begin{lstlisting}
$ su -c "apt-get update"
\end{lstlisting}

\item Next, configure the screen saver from the menu \verb|System -> Preferences -> Screensaver| and ensure that the following options are 
configured
\begin{itemize}
\item	Disable the option ``Lock screen when screensaver active''
\end{itemize}

\item 	The following instructions involve enabling headless support so that you can remote desktop to the machine without having a 
		monitor connected to the computer
\begin{itemize}
\item[a.] Edit the xorg.conf file and put the following in it

\lstset{language=bash,caption=Configuring Xorg for Headless Support}
\begin{lstlisting}
$ su -c "gedit /etc/X11/xorg.conf"

Section "Device"
Identifier "VNC Device"
Driver "vesa"
EndSection

Section "Screen"
Identifier "VNC Screen"
Device "VNC Device"
Monitor "VNC Monitor"
SubSection "Display"
Modes "1280x1024"
EndSubSection
EndSection

Section "Monitor"
Identifier "VNC Monitor"
HorizSync 30-70
VertRefresh 50-75
EndSection
\end{lstlisting}
	
\item[b.] Then edit grub and set the option ``nomodeset'', and proceed to update grub and reboot

\lstset{language=bash,caption=Configuring Grub for Headless Support}
\begin{lstlisting}
$ su -c "gedit /etc/default/grub"

GRUB_CMDLINE_LINUX="nomodeset"

$ su -c "update-grub"
\end{lstlisting}
\end{itemize}

\item	Now, reboot the machine, and ensure that the following work
\begin{itemize}
\item	It automatically boots up into the full desktop environment without having to login
\item	You have access to the machine using SSH and can login on the root account
\item	You have VNC access to the machine and can control the system using VNC	
\end{itemize}

\item	Finally, update the system from the menu \verb|System -> Administration -> Update Manager| and after it has 
		completed the updates reboot the system
\end{enumerate}




\subsection{Installing libvirt and virsh}
\label{sec:debianvirsh}
\begin{enumerate}
\item	The virtualization API libvirt and the command line tool virsh~\cite{libvirt} are the essential components required 
		for setting up a test client and must be installed and properly configured before any testing can begin. Ensure that
		you follow the proceeding directions carefully and validate that virsh is working properly before proceeding to 
		install and configure the various hypervisors.
		
\item	First, begin by reviewing the release news listed on the libvirt website, \url{http://libvirt.org/news.htm} and read 
		through the release notes for the latest version released to make sure that there are no regressions or deprecated 
		support for the platforms you wish to support. If you intend to set up an entire infrastructure and support all of the
		\cernvm virtualization platforms, which would include \emph{Xen, KVM, VirtualBox, and VMware}, then you must download
		a version later than \emph{0.8.7} as there was no support for VMware prior to that release.

\item	Next, download the latest release that is a {\bf src.rpm} file from the libvirt release server, 
		\url{http://libvirt.org/sources/} based on the latest release which does not have any regressions or deprecations for
		the virtualization platforms you wish to support~\footnote{This shouldn't be an issue but just in case there is a 
		newer version in which Xen support is deprecated, then you would need to use the last release which has Xen support}.
		As of this date, the latest release of libvirt is version 0.9.2, this is the release that will be used for the
		following instructions and examples.
		
\item	Next, install the following dependencies which are required to generate the libvirt rpm files from the src.rpm file
		that was downloaded, \emph{from now on execute all commands as root}.
		

% TODO: INSTRUCTIONS FOR COMPILING/INSTALLING LIBVIRT 0.9.2 ON DEBIAN SQUEEZE TEST CLIENTS

\item 	Finally, start the service libvirtd, and ensure that virsh installed correctly and is running by 
		connecting to the test hypervisor and ensuring that the test virtual machine, named ``test'' is running.

\lstset{language=bash,caption=Verify virsh was Installed Properly}
\begin{lstlisting}
# Change to root account, enter password if prompted
$ su

# Verify virsh is working, test should be running
$ service libvirtd start
$ virsh -c test:///default list --all
\end{lstlisting}
\end{enumerate}




\subsection{Installing and configuring KVM}
\label{sec:debiankvm}
\begin{enumerate}
\item	The first step is to install the KVM hypervisor, start by installing KVM and the other additional packages such
		as virt-manager, which is a graphical management tool and virt-install, which is a command line interface (CLI)
		virtual machine creation/installation/configuration tool using the following commands with root privileges. If
		you receive a message that a package is already installed then simply continue.

\lstset{language=bash,caption=Installing KVM and Other Related Programs}
\begin{lstlisting}
$  su -c "apt-get install qemu-kvm virt-manager"
\end{lstlisting}

\item	Next, verify that KVM has been installed properly and that virsh can connect to the KVM hypervisor using the
		following commands, if you are able to connect to the virsh console without any errors then virsh is able
		to connect to the KVM hypervisor.

\lstset{language=bash,caption=Verify that virsh can Access KVM}
\begin{lstlisting}
$ su
$ virsh -c qemu:///session
\end{lstlisting}

\item 	Finally, to ensure that KVM is properly configured and installed, follow this guide provided on the CernVM website
		\url{http://cernvm.cern.ch/portal/kvm} {\bf except, do not create a kvm definition file as the xml template file
		is provided by the test suite scripts} and verify that you are able to connect to the libvirtd kvm system session.
		
\lstset{language=bash,caption=Verify that KVM is Properly Configured}
\begin{lstlisting}
$ su
$ virsh -c qemu:///system 
\end{lstlisting}
\end{enumerate}



%\newpage
%\subsection{Setting up the Tapper Test Suite}
%\label{sec:debiantestsuite}
%\begin{enumerate}
%\item 	{\bf Before proceeding any further ensure that you have all other test clients set up this far, and then proceed
%		to follow the instructions for setting up and configuring the \tapper~server in the section ``Server Platform Setup''}		
%		~\ref{sec:serversetup}.
%\end{enumerate}