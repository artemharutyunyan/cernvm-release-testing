\section{Debian Based Test Client Setup}
\subsection{Installing the system}
\label{sec:debianinstall}
\flushleft
\begin{enumerate}
\item 	First, begin by loading the Debian Squeeze CD and select the first `install' option at the initial boot menu.

\item	For the first few installation options, such as default language and keyboard layout simply leave the default values
		and press enter unless you have different requirements than the default options.

\item	When prompted to enter the hostname for the computer, enter a relevant hostname for the machine based on the hardware or 
		operating system it is running; the hostname should be relevant and unique as \tapper will use this hostname to identify the 
		system. A good naming convention should refer to the hardware or operating system and call it a host to differentiate from the 
		virtual machine that will be running as a guest\footnote{This will make more sense later when you have to create and configure 
		CernVM images}, for example a hostname such as \emph{cernvm-debian6-host} could be used, {\bf whatever convention you use make 
		sure it is consistent}.
the following
\item	Next, you will be prompted to enter a domain name, delete the preset domain name ``local'' and leave it blank, \emph{unless
		your network environment has specific domain name requirements}.

\item	Next, when prompted to enter the root password, set the root password to something you will not forget and is
		fairly difficult with numbers and letters. {\bf Again, whatever password you use,  make sure it's consistent as you will 	
		frequently use this account}.

\item 	Next, when prompted to enter a username for the new user to be created on the system, set the username as something simple and 
		relevant such as \emph{cernvm} and the password to something you will not forget and is fairly complex with numbers and letters. 
		But, most importantly {\bf keep this username and password consistent across all systems created as part of the infrastructure} 
		as it makes administration and everything else much easier.

\item 	When prompted for the timezone simply configure the timezone according to your local time zone, such as CEST

\item	Now, when prompted by the installer to configure the partitioning layout, if there are other operating systems installed on the 
		system select the ``Guided - use entire disk'' option, and if available select the option ``Use Remaining Free Space''. 
		Otherwise, if there are no other operating system installed on the hard drive select the ``Manual'' option, \emph{beware that 
		doing so will risk erasing everything on the hard drive if you create a new partition table}. Using the manual option, create 
		two primary partitions, with the first taking up the size of the hard drive minus twice the size of the amount of RAM installed, 
		and the second primary partition as a SWAP file using the remaining free space. The following is an example of what the partiton 
		layout would look like for a 40.0 GB hard drive with 2GB of ram.

\lstset{language=bash,caption=Manual Partition Layout Example}
\begin{lstlisting}
#1	PRIMARY	36.0 GB		B f	 EXT4	/
#2	PRIMARY	4.0  GB		  f	 SWAP	SWAP
\end{lstlisting}

\item 	When prompted to scan another CD or DVD, select {\bf No} unless the system does not have access to the internet, as all the 
		packages will be installed using a network mirror.

\item	When prompted to use a network mirror, select {\bf Yes} and select a mirror near your location and in a country with a reputable 
		connection/internet infrastructure such as (USA, Netherlands, Germany, France).

\item 	When prompted to choose software to install, select the following
\begin{itemize}
\item 	Graphical desktop environment
\item 	SSH Server
\item	Standard system utilities
\end{itemize}

\item 	When prompted to install the GRUB boot loader to the master boot record, select {\bf Yes}

\item	After that the system should soon complete the installation, after the installation has finished ensure that you reboot the 
		system and remove the CD so that the system does not load the CD again when it starts.
\end{enumerate}


\newpage
\subsection{Configuring the system}
\label{sec:debianconfig}
\begin{enumerate}
\item	After the system has booted remove the follow unnecessary startup applications by selecting from the menu  
		\verb|System -> Preferences -> Startup Applications|
\begin{itemize}
\item	bluetooth
\item	evolution alarm
\item	Gnome Login Sound
\item	print queue
\item	screensaver
\item	update notifier
\item	visual assistance/aid
\item	volume control
\item	any others you think are unnecessary based on your own discretion
\end{itemize}

\item Remove the follow unnecessary services by selecting from the menu \verb|System -> Administration -> Services|
\begin{itemize}
\item	alsa utils
\item	bluetooth
\item	CUPS
\item	exim4
\item	any others you think are unnecessary based on your own discretion
\end{itemize}

\item Next enable and configure remote desktop from the menu \verb|System -> Preferences -> Remote Desktop| and ensure
that the following options are configured
\begin{itemize}
\item	Enable the option ``Allow others to view your desktop''
\item	Enable the option ``Allow other users to control your desktop''
\item	Disable the option ``You must confirm access to this machine''
\end{itemize}

\item Next configure the system to login automatically at boot from the menu select \verb|System -> Administration -> Login Screen|
and then set it to login to the user account you created previously (such as cernvm) automatically.

\item Next, remove cd-rom support from sources.list, which is used by Debian for updates \footnote{And is a nuisance for any new user
as it forces you to find the CD and put it in the computer for the update to continue}, execute the following command with root privileges
and comment out any lines that start with ``deb cdrom'' by using a \#

\lstset{language=bash,caption=Removing CD-ROM Requirement for Updates}
\begin{lstlisting}
$ su -c "gedit /etc/apt/sources.list"
\end{lstlisting}

\item	Again, continue to edit /etc/apt/sources.list still with root privileges and ensure that each line ends with ``main 
		contrib non-free'', then save the file and do the following command with root privileges.

\lstset{language=bash,caption=Updating the System}
\begin{lstlisting}
$ su -c "apt-get update"
\end{lstlisting}

\item Next, configure the screen saver from the menu \verb|System -> Preferences -> Screensaver| and ensure that the following options are 
configured
\begin{itemize}
\item	Disable the option ``Lock screen when screensaver active''
\end{itemize}

\item 	The following instructions involve enabling headless support so that you can remote desktop to the machine without having a 
		monitor connected to the computer
\begin{itemize}
\item[a.] Edit the xorg.conf file and put the following in it

\lstset{language=bash,caption=Configuring Xorg for Headless Support}
\begin{lstlisting}
$ su -c "gedit /etc/X11/xorg.conf"

Section "Device"
Identifier "VNC Device"
Driver "vesa"
EndSection

Section "Screen"
Identifier "VNC Screen"
Device "VNC Device"
Monitor "VNC Monitor"
SubSection "Display"
Modes "1280x1024"
EndSubSection
EndSection

Section "Monitor"
Identifier "VNC Monitor"
HorizSync 30-70
VertRefresh 50-75
EndSection
\end{lstlisting}
	
\item[b.] Then edit grub and set the option ``nomodeset'', and proceed to update grub and reboot

\lstset{language=bash,caption=Configuring Grub for Headless Support}
\begin{lstlisting}
$ su -c "gedit /etc/default/grub"

GRUB_CMDLINE_LINUX="nomodeset"

$ su -c "update-grub"
\end{lstlisting}
\end{itemize}

\item	Now, reboot the machine, and ensure that the following work
\begin{itemize}
\item	It automatically boots up into the full desktop environment without having to login
\item	You have access to the machine using SSH and can login on the root account
\item	You have VNC access to the machine and can control the system using VNC	
\end{itemize}

\item	Finally, update the system from the menu \verb|System -> Administration -> Update Manager| and after it has 
		completed the updates reboot the system
\end{enumerate}


\newpage
\subsection{Installing and configuring the hypervisors}
\label{sec:debianhypervisor}
\begin{enumerate}
\item The first step is to install the hypervisor, start by installing kvm and virsh using the following command with root privileges

\lstset{language=bash,caption=Installing KVM and Virsh Support}
\begin{lstlisting}
$ su 	-c "apt-get install qemu-kvm libvirt-bin"
\end{lstlisting}

\item 	Now, proceed to download and extract the desired kvm virtual machine image onto the system from the CernVM download 		
		portal\footnote{The images can be obtained here: \url{http://cernvm.cern.ch/portal/downloads} it is recommended that you 
		download the KVM Basic image for your architecture}, for this guide the basic image will be downloaded as it is the most 
		practical image for the majority of users.

\lstset{language=bash,caption=Download and Extract CernVM KVM Basic Image}
\begin{lstlisting}
$ wget http://rbuilder.cern.ch/downloadImage?fileId=1719
$ gunzip cernvm*.gz
\end{lstlisting}

\item 	Now, to ensure that KVM is properly configured and installed, follow this guide provided on the CernVM website
		\url{http://cernvm.cern.ch/portal/kvm} {\bf except, instead use following virtual machine definition file to 
		create the virtual machine}. \emph{You will need to change the following XML tags in the configuration file accordingly}.
		
\begin{itemize}
\item \verb|<uuid>| by generating a uuid using the uuid command
\item \verb|<source file>| according to wherever you extracted the cernvm kvm image
\item \verb|<mac address>| not a necessity, but change it to something slightly different
\end{itemize}

\lstset{language=bash,caption=Create CernVM KVM Definition File}
\begin{lstlisting}
# Install uuid tool and generate uuid
$ su -c "apt-get install uuid"
$ uuid

# Create the cernvm.xml definition file and set the XML tags accordingly
$ gedit cernvm.xml

<domain type='kvm'>
  <name>cernvm</name>
  <uuid>b32147e7-9b89-dda9-b15d-53ba5f54f590</uuid>
  <memory>524288</memory>
  <currentMemory>524288</currentMemory>
  <vcpu>1</vcpu>
  <os>
    <type arch='x86_64' machine='pc-0.12'>hvm</type>
    <boot dev='hd'/>
  </os>
  <features>
    <acpi/>
    <apic/>
    <pae/>
  </features>
  <clock offset='utc'/>
  <on_reboot>restart</on_reboot>
  <on_crash>restart</on_crash>
  <devices>
    <emulator>/usr/bin/kvm</emulator>
    <disk type='file' device='disk'>
      <driver name='qemu' type='raw'/>
      <source file='/home/cernvm/image/cernvm-2.3.0-x86_64.hdd'/>
      <target dev='hda' bus='ide'/>
      <address type='drive' controller='0' bus='0' unit='0'/>
    </disk>
    <controller type='ide' index='0'>
      <address type='pci' domain='0x0000' bus='0x00' slot='0x01' 
      function='0x1'/>
    </controller>
    <interface type='network'>
      <mac address='52:54:00:ca:d5:d3'/>
      <source network='default'/>
      <target dev='vnet0'/>
      <address type='pci' domain='0x0000' bus='0x00' slot='0x03' 
      function='0x0'/>
    </interface>
    <serial type='pty'>
      <target port='0'/>
    </serial>
    <console type='pty'>
      <target type='serial' port='0'/>
    </console>
    <input type='mouse' bus='ps2'/>
    <graphics type='vnc' port='-1' autoport='yes'/>
    <video>
      <model type='cirrus' vram='9216' heads='1'/>
      <address type='pci' domain='0x0000' bus='0x00' slot='0x02'
      function='0x0'/>
    </video>
    <memballoon model='virtio'>
      <address type='pci' domain='0x0000' bus='0x00' slot='0x04'
      function='0x0'/>
    </memballoon>
  </devices>
</domain>
\end{lstlisting}

\item	Finally, configure the the virtual machine network to automatically start when the system boots, then create 
		the virual machine and ensure that it can be started and stopped using virsh, \emph{all virsh commands require
		root privileges, so it easiest to simply run as as root}.

\lstset{language=bash,caption= Configure the Virtual Machine and Verify it Works}
\begin{lstlisting}
$ su
$ virsh net-autostart default
$ virsh define /home/cernvm/image/cernvm.xml

# Verify the virtual machine was added, should be in list
$ virsh list --all

# Verify the virtual machine can be turned on/off
$ virsh start cernvm
# Wait about 2 minutes for the system to boot
$ virsh shutdown cernvm
\end{lstlisting}
\end{enumerate}


\newpage
\subsection{Configuring the CernVM Image}
\label{sec:cernvmconfig}
\begin{enumerate}
\item 	The next steps involve configuring the CernVM image to integrate with virsh as well as the test suite, first
		start the virtual machine, execute the command to get the IP address of the CernVM image. Then follow this guide
		provided on the CernVM website on how to configure the CernVM image and create a new user using the web interface 
		\url{http://cernvm.cern.ch/portal/cvmconfiguration} and reboot the system, \emph{all virsh commands require root privileges}.

\begin{lstlisting}
$ virsh start cernvm
# Wait about 2 minutes for the system to boot 
# Then get IP Address using the following command
$ arp -an
\end{lstlisting}

\item Now that the system has booted, login in to the system using SSH for the user you created using the CernVM web interface
\begin{lstlisting}
$ ssh <user you created>@cernvm-image-ipaddress
\end{lstlisting}

\item	Now that you are logged into the system through SSH, execute the following instructions to enable virsh console access.
		
\lstset{language=bash,caption= Enable Virsh Console Access}
\begin{lstlisting}
# Type the following command, and enter a root password you won't forget
$ sudo passwd root
# Change to the root account and enable console for root so that you can 
# login using virsh console
$ su

# Enable root login on tty
$ echo “ttyS0″ >> /etc/securetty

# Then add console=ttyS0 to the kernel parameter line in /etc/grub.conf
$ vi /etc/grub.conf

# Add getty to /etc/inittab file after all the other "tty" lines, 
# add the following line to /etc/inittab
s0:2345:respawn:/sbin/agetty -L 38400 ttyS0 vt100

\end{lstlisting}

\item	Now, reboot and login to the system using virsh console and then simply type ``root'' and enter the password to login as the 
		root account, \emph{there will most likely not be any console display from the virtual machine until you press 
		enter after entering the password}, then enable root login through SSH using the following commands. {\bf To 
		disconnect from the virsh console and return to the host machine console, press CTRL + ]  which is ( \^~] )}
		
\lstset{language=bash,caption= Enable SSH Root Login}
\begin{lstlisting}
# Reboot the system and the console should now work
# Wait for it to completely shut off and turn on
$ virsh shutdown cernvm
$ virsh start cernvm

# Login as root using password you set with the passwd command
$ virsh console cernvm

# edit the file /etc/ssh/sshd_config and uncomment the line 
# PermitRootLogin yes	in order to enable root login
$ vi /etc/ssh/sshd_config
\end{lstlisting}

\item 	Next from the host machine \emph{ie. the machine you're currently using} enable automatic login as root 
		through ssh on the KVM guest, first ensure that the guest machine has been started.

\lstset{language=bash,caption=Enable Automatic SSH Root Login}
\begin{lstlisting}
# Restart the virtual machine wait for 
# it to completely shut off and turn on
$ virsh shutdown cernvm
$ virsh start cernvm

# Generate a public key, when prompted press enter for everything
$ ssh-keygen -t rsa

# Get the ip address of the running cernvm guest, as done previously
$ arp -an

# Next, run the following command to setup automatic login for ssh
# without having to type the password. When prompted for the password
# enter the root password you just set previously
$ ssh-copy-id -i ~/.ssh/id_rsa.pub root@cernvm-image-ipaddress
		
# Disconnect, and try and login to the machine using ssh and ensure 
# that you can login as root without having to type the password
$ ssh root@cernvm-image-ipaddress
\end{lstlisting}

\item	If everything so far has worked, then the test client and CernVM image have been installed and configured properly,
		if you have any outstanding issues solve them before proceeding further, or go to the section ``Server Platform 		
		Setup''~\ref{sec:serversetup} as the \tapper~server does not require virtual machine creation and configuration.
\end{enumerate}


\newpage
\subsection{Setting up the Tapper Test Suite}
\label{sec:debiantestsuite}
\begin{enumerate}
\item 	{\bf Before proceeding any further ensure that you have all other test clients set up this far, and then proceed
		to follow the instructions for setting up and configuring the \tapper~server in the section ``Server Platform Setup''}		
		~\ref{sec:serversetup}.
\end{enumerate}