\section{OS X Test Client Setup}
\subsection{Configuring the system}
\label{sec:osxconfig}

\begin{enumerate}
\item	After the system has booted, the first thing to configure are the power management settings and other preferences, as this system
		will be running as a test client, sleep and other automatic energy saving features must be disabled. Begin by navigating to the
		power options, \verb|Apple logo -> System Preferences -> Energy Saver| for the option ``Computer sleep'' slide the bar to the
		far right so that it is set to ``Never'' and ensure that the following options are all disabled.
\begin{itemize}
\item	Put the hard disk(s) to sleep when possible

\item	Wake for Ethernet network access

\item 	Allow power button to put the computer to sleep
\end{itemize}

\item	Next, set a hostname for the system from the menu \verb|Apple logo -> System Preferences -> Sharing| beside the ``Computer Name:''
		option at the top, click the ``Edit...'' button. Then enter a relevant hostname for the system based on the hardware or operating
		system it is running; the hostname should be relevant and unique to better identify the system. A good naming convention should 
		refer to the hardware or operating system and call it a host to differentiate from the virtual machine that will be running as 
		a guest, for example a hostname such as \emph{cernvm-osx-host} could be used, {\bf whatever convention you use make sure it is 
		consistent}.

\item	Next, enable SSH access to the system by navigating to \verb|Apple logo -> System Preferences -> Sharing| and from the list of 
		services that can be shared, enable ``Remote Login'', which is SSH.
		
\item	Now, to enable VNC access to the system, select from the same list of services that can be shared, ``Remote Management'' and
		for local access options window that appears, enable all of the options listed such as ``Observe'' and ``Change settings''.
		Then to enable VNC compatibilty so that the OS X system can be accessed by other non-Apple computers, click the ``Computer
		Settings...'' button and enable the following options and set a password for the ``VNC viewers...'' option.
\begin{itemize}
\item	Show Remote Management status in menu bar
\item	Anyone may request permission to control screen
\item	VNC viewers may control screen with password
\end{itemize}

\item	Now, to ensure that your user logs in automatically, navigate to \verb|Apple logo -> System Preferences -> Accounts| and click
		``Login Options'', you may have to click on the lock icon and enter your password in order to make changes to the login options. 
		Then for the option ``Automatic login:'' select your user from the list of accounts to enable automatic login.
		
\item 	Finally, to ensure that the settings were configured properly, reboot the machine and ensure that the following work.
\begin{itemize}
\item	It automatically boots up into the full desktop environment without having to login
\item	You have access to the machine using SSH and can login
\item	You have VNC access to the machine and can control the system using VNC	
\end{itemize}
\end{enumerate}




\newpage
\subsection{Installing libvirt and virsh}
\label{sec:osxvirsh}
\begin{enumerate}
\item	The virtualization API libvirt and the command line tool virsh~\cite{libvirt} are the essential components required 
		for setting up a test client and must be installed and properly configured before any testing can begin. Ensure that
		you follow the proceeding directions carefully and validate that virsh is working properly before proceeding to 
		install and configure the various hypervisors.

\item	First, to build libvirt from source Xcode must be installed, \url{http://developer.apple.com/xcode/}, Xcode 4 requires 
		either a paid developer membership, or must be purchased from the App Store; but, Xcode 3 is freely available, {\bf this 
		guide uses Xcode 3 to build libvirt from source}.

\item	Next, review the release news listed on the libvirt website, \url{http://libvirt.org/news.htm} and read 
		through the release notes for the latest version released to make sure that there are no regressions or deprecated 
		support for the platforms you wish to support. If you intend to set up a test client which supports VMware then you must 
		download a version later than \emph{0.8.7} as there was no support for VMware prior to that release.

\item	Next, download the latest source release that is a {\bf tar.gz} file from the libvirt release server, 
		\url{http://libvirt.org/sources/} based on the latest release which does not have any regressions or deprecations for
		the virtualization platforms you wish to support. As of this date, the latest release of libvirt is version 0.9.2, this 
		is the release that will be used for the following instructions and examples.
		
\item	Next, download and install the following dependencies which are required by libvirt, some of the newer releases are broken
		on OS X so the follow versions must be used. Before issuing make, configure each with \emph{--prefix=/usr/local} and for
		libgcrypt also add \emph{--disable-asm}.

\begin{itemize}
\item	libgpgerror-1.7	-	ftp://ftp.gnupg.org/gcrypt/libgpg-error/libgpg-error-1.7.tar.gz
\item	libgcrypt-1.4.5 -	ftp://ftp.gnupg.org/gcrypt/libgcrypt/libgcrypt-1.4.5.tar.gz
\item	gnutls-2.8.5 	-	ftp://ftp.gnu.org/pub/gnu/gnutls/gnutls-2.8.5.tar.bz2
\end{itemize}

\item	Next, configure and install libvirt using the following commands, which builds libvirt with support for VMware and
		VirtualBox.

\lstset{language=bash,caption=Configure and Install libvirt}
\begin{lstlisting}
# Configure libvirt with VMware/VirtualBox support
./configure --without-xen --without-sasl --without-avahi \
--without-polkit --without-qemu --without-lxc --without-openvz \
--without-remote --with-libvirtd --without-uml --with-vmware --with-vbox

# Build/install libvirt
sudo make
sudo make install
\end{lstlisting}	

\item 	Finally, ensure that virsh installed correctly and is running by connecting to the test hypervisor and ensuring 
		that the test virtual machine, named ``test'' is running.

\lstset{language=bash,caption=Verify virsh was Installed Properly}
\begin{lstlisting}
# Verify virsh is working, test should be running
$ virsh -c test:///default list --all
\end{lstlisting}
\end{enumerate}




\newpage
\subsection{Installing and configuring VirtualBox}
\label{sec:osxvbox}
\begin{enumerate}
\item	First, begin by downloading and installing the latest version of VirtualBox for OS X from the VirtualBox download 
		page, \url{http://www.virtualbox.org/wiki/Downloads} ensure that you select the appropriate architecture for your 
		system. The following instructions for this section of the guide uses VirtualBox 4.0.10 for AMD64.
		
\item	Next, to verify that VirtualBox has been installed properly and that virsh can connect to the VirtualBox hypervisor, 
		verify that the following VirtualBox kernel extensions are loaded and that you are able to connect to the virsh console 
		without any errors.
		
\begin{itemize}
\item	org.virtualbox.kext.VBoxDrv
\item	org.virtualbox.kext.VBoxUSB
\item	org.virtualbox.kext.VBoxNetFlt
\item	org.virtualbox.kext.VBoxNetAdp
\end{itemize}

\lstset{language=bash,caption=Verify that virsh can Access VirtualBox}
\begin{lstlisting}
# Verify that the kernel extentsions are loaded
$ kextstat | grep -i virtualbox

# Verify that virsh can connect to virtualbox
$ virsh -c vbox:///session
\end{lstlisting}

\item	Now, proceed to download and extract the desired VirtualBox virtual machine image onto the system from the \cernvm download 
		portal, \url{http://cernvm.cern.ch/portal/downloads} it is recommended that you download the VirtualBox Desktop image for your 
		architecture. For this guide the Desktop image will be downloaded as it is the most practical image for the majority of 
		users.

%\item  TODO: ADD INSTRUCTIONS ON EXECUTING THE SCRIPT WHICH WILL AUTOMATICALLY GENERATE AN XML DEFINTION FILE FOR THE VIRTUAL MACHINE
%		WITHOUT HAVING TO MANUALLY (AND TEDIOUSLY) GO THROUGH AND CONFIGURE THE VIRTUAL MACHINE AND CREATE AN XML FILE MANUALLY...
		

% THIS IS AUTOMATED ANYWAYS AND PART OF THE PRECONDITION TESTS, IT'S STILL GOOD TO MANUALLY MAKE SURE VIRTUAL IMAGE ACTUALLY WORKS
\item	Now that the virtual machine has been created and configured verify that it is able to boot completely without crashing,
		\emph{you will be presented with a login screen when it has booted completely}. Then shutdown the virtual machine by clicking 
		``Actions'' from within the virtual machine and selecting `Shutdown'', after the virtual machine has shutdown close
		VirtualBox and then connect to the VirtualBox hypervisor and determine that you can view, start, and stop virtual machine.

\lstset{language=bash,caption=Verify VirtualBox Works with Virsh}
\begin{lstlisting}
$ virsh --connect vbox:///session

# Verify the virtualbox virtual machine is accessible
# Name of the virtual machine created should be listed
$ list --all

# Verify the virtual machine can be turned on/off
$ start <name of virtual machine>
# Wait about 2 minutes for the system to boot
$ shutdown <name of virtual machine>
\end{lstlisting}

\item	Finally, configure the the virtual machine network to automatically start when the system boots.

\lstset{language=bash,caption=Configure Network}
\begin{lstlisting}
# Configure virsh network for VirtualBox
$ virsh --connect vbox:///session
$ net-start vboxnet0
$ net-autostart vboxnet0
\end{lstlisting}				
\end{enumerate}




\newpage
\subsection{Installing and configuring VMware}
\label{sec:osxvmware}
\begin{enumerate}
\item	First, begin by downloading and installing the latest version of VMware Fusion for OS X from the VMware product 
		page, \url{http://www.vmware.com/products/}, VMware Fusion requires a license, so you will have to purchase it
		in order to continue.
		
% VMWARE FUSION DOES NOT WORK WITH VIRSH, WHEN YOU CONNECT TO THE VIRSH CONSOLE VMWARE WINDOW LOADS IN AND YOU CANNOT
% EXECUTE ANY COMMANDS FROM WITHIN VIRSH, ONLY WAY TO FIX IT IS FOR ME TO START HACKING THE LIBVIRT CODE AND SUBMIT
% CHANGES BACK UPSTREAM ON THE MAILING LISTS, WHICH REQUIRES A LOT OF WORK AND READING THROUGH THE DEVELOPERS MANUAL
\item	Next, to verify that VMware Fusion has been installed properly, verify that the following VMware kernel extensions 
		are loaded, currently virsh has support to connect to the VMware hypervisor, but does not support interacting with
		VMware through VMware Fusion.
		
\begin{itemize}
\item	com.vmware.kext.vmx86
\item	com.vmware.kext.vmci
\item	com.vmware.kext.vmioplug
\item	com.vmware.kext.vmnet
\end{itemize}

\lstset{language=bash,caption=Verify VMware Kernel Extensions Loaded}
\begin{lstlisting}
# Verify that the kernel extentsions are loaded
$ kextstat | grep -i vmware
\end{lstlisting}

\item	Now, proceed to download and extract the desired VMware virtual machine image onto the system from the \cernvm download 
		portal, url{http://cernvm.cern.ch/portal/downloads} it is recommended that you download the VMware Desktop image for your 
		architecture. For this guide the Desktop image will be downloaded as it is the most practical image for the majority of 
		users.

%\item  TODO: ADD INSTRUCTIONS ON EXECUTING THE SCRIPT WHICH WILL AUTOMATICALLY GENERATE AN XML DEFINTION FILE FOR THE VIRTUAL MACHINE
%		WITHOUT HAVING TO MANUALLY (AND TEDIOUSLY) GO THROUGH AND CONFIGURE THE VIRTUAL MACHINE AND CREATE AN XML FILE MANUALLY...

\end{enumerate}




%\newpage
%\subsection{Installing and configuring the hypervisors}
%\subsection{Configuring the CernVM Image}
%\subsection{Setting up the Tapper Test Suite}