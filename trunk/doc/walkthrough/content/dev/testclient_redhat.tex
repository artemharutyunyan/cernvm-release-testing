\section{Red Hat Based Test Client Setup}
\subsection{Configuring the system}
\label{sec:rhconfig}
\begin{enumerate}
\item 	After the system has booted remove the follow unnecessary startup applications by selecting from the menu  
		\verb|System -> Preferences -> Startup Applications|
\begin{itemize}
\item	bluetooth
\item	evolution alarm
\item	Gnome Login Sound
\item	PackageKit Update Applet
\item	print queue
\item	screensaver
\item	visual assistance/aid
\item	volume control
\item	any others you think are unnecessary based on your own discretion
\end{itemize}

\item	Next enable and configure the remote desktop from the menu \verb|System -> Preferences| \verb| -> Remote Desktop| 
		and ensure that the following options are configured	
\begin{itemize}
\item	Enable the option ``Allow others to view your desktop''
\item	Enable the option ``Allow other users to control your desktop''
\item	Disable the option ``You must confirm access to this machine''
\end{itemize}
			
\item	Next enable SSH access to the machine, in order for SSH and VNC access to work the firewall will have to be disabled
\begin{itemize}
\item[a.]	First disable the firewall from the menu \verb|System -> Adminisration -> Firewall| and click the ``Disable'' 
			button and then click ``Apply'' to apply the changes! {\bf This is a quick solution for now because it's too 
			much work to configure the firewall for VNC, SSH, Apache, MySQL, PHPMyAdmin, MCP, and all the other network 
			daemons and should not be a problem if this is just being accessed internally}.

\item[b.]	Now that the firewall is disabled, configure sshd, the ssh daemon, to run on startup

\begin{lstlisting}
$ su -c "chkconfig --level 345 sshd on"
\end{lstlisting}
\end{itemize}

\item	Next, configure the system to login automatically at boot
\begin{itemize}
\item[a.]	Edit the login screen configuration file for gdm using the following command
\begin{lstlisting}
$ su -c "gedit /etc/gdm/custom.conf"
\end{lstlisting}

\item[b.]	Then in the custom.conf file, put the following under the heading [daemon], which will automatically
			log the system in as the user you created, \emph{make sure you replace the user cernvm with the user
			that you created}.

\lstset{language=bash,caption=Configure Automatic Login}
\begin{lstlisting}
AutomaticLoginEnable=true
AutomaticLogin=cernvm
\end{lstlisting}		
\end{itemize}

\item	Next, configure the screen saver from the menu \verb|System -> Preferences -> Screensaver| and ensure that
		the following options are configured
\begin{itemize}
\item	Disable the option ``Lock screen when screensaver active''
\end{itemize}

\item	Now, reboot the machine, and ensure that the following work
\begin{itemize}
\item	It automatically boots up into the full desktop environment without having to login
\item	You have access to the machine using SSH and can login on the root account
\item	You have VNC access to the machine and can control the system using VNC	
\end{itemize}

\item	Finally, update the system from the menu \verb|System -> Administration -> Software Update| and after it has 
		completed the updates reboot the system
\end{enumerate}




\subsection{Installing libvirt and virsh}
\label{sec:rhvirsh}
\begin{enumerate}
\item	The virtualization API libvirt and the command line tool virsh~\cite{libvirt} are the essential components required 
		for setting up a test client and must be installed and properly configured before any testing can begin. Ensure that
		you follow the proceeding directions carefully and validate that virsh is working properly before proceeding to 
		install and configure the various hypervisors.
		
\item	First, begin by reviewing the release news listed on the libvirt website, \url{http://libvirt.org/news.htm} and read 
		through the release notes for the latest version released to make sure that there are no regressions or deprecated 
		support for the platforms you wish to support. If you intend to set up an entire infrastructure and support all of the
		\cernvm virtualization platforms, which would include \emph{Xen, KVM, VirtualBox, and VMware}, then you must download
		a version later than \emph{0.8.7} as there was no support for VMware prior to that release.

\item	Next, download the latest release that is a {\bf tar.gz} file from the libvirt release server, 
		\url{http://libvirt.org/sources/} based on the latest release which does not have any regressions or deprecations for
		the virtualization platforms you wish to support~\footnote{This shouldn't be an issue but just in case there is a 
		newer version in which Xen support is deprecated, then you would need to use the last release which has Xen support}.
		As of this date, the latest release of libvirt is version 0.9.2, this is the release that will be used for the
		following instructions and examples.
		
\item	Next, install the following dependencies which are required to build libvirt from source, \emph{from now on execute all 
			commands as root}.
		
\lstset{language=bash,caption=Install Dependencies}
\begin{lstlisting}
# Change to root account, enter password if prompted
$ su

# Install dependencies for building from source
$ yum install gnutls-devel numactl numactl-devel python-devel libnl-devel \
libxml2-devel device-mapper-libs device-mapper-devel

# Install GCC
$ yum install gcc make
\end{lstlisting}

\item	Next, install software for managing and viewing virtual machines, \emph{it is imperative that these applications are
			installed before compiling and installing libvirt otherwise there will be conflicts}.

\lstset{language=bash,caption=Install Virtual Machine Management Software}
\begin{lstlisting}
# Change to root account, enter password if prompted
$ su

# Install management software
$ yum install virt-manager python-virtinst virt-viewer
\end{lstlisting}
	
\item	Next extract the files and execute configure with the following options, then finally compile and install libvirt.

\lstset{language=bash,caption=Compile and Install libvirt}
\begin{lstlisting}
# Change to root account, enter password if prompted
$ su

# Extract and execute configure
$ tar -xvvzf libvirt-*.tar.gz
$ cd libvirt-*/
$	./configure --prefix=/usr --disable-silent-rules --disable-shared \
--enable-static --enable-dependency-tracking --with-qemu \
--with-vmware --with-libssh2 --with-vbox --with-test --with-remote \
--with-libvirtd --with-numactl --with-network --with-storage-dir

# Compile and install libvirt
$ make 
$ make install	
\end{lstlisting}
	
\item 	Finally, verify that the the service libvirtd can start, because it is required for libvirt to function,  and then
			configure libvirtd to start automatically at boot. As well, ensure that virsh installed correctly and is running by 
			connecting to the test hypervisor and ensuring that the test virtual machine, named ``test'' is running.

\lstset{language=bash,caption=Verify virsh was Installed Properly}
\begin{lstlisting}
# Change to root account, enter password if prompted
$ su

# Verify libvirtd starts, set to start automatically at boot
$ service libvirtd start
$ chkconfig --level 2345 libvirtd on

# Verify vitual machine "test" running
$ virsh -c test:///default list --all
\end{lstlisting}
\end{enumerate}




\subsection{Installing and configuring KVM}
\label{sec:rhkvm}
\begin{enumerate}
\item	The first step is to start by installing the KVM package, KVM may have alrready been installed by default, If
			you receive a message that a package is already installed then simply continue.

\lstset{language=bash,caption=Install KVM}
\begin{lstlisting}
$ yum install qemu-kvm
\end{lstlisting}

\item	Next, verify the location of the qemu-kvm binary, at the moment virsh still expects that the binary is located
			in \verb|/usr/bin|, but since Red Hat 6, the binary has been moved into the directory \verb|/usr/libexec|.
			To resolve the issue simply create a symbolic link to the binary if qemu-kvm does not exist.
			
\lstset{language=bash,caption=Symbolic Link to qemu-kvm}
\begin{lstlisting}
# If qemu-kvm does not exist create symbolic link
$ ln -s /usr/libexec/qemu-kvm /usr/bin/qemu-kvm
\end{lstlisting}

\item	Next, verify that KVM has been installed properly and that virsh can connect to the KVM hypervisor using the
			following commands, if you are able to connect to the virsh console without any errors then virsh is able
			to connect to the KVM hypervisor.

\lstset{language=bash,caption=Verify that virsh can Access KVM}
\begin{lstlisting}
$ su
$ virsh -c qemu:///session
\end{lstlisting}

\item 	Next, to ensure that KVM is properly configured and installed, follow this guide provided on the CernVM website
			\url{http://cernvm.cern.ch/portal/kvm} {\bf except, do not create a kvm definition file as the xml template file
			is provided by the test suite scripts} and verify that you are able to connect to the libvirtd kvm system session.
		
\lstset{language=bash,caption=Verify that KVM is Properly Configured}
\begin{lstlisting}
$ su
$ virsh -c qemu:///system 
\end{lstlisting}

\item	Finally, ensure that you are able to connect to the QEMU/KVM hypervisor using the virtual machine manager, as it 
			is necessary to view the status of the CernVM images and must be installed to troubleshoot and view the CernVM 
			images. Simply launch the virtual machine manager application from 
			\verb|Applications -> System Tools -> Virtual Machine Manager| and{\bf if you are prompted to install libvirt, 
			select ``No'' } as a custom version of libvirt was installed previously.
\end{enumerate}




\subsection{Installing and configuring VirtualBox}
\label{sec:rhvbox}
\begin{enumerate}
\item	First, begin by downloading and installing a version of VirtualBox supported by libvirt from the VirtualBox download 
			page, it is best to download the latest version within the series \emph{that has been available for at least a month 
			prior to the release of the version of libvirt installed}. VirtualBox can be downloaded from the following location,  
			\url{http://www.virtualbox.org/wiki/Downloads} ensure that you select the appropriate Red Hat based distribution, 
			version and architecture for your system. The following instructions for this section of the guide uses VirtualBox 
			4.0.12 for Red Hat 6, AMD64.
		 	
\item	Before installing VirtualBox, install the dependencies required to build the VirtualBox kernel modules.

\lstset{language=bash,caption=Install VirtualBox Dependencies}
\begin{lstlisting}
$ su
$ yum install kernel-headers kernel-devel
\end{lstlisting}
		 	
\item	Next, after downloading the latest version of VirtualBox for your distribution and installing the dependencies 
			install VirtualBox as the root account using the following command.
			
\begin{lstlisting}
# Enter the root password when prompted
$ su
$ rpm -iv VirtualBox-*.rpm
\end{lstlisting}	

\item	Next, in order to use VirtualBox and have full access to the drivers needed, ensure that the root
			account belongs to the group ``vboxusers''. Add the root account to the group ``vboxusers'' using
			the following command.

\lstset{language=bash,caption=Add root to vboxusers}
\begin{lstlisting}
$ su -c "usermod -a -G vboxusers root"
\end{lstlisting}
		
\item	Due to an issue with VirtualBox\footnote{The issues is that VirtualBox looks for virtual machine configuration files (*.vbox)
		in the ``VirtualBox VMs'' folder of the user that launched VirtualBox. The issue is worsened by the fact that there can
		only be one ``VirtualBox VMs'' folder which causes conflicts with multiple users.}, in order for it to work with virsh 
		the virtual machine(s) must be created and configured as the root account, otherwise when you try to connect or start a 
		VirtualBox virtual machine with virsh you will get an ``unknown error'', which is obviously very vague and difficult to 
		resolve. {\bf Therefore ALWAYS start VirtualBox as the root account using the following procedure}.

\lstset{language=bash,caption=Always Start VirtualBox as Root}
\begin{lstlisting}
# Switch to the root account, enter root password
$ su

# Start VirtualBox as root
$ virtualbox
\end{lstlisting}

\item	Finally, verify that VirtualBox has been installed properly and that virsh can connect to the VirtualBox hypervisor, 
		verify that the VirtualBox module, \emph{vboxdrv} has been loaded and that you are able to connect to the virsh console 
		without any errors.

\lstset{language=bash,caption=Verify that virsh can Access VirtualBox}
\begin{lstlisting}
$ su

# Verify that the vboxdrv module is loaded
$ lsmod | grep -i vboxdrv

# Verify that virsh can connect to virtualbox
$ virsh -c vbox:///session
\end{lstlisting}
\end{enumerate}




\subsection{Installing and configuring VMware}
\label{sec:rhvmware}
\begin{enumerate}
\item	First, begin by downloading the latest version of VMware Workstation or VMware Player  from the VMware product 
			page, \url{http://www.vmware.com/products/}, VMware Player is free, whereas VMware Workstation requires a
			license. So if you decide to use VMware Workstation instead of VMware Player you will have to purchase a license
			for it in order to continue.

\item	Before installing VMware, install the dependencies required to build the VMware kernel modules.

\lstset{language=bash,caption=Install VMware Dependencies}
\begin{lstlisting}
$ su
$ yum install kernel-headers kernel-devel
\end{lstlisting}
		
\item 	Next, to install VMware simply set the bundle file as executable and execute the file as root.
\lstset{language=bash,caption=Install VMware}
\begin{lstlisting}
$ su
$ chmod +x VMware*.bundle
$ ./VMware*.bundle
\end{lstlisting}

\item	Next, launch VMware as root and wait for it to compile the kernel modules, then verify that the following VMware 
			kernel modules are loaded, currently virsh has support to connect to the VMware hypervisor, but there are some minor
			issues such as a lack of support for VMware network configurations, currently only ``bridged'' mode is supported.
		
\begin{itemize}
\item        vmnet
\item        vmblock
\item        vmci
\item        vmmon

\end{itemize}

\lstset{language=bash,caption=Verify VMware Kernel Modules Loaded}
\begin{lstlisting}
# Launch VMware as root, this will build kernel modules
$ su
$ vmware

# Verify that the kernel extentsions are loaded
$ lsmod | grep -i vm
\end{lstlisting}

.\item      Finally, verify that VMware has been configured properly and that virsh supports the current version of VMware 
                installed by connecting to the virsh console for the VMware hypervisor.

\lstset{language=bash,caption=Verify VMware Works with Virsh}
\begin{lstlisting}
# Verify that virsh can connect to vmware
$ virsh --connect vmwarews:///session
\end{lstlisting}

\item	If everything so far has worked, then libvirt, virsh, and the hypervisors have been installed and configured properly,
			if you have any outstanding issues solve them before proceeding further, or go to the section ``Server Platform 		
			Setup''~\ref{sec:serversetup} as the \tapper~server does not require libvirt, virsh, or hypervisor configuration.
		
\end{enumerate}




\subsection{Setting up the Tapper Test Suite}
\label{sec:rhtestsuite}
\begin{enumerate}
\item 	{\bf Before proceeding any further ensure that you have all other test clients set up this far, and then proceed
		to follow the instructions for setting up and configuring the \tapper~server in the section ``Server Platform Setup''}		
		~\ref{sec:serversetup}.
		
\item 	Now that the \tapper~server has been installed and configured and the \tapper web interface and database have proven
		to be working, the next step is to verify that the test client can actually send a report to the \tapper~server in
		the form of a TAP file. After sending the TAP report to the server, ensure that the test client is working by viewing 
		the tapper reports in your browser at the following url: \url{http:/<tapper\_server>/tapper/reports}. You should now see a 
		report from the test client, there should be a report from a system named whatever the ``Tapper-Machine-Name'' in 
		demo\_report.tap was set as. \emph{For the example demo\_report.tap provided below it would be cernvm-rhtestclient}.
		\footnote{This is why a consistent hostname convention was emphasized earlier, as reports are often sorted and organized 
		based on hostnames}.
		
\lstset{language=bash,caption=Send a Basic Report to the \tapper~Server}
\begin{lstlisting}
# Save the following in a file named demo_report.tap

	1..2
	# Tapper-Suite-Name: Tapper-Deployment
	# Tapper-Suite-Version: 1.001
	# Tapper-Machine-Name: cernvm-rhtestclient
	ok - Hello World
	ok - Just another description

# Send the report to the tapper server using netcat
$ cat demo_report.tap | nc -w10 cernvm-server 7357
\end{lstlisting}

\item 	Next, download a copy of the CernVM Test Suite and the CernVM Test Cases from the Google Code svn repository
		\cite{GCreleasetesting} and install the the following dependencies.
		
\lstset{language=bash,caption=Install CernVM Test Suite and Dependencies}
\begin{lstlisting}
# Install subversion, required to checkout auto-tapper
$ yum install subversion

# Checkout a copy of cernvm testsuite and cernvm testcases
$ svn checkout http://cernvm-release-testing.googlecode.com/svn/\
trunk/tapper/tapper-autoreport/ cernvm-testsuite

# Install the missing dependencies
$ yum install prove
$ yum install uuid
$ yum install spawn
$ yum install expect
$ yum install expectk
$ yum install nmh

# Run install-mh and accept the defaults
$ install-mh
\end{lstlisting}

\item	Now that cernvm-testsuite has been installed, configure the variables in the configuration file for the
			hypervisors you want to test and according to your \tapper infrastructure setup. Sample configuration files are provided
			in the {\bf config} folder, all of the settings provided in the configuration files by default are the {\bf mandatory, minimum 
			configuration options} and in most cases the	defaults should be sufficient for testing, the only mandatory settings that are
			not provided by default {\bf and must be configured manually are CVM\_TS\_REPORT\_SERVER and 
			CVM\_VM\_IMAGE\_VERSION }. Please refer to the developer manual for a more complete detailed list of the mandatory and 
			optional configuration settings.

\begin{description}
\item[CVM\_TS\_SUITENAME]	Must ALWAYS be set, the default suite name in the configuration file should be suitable

	  		  
\item[CVM\_TS\_SUITEVERSION] 	Must ALWAYS be set, reflects the release version number of the test suite framework, the  default
								version given in the configuration should only be changed if you customize/update the scripts

\item[CVM\_TS\_REPORT\_SERVER]	Must ALWAYS be set, this is the ip address or hostname of the Tapper report server which the reports
      							from the test results are sent to
		
\item[CVM\_TS\_DOWNLOAD\_PAGE]	Must ALWAYS be set, normally the default url provided in the configuration file is accurate, but in the
								event that the internal \cernvm image release page is relocated then this url must be changed.

\item[CVM\_VM\_HYPERVISOR]	Must ALWAYS be set, should not have to change the defaul hypervisor in the configuration files, valid values 
							(case sensitive) are {\bf kvm,vbox,vmware}

\item[CVM\_VM\_TEMPLATE]		Must ALWAYS be set, normally the default template provided in the configuration file should not be changed,
								only change this to use a custom template file. The custom template file \emph{must be placed within the
								templates folder}

\item[CVM\_VM\_NET\_TEMPLATE]	Must ALWAYS be set, normally the default network template provided in the configuration file should not be
	 							changed, only change this to use a custom network template file for the \cernvm image,  {\bf only applies to kvm
	 							and virtualbox}. The custom network template file \emph{must be placed within the templates folder}

\item[CVM\_VM\_IMAGE\_VERSION]	Must ALWAYS be set, specifies the version of the CernVM image to use from the release page
	
\item[CVM\_VM\_IMAGE\_TYPE]		Must ALWAYS be set,  specifies the type of CernVM image, valid image types supported, (case sensitive) are 
								{\bf basic and desktop}

\item[CVM\_VM\_ARCH]		Must ALWAYS be set, specifies the architecture of the \cernvm image, valid architectures (case sensitive) are {\bf 
							x86 and x86\_64}
\end{description}	

\item	Finally, now that cernvm-testsuite has been installed and configured on the test client and the test client and \tapper~Server
			have proven to be working, the next step is to verify that cernvm-testsuite works correctly and can actually send a report to 	
			the \tapper~server in the form of a TAP file. To execute the \cernvm Test Cases script, ``cernvm-tests.sh'' simply source
			the configuration file you wish to use and execute the script. Once the script has completed and sent a TAP report to the server, 
			ensure that the test client is working by viewing the tapper reports in your browser at the  following 
			url: \url{http:/<tapper\_server>/tapper/reports}. You should now see a new report from the test client, there should be a 
			report from a system with the same hostname.\footnote{This is why a consistent hostname convention was emphasized earlier, as 
			reports are often sorted and organized based on hostnames}.

\lstset{language=bash,caption=Execute CernVM Test Cases Script}
\begin{lstlisting}
# Simply source the configuration file, and execute the script
$ . ./config/<configuration_file>
$ ./cernvm-tests.sh
\end{lstlisting}
\end{enumerate}