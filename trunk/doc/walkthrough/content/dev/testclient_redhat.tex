\section{Red Hat Based Test Client Setup}
\subsection{Configuring the system}
\label{sec:rhconfig}
\begin{enumerate}
\item 	After the system has booted remove the follow unnecessary startup applications by selecting from the menu  
		\verb|System -> Preferences -> Startup Applications|
\begin{itemize}
\item	bluetooth
\item	evolution alarm
\item	Gnome Login Sound
\item	PackageKit Update Applet
\item	print queue
\item	screensaver
\item	visual assistance/aid
\item	volume control
\item	any others you think are unnecessary based on your own discretion
\end{itemize}

\item	Next enable and configure the remote desktop from the menu \verb|System -> Preferences -> Remote Desktop| and ensure
		that the following options are configured	
\begin{itemize}
\item	Enable the option ``Allow others to view your desktop''
\item	Enable the option ``Allow other users to control your desktop''
\item	Disable the option ``You must confirm access to this machine''
\end{itemize}

\item	Next enable SSH access to the machine, in order for SSH and VNC access to work the firewall will have to be disabled
\begin{itemize}
\item[a.]	First disable the firewall from the menu \verb|System -> Adminisration -> Firewall| and click the ``Disable'' 
			button and then click ``Apply'' to apply the changes! {\bf This is a quick solution for now because it's too 
			much work to configure the firewall for VNC, SSH, Apache, MySQL, PHPMyAdmin, MCP, and all the other network 
			daemons and should not be a problem if this is just being accessed internally}.

\item[b.]	Now that the firewall is disabled, configure sshd, the ssh daemon, to run on startup

\begin{lstlisting}
$ su -c "chkconfig --level 345 sshd on"
\end{lstlisting}
\end{itemize}

\item	Next, configure the system to login automatically at boot
\begin{itemize}
\item[a.]	Edit the login screen configuration file for gdm using the following command
\begin{lstlisting}
$ su -c "gedit /etc/gdm/custom.conf"
\end{lstlisting}

\item[b.]	Then in the custom.conf file, put the following under the heading [daemon], which will automatically
			log the system in as the user you created, \emph{make sure you replace the user cernvm with the user
			that you created}.

\lstset{language=bash,caption=Configure Automatic Login}
\begin{lstlisting}
AutomaticLoginEnable=true
AutomaticLogin=cernvm
TimedLoginEnable=true
TimedLogin=cernvm
TimedLoginDelay=0
\end{lstlisting}		
\end{itemize}

\item	Next, configure the screen saver from the menu \verb|System -> Preferences -> Screensaver| and ensure that
		the following options are configured
\begin{itemize}
\item	Disable the option ``Lock screen when screensaver active''
\end{itemize}

\item	Now, reboot the machine, and ensure that the following work
\begin{itemize}
\item	It automatically boots up into the full desktop environment without having to login
\item	You have access to the machine using SSH and can login on the root account
\item	You have VNC access to the machine and can control the system using VNC	
\end{itemize}

\item	Finally, update the system from the menu \verb|System -> Administration -> Software Update| and after it has 
		completed the updates reboot the system
\end{enumerate}


\newpage
\subsection{Installing libvirt and virsh}
\label{sec:rhvirsh}
\begin{enumerate}
\item	The virtualization API libvirt and the command line tool virsh~\cite{libvirt} are the essential components required 
		for setting up a test client and must be installed and properly configured before any testing can begin. Ensure that
		you follow the proceeding directions carefully and validate that virsh is working properly before proceeding to 
		install and configure the various hypervisors.
		
\item	First, begin by reviewing the release news listed on the libvirt website, \url{http://libvirt.org/news.htm} and read 
		through the release notes for the latest version released to make sure that there are no regressions or deprecated 
		support for the platforms you wish to support. If you intend to set up an entire infrastructure and support all of the
		\cernvm virtualization platforms, which would include \emph{Xen, KVM, VirtualBox, and VMware}, then you must download
		a version later than \emph{0.8.7} as there was no support for VMware prior to that release.

\item	Next, download the latest release that is a {\bf src.rpm} file from the libvirt release server, 
		\url{http://libvirt.org/sources/} based on the latest release which does not have any regressions or deprecations for
		the virtualization platforms you wish to support~\footnote{This shouldn't be an issue but just in case there is a 
		newer version in which Xen support is deprecated, then you would need to use the last release which has Xen support}.
		As of this date, the latest release of libvirt is version 0.9.2, this is the release that will be used for the
		following instructions and examples.
		
\item	Next, install the following dependencies which are required to generate the libvirt rpm files from the src.rpm file
		that was downloaded, \emph{from now on execute all commands as root}.
		
\lstset{language=bash,caption=Install src.rpm Dependencies}
\begin{lstlisting}
# Change to root account, enter password if prompted
$ su

# Install dependencies for using a src.rpm
$ yum install rpm rpm-devel rpm-libs rpmdevtools rpm-python \
rpm-build rpmrebuild
\end{lstlisting}

\item	Next, install the following dependencies which are required to install libvirt.

\lstset{language=bash,caption=Install libvirt Dependencies}
\begin{lstlisting}
# Change to root account, enter password if prompted
$ su

# Install dependencies for libvirt
$ yum install xhtml1-dtds readline-devel ncurses-devel gettext augeas \
libpciaccess-devel yajl-devel libpcap-devel avahi-devel radvd \
cyrus-sasl-devel parted-devel libcap-ng-devel libssh2-devel \
audit-libs-devel systemtap-sdt-devel gnutls-utils gnutls-devel \
python-devel xen-devel libudev-devel libnl-devel device-mapper-devel \
numactl-devel netcf-devel libcurl-devel libcgroup
\end{lstlisting}
	
\item 	Next, create the libvirt RPM installation files using the following command, replace the src.rpm file
		shown in the example with the file you downloaded previously.

\lstset{language=bash,caption=Create libvirt RPM Files}
\begin{lstlisting}
# Change to root account, enter password if prompted
$ su

$ rpmbuild --rebuild libvirt-0.9.2-1.fc14.src.rpm
\end{lstlisting}

\item	Finally, install the libvirt RPM files by navigating to the \verb|/root/rpmbuild/RPMS| folder and
		then changing to the directory for your computer architecture \emph{such as x86\_64}. Then install
		the files in the same order as shown in the example, replacing the version of src.rpm file shown in 
		the example	with the version of the files in your directory, most importantly install the files
		in the following order: \emph{libvirt-client, libvirt-devel, libvirt-python, libvirt}. Finally,
		if a package does not install or complains about conflicts use the \emph{--force} argument to
		force the installation.

\lstset{language=bash,caption=Install libvirt}
\begin{lstlisting}
# Change to root account, enter password if prompted
$ su

# Change to location of RPM files
$ cd /root/rpmbuild/RPMS/x86_64/

# Install the files in the following order, if install
# fails or has conflicts use   rpm -iv --force   
$ rpm -iv libvirt-client-0.9.2-1.fc13.x86_64.rpm 
$ rpm -iv libvirt-devel-0.9.2-1.fc13.x86_64.rpm
$ rpm -iv libvirt-python-0.9.2-1.fc13.x86_64.rpm
$ rpm -iv libvirt-0.9.2-1.fc13.x86_64.rpm
\end{lstlisting}

\item 	Finally, start the service libvirtd, and ensure that virsh installed correctly and is running by 
		connecting to the test hypervisor and ensuring that the test virtual machine, named ``test'' is running.

\lstset{language=bash,caption=Verify virsh was Installed Properly}
\begin{lstlisting}
# Change to root account, enter password if prompted
$ su

# Verify virsh is working, test should be running
$ service libvirtd start
$ virsh -c test:///default list --all
\end{lstlisting}
\end{enumerate}


\newpage
\subsection{Installing and configuring KVM}
\label{sec:rhkvm}
\begin{enumerate}
\item	The first step is to install the KVM hypervisor, start by installing KVM and the other additional packages such
		as virt-manager, which is a graphical management tool and virt-install, which is a command line interface (CLI)
		virtual machine creation/installation/configuration tool using the following commands with root privileges. If
		you receive a message that a package is already installed then simply continue.

\lstset{language=bash,caption=Installing KVM and Other Related Programs}
\begin{lstlisting}
$ yum install virt-manager qemu-kvm python-virtinst virt-viewer
\end{lstlisting}

\item	Next, verify that KVM has been installed properly and that virsh can connect to the KVM hypervisor using the
		following commands, if you are able to connect to the virsh console without any errors then virsh is able
		to connect to the KVM hypervisor.

\lstset{language=bash,caption=Verify that virsh can Access KVM}
\begin{lstlisting}
$ su
$ virsh -c qemu:///session
\end{lstlisting}

\item 	Finally, to ensure that KVM is properly configured and installed, follow this guide provided on the CernVM website
		\url{http://cernvm.cern.ch/portal/kvm} {\bf except, do not create a kvm definition file as the xml template file
		is provided by the test suite scripts} and verify that you are able to connect to the libvirtd kvm system session.
		
\lstset{language=bash,caption=Verify that KVM is Properly Configured}
\begin{lstlisting}
$ su
$ virsh -c qemu:///system 
\end{lstlisting}
\end{enumerate}




\newpage
\subsection{Installing and configuring VirtualBox}
\label{sec:rhvbox}
\begin{enumerate}
\item	First, begin by downloading the latest version of VirtualBox from the VirtualBox download page,
		\url{http://www.virtualbox.org/wiki/Downloads} ensure that you select the appropriate Red Hat based
		distribution, version and architecture for your system. The following instructions for this section of
		the guide uses VirtualBox 4.0.10 for Fedora 13, AMD64.
		
\item	Next, after downloading the latest version of VirtualBox for your distribution install VirtualBox as the root
		account using the following command.
\begin{lstlisting}
# Enter the root password when prompted
$ su
$ rpm -iv VirtualBox-*.rpm
\end{lstlisting}	

\item	Next, in order to use VirtualBox and have full access to the drivers needed for USB support, ensure that the root
		account belongs to the group ``vboxusers''. Begin by navigating to \verb|System -> Users and Groups| and then from
		the ``User Manager'' window click the ``Groups'' tab, under the column ``Group Name'' for the group ``vboxusers'' 
		ensure that root is one of the group members. If the root account is not a group member of ``vboxusers'' highlight
		the ``vboxusers'' entry and click the ``Properties'' button, then enable the root account from the list of users
		and click ``OK'' to apply the changes.
		
\item	Due to an issue with VirtualBox\footnote{The issues is that VirtualBox looks for virtual machine configuration files (*.vbox)
		in the ``VirtualBox VMs'' folder of the user that launched VirtualBox. The issue is worsened by the fact that there can
		only be one ``VirtualBox VMs'' folder which causes conflicts with multiple users.}, in order for it to work with virsh 
		the virtual machine(s) must be created and configured as the root account, otherwise when you try to connect or start a 
		VirtualBox virtual machine with virsh you will get an ``unknown error'', which is obviously very vague and difficult to 
		resolve. {\bf Therefore ALWAYS start VirtualBox as the root account using the following procedure}.

\lstset{language=bash,caption=Always Start VirtualBox as Root}
\begin{lstlisting}
# Switch to the root account, enter root password
$ su

# Start VirtualBox as root
$ virtualbox
\end{lstlisting}

\item	Next, to verify that VirtualBox has been installed properly and that virsh can connect to the VirtualBox hypervisor, 
		verify that the VirtualBox module, \emph{vboxdrv} has been loaded and that you are able to connect to the virsh console 
		without any errors.

\lstset{language=bash,caption=Verify that virsh can Access VirtualBox}
\begin{lstlisting}
$ su

# Verify that the vboxdrv module is loaded
$ lsmod | grep -i vboxdrv

# Verify that virsh can connect to virtualbox
$ virsh -c vbox:///session
\end{lstlisting}

\item	Now, proceed to download and extract the desired VirtualBox virtual machine image onto the system from the \cernvm download 
		portal, url{http://cernvm.cern.ch/portal/downloads} it is recommended that you download the VirtualBox Desktop image for your 
		architecture. For this guide the Desktop image will be downloaded as it is the most practical image for the majority of 
		users.
		
\lstset{language=bash,caption=Download and Extract CernVM VirtualBox Desktop Image}
\begin{lstlisting}
$ wget http://rbuilder.cern.ch/downloadImage?fileId=1711
$ gunzip cernvm*.gz
\end{lstlisting}

\item 	Now, to ensure that VirtualBox is properly configured and installed, follow this guide provided on the CernVM website
		\url{http://cernvm.cern.ch/portal/vbinstallation} which provides step by step instructions.{\bf Again, ensure that you 
		ALWAYS start VirtualBox as the root account}.

\item	Furthermore the following VirtualBox options must also be applied, as they are specific to configuring a \cernvm test client.
\begin{itemize}
\item[a.]	Navigate to \verb|Settings -> System -> Motherboard| for boot order list, floppy disks are redundant nowadays so disable 
			floppy from the boot order list. Then, disable the option ``Enable absolute pointing device'', which is for tablets.
			
\item[b.]	Next, go to the option \verb|System -> Processor|, and ensure that the option ``Enable PAE/NX'' is disabled
			as it best left disabled for \cernvm images~\footnote{Physical Address Extension (PAE) is useful if you are running a 
			32-bit processor as this enables a 32-bit operating system to access to more than 4GB of memory}.
			
\item[c.] 	Now, go to the next option \verb|System -> Acceleration|, if your system supports VT-x or AMD-V it is {\bf HIGHLY} 
			recommended that you enable these options ``Enable Nested Paging'' and ``Enable VT-x/AMD-V'' for performance gains. 
			To verify that your system supports this execute the following command, the output should not be empty. 

\lstset{language=bash}
\begin{lstlisting}
egrep '(vmx|svm)' --color=always /proc/cpuinfo
\end{lstlisting}

\item[d.]	Next go the setting for ``Audio'', audio is mostly redundant and unnecessary for \cernvm, so disable the audio unless
			you explicitly require audio support.
			
\item[e.]	Finally, since virsh only supports console access through a serial port for LXC, Xen, QEMU/KVM, and UML, go to the
			setting ``Serial Ports'' and simply verify that all of the serial ports as disabled for VirtualBox.
\end{itemize}

\item	Now that the virtual machine has been created and configured verify that it is able to boot completely without crashing,
		\emph{you will be presented with a login screen when it has booted completely}. Then shutdown the virtual machine by clicking 
		``Actions'' from within the virtual machine and selecting `Shutdown'', after the virtual machine has shutdown close
		VirtualBox and then connect to the VirtualBox hypervisor and determine that you can view, start, and stop virtual machine.

\lstset{language=bash,caption=Verify VirtualBox Works with Virsh}
\begin{lstlisting}
$ su
$ virsh --connect vbox:///session

# Verify the virtualbox virtual machine is accessible
# Name of the virtual machine created should be listed
$ list --all

# Verify the virtual machine can be turned on/off
$ start <name of virtual machine>
# Wait about 2 minutes for the system to boot
$ shutdown <name of virtual machine>
\end{lstlisting}

\item	Finally, configure the the virtual machine network to automatically start when the system boots and then create an
		XML definition file of the virtual machine, which will be used later by the test scripts. In the following example 
		save the XML definition file with the same name as the virtual machine that was created, such as 
		\emph{cernvm-vbox-2.3.0.xml} so that it is easy to differentiate between multiple virtual machines for different
		hypervisors.

\lstset{language=bash,caption=Create XML Definition File and Configure Network}
\begin{lstlisting}
$ su

# Create virtual machine XML definition file
$ virsh --connect vbox:///session dumpxml <name of virtual machine> \
> <name of virtual machine>.xml

# Configure virsh network for VirtualBox
$ virsh --connect vbox:///session
$ net-start vboxnet0
$ net-autostart vboxnet0
\end{lstlisting}				
\end{enumerate}




\newpage
\subsection{Configuring the CernVM Image}
\label{sec:rhcernvmconfig}
\begin{enumerate}
\item 	The next steps involve configuring the CernVM image to integrate with virsh as well as the test suite, first
		start the virtual machine, execute the command to get the IP address of the CernVM image. Then follow this guide
		provided on the CernVM website on how to configure the CernVM image and create a new user using the web interface 
		\url{http://cernvm.cern.ch/portal/cvmconfiguration} and reboot the system, \emph{all virsh commands require root 
		privileges}.
		
\begin{lstlisting}
$ virsh start cernvm
# Wait about 2 minutes for the system to boot 
# Then get IP Address using the following command
$ arp -an
\end{lstlisting}

\item 	Now that the system has booted, login in to the system using SSH for the user you created using the CernVM web interface
		and set the root password
		
\lstset{language=bash,caption=Set Root Password}
\begin{lstlisting}
$ ssh <user you created>@cernvm-image-ipaddress

# Type the following command, and enter a root password you won't forget
$ sudo passwd root
\end{lstlisting}

\item	Now that you are logged into the system through SSH, then enable root login through SSH using 
		the following commands.
				
\lstset{language=bash,caption=Enable SSH Root Login}
\begin{lstlisting}
# edit the file /etc/ssh/sshd_config and uncomment the line 
# PermitRootLogin yes	in order to enable root login
$ vi /etc/ssh/sshd_config
\end{lstlisting}

\item 	Next from the host machine \emph{ie. the machine you're currently using} enable automatic login as root 
		through ssh on the KVM guest, first ensure that the guest machine has been started.
		
\lstset{language=bash,caption=Enable Automatic SSH Root Login}
\begin{lstlisting}
# Restart the virtual machine wait for 
# it to completely shut off and turn on
$ virsh shutdown cernvm
$ virsh start cernvm

# Generate a public key, when prompted press enter for everything
$ ssh-keygen -t rsa

# Get the ip address of the running cernvm guest, as done previously
$ arp -an

# Next, run the following command to setup automatic login for ssh
# without having to type the password. When prompted for the password
# enter the password you just set previously
$ ssh-copy-id -i ~/.ssh/id_rsa.pub root@cernvm-image-ipaddress
		
# Disconnect, and try and login to the machine using ssh and ensure 
# that you can login as root without having to type the password
$ ssh root@cernvm-image-ipaddress
\end{lstlisting}

\item	If everything so far has worked, then the test client and CernVM image have been installed and configured properly,
		if you have any outstanding issues solve them before proceeding further, or go to the section ``Server Platform 		
		Setup''~\ref{sec:serversetup} as the \tapper~server does not require virtual machine creation and configuration.
\end{enumerate}


\newpage
\subsection{Setting up the Tapper Test Suite}
\label{sec:rhtestsuite}
\begin{enumerate}
\item 	{\bf Before proceeding any further ensure that you have all other test clients set up this far, and then proceed
		to follow the instructions for setting up and configuring the \tapper~server in the section ``Server Platform Setup''}		
		~\ref{sec:serversetup}.
		
\item 	Now that the \tapper~server has been installed and configured and the \tapper web interface and database have proven
		to be working, the next step is to verify that the test client can actually send a report to the \tapper~server in
		the form of a TAP file. After sending the TAP report to the server, ensure that the test client is working by viewing 
		the tapper reports in your browser at the following url: \url{http://localhost/tapper/reports}. You should now see a 
		report from the test client, there should be a report from a system named whatever the ``Tapper-Machine-Name'' in 
		demo\_report.tap was set as. \emph{For the example demo\_report.tap provided below it would be cernvm-rhtestclient}.
		\footnote{This is why a consistent hostname convention was emphasized earlier, as reports are often sorted and organized 
		based on hostnames}.
		
\lstset{language=bash,caption=Send a Basic Report to the \tapper~Server}
\begin{lstlisting}
# Save the following in a file named demo_report.tap
$ vi demo_report.tap

	1..2
	# Tapper-Suite-Name: Tapper-Deployment
	# Tapper-Suite-Version: 1.001
	# Tapper-Machine-Name: cernvm-rhtestclient
	ok - Hello test world
	ok - Just another description

# Send the report to the tapper server using netcat
$ cat demo_report.tap | nc -w10 cernvm-server 7357
\end{lstlisting}

\item 	Next, download a copy of tapper-autoreport and the CernVM Test Cases from the Google Code svn repository
		\cite{GCreleasetesting} and install the tapper-autoreport dependencies.
		
\lstset{language=bash,caption=Install \tapper~AutoReport and Dependencies}
\begin{lstlisting}
# Install subversion, required to checkout auto-tapper
$ yum install subversion

# Checkout a copy of auto-tapper and cernvm testcases
$ svn checkout http://cernvm-release-testing.googlecode.com/svn/\
trunk/tapper/tapper-autoreport/ cernvm-release-testing-read-only

# Install the missing dependencies
$ yum install perl-Module-CoreList
$ yum install perl-CPAN

# Install the required perl modules
$ cpan
$ install prove
$ install XML::XPath
\end{lstlisting}

\item	Now that that tapper-autoreport has been installed, configure the following variables in the script
		``cernvm-tests.sh'' according to your \tapper infrastructure setup.
\begin{itemize}
\item	OSNAME		- The operating system of the test client, such as ``Red Hat 5''
\item	VMNAME 		- The domain name of the virtual machine used earlier with virsh
\item	VM\_XML\_DEFINITION	- The location and name of cernvm virtual machine definition XML file used to create virtual machine
\item	HOSTNAME	- The hostname of the test client
\item	GUESTIP		- This is the IP address of the CernVM image
\item	TAPPER\_REPORT\_SERVER	- The hostname of the \tapper~Report~Server
\end{itemize}	

\item	Finally, now that tapper-autoreport has been installed and configured on the test client and the test client and \tapper~Server
		have proven to be working, the next step is to verify that tapper-autoreport works correctly and can actually send a report to 	
		the \tapper~server in the form of a TAP file. After the \cernvm Test Cases script, ``cernvm-tests.sh'' has completed and sent
		a TAP report to the server, ensure that the test client is working by viewing the tapper reports in your browser at the 
		following url: \url{http://localhost/tapper/reports}. You should now see new report from the test client, there should be a 
		report from a system with the same hostname\footnote{This is why a consistent hostname convention was emphasized earlier, as 
		reports are often sorted and organized based on hostnames}.

\lstset{language=bash,caption=Run \tapper-AutoReport for CernVM Test Cases}
\begin{lstlisting}
# Simply execute the script and wait for it to finish
./cernvm-tests.sh
\end{lstlisting}
\end{enumerate}