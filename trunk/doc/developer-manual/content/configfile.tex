\chapter{Test Suite Configuration File}
\label{sct:configfile}

The configuration file is essential to setting up the initial \cernvm test suite for testing, while most of the default 
settings provided in the configuration file are sufficient for most \cernvm image testing environments, there are still
some mandatory settings which {\bf must be configured before testing can begin}. In addition to the mandatory settings
that must be specified before tests can be executed, there are also optional configuration settings which provide settings
that can override the default settings normally taken when the default configuration file is used, these include options
to override the default virtual machine settings specified in the template files.




\subsection{Mandatory Settings}
\label{sct:mandatorysettings}

In most testing scenarios only the mandatory configuration settings need to be specified such as the hypervisor and 
the download page, but optional settings are also provided to override internal default settings used by the 
\cernvmtestframework\. The following is a list of the mandatory settings that must be configured in order for the
tests to work, ensure that you enter valid values, \emph{in lower-case}, for the settings indicated.


\begin{itemize}
\item	SUITENAME
		\begin{itemize}
		\item	Must ALWAYS be set, only define once at the top of the configuration file, 
	  		  	usually the default suite name given in the test suite configuration file is fine
		\end{itemize}
	  		  
\item	SUITEVERSION
		\begin{itemize}
		\item	Must ALWAYS be set, only define once at the top of the configuration file,
	  		  	reflects the release version number of the test suite framework, the default 
      		  	suite version given in the test suite configuration should only be changed if 
     		  	you make modifications to the test suite framework which differentiate it from
      		  	the version released on Google Code.
		\end{itemize}

\item	REPORT\_SERVER
		\begin{itemize}
		\item	Must ALWAYS be set, only define once at the top of the configuration file,
      			this is the ip address or hostname of the Tapper report server which the reports
      			from the test results are sent to
		\end{itemize}
		
\item	DOWNLOAD\_PAGE
		\begin{itemize}
		\item	Must ALWAYS be set, normally the default url provided in the configuration file is
				accurate, but in the event that the internal \cernvm image release page is relocated
				then this url must be changed.
		\end{itemize}

\item	HYPERVISOR
		\begin{itemize}
		\item	Must ALWAYS be set, MUST be the first setting before the rest of the mandatory
	  			and optional settings specific to the hypervisor are set
	  	\item	Valid values (case sensitive) are {\bf kvm,vbox,vmware}
		\end{itemize}
		
\item	IMAGE\_TYPE
		\begin{itemize}
		\item	Must ALWAYS be set,  MUST be defined for each HYPERVISOR entry in the configuration
				file, specifies the type of CernVM image, such as desktop, basic, head node, etc
		\item	Valid image types supported, (case sensitive) are {\bf basic and desktop}
		\end{itemize}
		
\item	ARCH
		\begin{itemize}
		\item	Must ALWAYS be set, MUST be defined for each HYPERVISOR entry in the configuration
				file, specifies the architecture of the \cernvm image
		\item	Valid architectures (case sensitive) are {\bf x86 and x86\_64}
		\end{itemize}
\end{itemize}




\subsection{Optional Settings}
\label{sct:optionalsettings}

In most testing scenarios only the mandatory configuration settings need to be specified such as the hypervisor and 
the download page, but optional settings are also provided to override internal default settings used by the 
\cernvmtestframework\. The following is a list of the optional settings that may be specified to override the default
settings, the optional settings must be configured for each of the HYPERVISOR settings defined in the configuration file.
The optional settings are separated primarily into four categories, host settings, virtual machine settings, web interface
settings, and test case settings. 

Again, only the mandatory settings are required to be specified in order for the tests to work, the optional settings 
can be ignored completely and the test suite scripts should still execute correctly. Therefore, optional settings should
only be specified by advanced users as improper optional settings can cause precondition tests to return failures,
\emph{it is only recommended that you start configuring optional settings after verifying the results of the scripts 
using only the mandatory settings}.


\paragraph*{Optional Host Settings}
\begin{itemize}
\item	IMAGES\_DIR
		\begin{itemize}
		\item	The root directory for the location of the \cernvm images and all
				configuration files and settings, by default /usr/share/images on
				Linux/OS X systems and \verb|C:\users\default\application data\images|
				on Windows systems
		\end{itemize}
		
\item	OSNAME
		\begin{itemize}
		\item	The name of the host operating system, such as Red Hat 5, OS X
				Snow Leopard, or Windows 7, configure accordingly
		\item	\emph{Support may be added eventually to automatically configure OSNAME}
		\end{itemize}
		
\item	HOSTNAME
		\begin{itemize}
		\item	The hostname of the system, determined automatically by the script,
				only set this if you wish to override the default hostname of the
				system
		\end{itemize}
\end{itemize}