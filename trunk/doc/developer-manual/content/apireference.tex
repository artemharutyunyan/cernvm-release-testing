% Document: ./content/apireference
% Source: ./../../tapper/tapper-autoreport/
% Generated with ROBODoc Version 4.99.41 (Aug 17 2011)
\newpage
\section{test-suite/cernvm-preconditions}
\textsl{[ Generics ]}

\label{ch:robo32}
\label{ch:test_suite_cernvm_preconditions}
\index{unsorted!cernvm-preconditions}\index{Generics!cernvm-preconditions}
\textbf{NAME}
\begin{verbatim}
   cernvm-preconditions
\end{verbatim}
\textbf{DESCRIPTION}
\begin{verbatim}
   This script contains each of the CernVM Release Testing precondition
   tests, which are required preconditions that must pass for the results 
   of test cases to be accurate. The precondition tests have a simple
   interface to execute each test and each test returns either a success or 
   failure, (0 or 1)

   More complex precondition tests can be created by combining other 
   precondition tests as prerequisites for a precondition test
\end{verbatim}
\textbf{TODO}
\begin{verbatim}
   CLEAN UP THE FOLLOWING PRECONDITON TESTS AND PLACE THEM IN THIS FILE
   Precondition Test 2 - Verify that virtual machine domain has been created 
                         from an xml file

   Precondition Test 3 - Verify that virtual machine can be started

   Precondition Test 4 - Verify that virtual machine has been stopped

   Precondition Test 5 - Verify that the virtual has console support

   Precondition Test 6 - Verify that virtual machine has web interface support

   Precondition Test 7 - Verify that it is possible to login on web interface
\end{verbatim}
\newpage
\subsection{cernvm-preconditions/configure\_image\_web}
\textsl{[ cernvm-preconditions ]}
\textsl{[ Functions ]}

\label{ch:robo0}
\label{ch:cernvm_preconditions_configure_image_web}
\index{unsorted!configure\_image\_web}\index{Functions!configure\_image\_web}
\textbf{NAME}
\begin{verbatim}
   configure_image_web
\end{verbatim}
\textbf{DESCRIPTION}
\begin{verbatim}
   Precondition Test - Setup and configure the initial CernVM image through the
   web interface
\end{verbatim}
\textbf{ARGUMENTS}
\begin{verbatim}
   $1 - The hostname or ip address for the web interface
   $2 - The default username to access web interface
   $3 - The default password to access web interface
   $4 - The name of the logfile
\end{verbatim}
\textbf{RESULT}
\begin{verbatim}
   exitstatus - Sets $? as a zero for success, otherwise sets an error code
\end{verbatim}
\textbf{EXAMPLE}
\begin{verbatim}
   configure_image_web 192.168.1.125 admin password config-image.log
\end{verbatim}
\textbf{TODO}
\begin{verbatim}
   Implement a function that uses curl to get the updates from the web server 
   to determine when the system has rebooted
\end{verbatim}
\newpage
\subsection{cernvm-preconditions/create\_def}
\textsl{[ cernvm-preconditions ]}
\textsl{[ Functions ]}

\label{ch:robo1}
\label{ch:cernvm_preconditions_create_def}
\index{unsorted!create\_def}\index{Functions!create\_def}
\textbf{NAME}
\begin{verbatim}
   create_def
\end{verbatim}
\textbf{DESCRIPTION}
\begin{verbatim}
   Precondition Test - Create an XML definition file for the virtual machine based
   on the template XML definition file and settings defined and return the
   location of the xml defintion file created
\end{verbatim}
\textbf{ARGUMENTS}
\begin{verbatim}
   $1 - The template file to use
   $2 - The directory to save the final xml definition file in
\end{verbatim}
\textbf{RETURN VALUE}
\begin{verbatim}
   definitionfile - The location of the xml defintion file created
\end{verbatim}
\textbf{RESULT}
\begin{verbatim}
   exitstatus - Sets $? as a zero for success, otherwise sets an error code
\end{verbatim}
\textbf{EXAMPLE}
\begin{verbatim}
   create_def vm-template.xml /root
\end{verbatim}
\newpage
\subsection{cernvm-preconditions/create\_net}
\textsl{[ cernvm-preconditions ]}
\textsl{[ Functions ]}

\label{ch:robo2}
\label{ch:cernvm_preconditions_create_net}
\index{unsorted!create\_net}\index{Functions!create\_net}
\textbf{NAME}
\begin{verbatim}
   create_net
\end{verbatim}
\textbf{DESCRIPTION}
\begin{verbatim}
   Precondition Test - Verify that the virtual machine network has been created 
   from an xml file
\end{verbatim}
\textbf{ARGUMENTS}
\begin{verbatim}
   $1 - The path to the network XML definition file
   $2 - The virtual machine network name 
\end{verbatim}
\textbf{RESULT}
\begin{verbatim}
   exitstatus - Sets $? as a zero for success, otherwise sets an error code
\end{verbatim}
\textbf{EXAMPLE}
\begin{verbatim}
   create_net ./network-definition.xml default
\end{verbatim}
\newpage
\subsection{cernvm-preconditions/create\_net\_def}
\textsl{[ cernvm-preconditions ]}
\textsl{[ Functions ]}

\label{ch:robo3}
\label{ch:cernvm_preconditions_create_net_def}
\index{unsorted!create\_net\_def}\index{Functions!create\_net\_def}
\textbf{NAME}
\begin{verbatim}
   create_net_def
\end{verbatim}
\textbf{DESCRIPTION}
\begin{verbatim}
   Precondition Test - Create an XML definition file for the virtual machine network
   based on the template network XML definition file and settings defined and 
   return the location of the created xml defintion file
\end{verbatim}
\textbf{ARGUMENTS}
\begin{verbatim}
   $1 - The network template file to use
   $2 - The directory to save the final network xml definition file in
\end{verbatim}
\textbf{RETURN VALUE}
\begin{verbatim}
   netdefinitionfile - The location of the network xml defintion file created
\end{verbatim}
\textbf{RESULT}
\begin{verbatim}
   exitstatus - Sets $? as a zero for success, otherwise sets an error code
\end{verbatim}
\textbf{EXAMPLE}
\begin{verbatim}
   create_net_def network-template.xml /root
\end{verbatim}
\newpage
\subsection{cernvm-preconditions/download\_extract}
\textsl{[ cernvm-preconditions ]}
\textsl{[ Functions ]}

\label{ch:robo4}
\label{ch:cernvm_preconditions_download_extract}
\index{unsorted!download\_extract}\index{Functions!download\_extract}
\textbf{NAME}
\begin{verbatim}
   download_extract
\end{verbatim}
\textbf{DESCRIPTION}
\begin{verbatim}
   Precondition Test - Download and extract the CernVM image, returns the location 
   of the extracted cernvm image file
\end{verbatim}
\textbf{ARGUMENTS}
\begin{verbatim}
   $1 - The CernVM image download url
   $2 - The directory to place the downloaded image in
   $3 - The name of the log file
\end{verbatim}
\textbf{RETURN VALUE}
\begin{verbatim}
   imagelocation - The location of the extracted CernVM image file
\end{verbatim}
\textbf{RESULT}
\begin{verbatim}
   exitstatus - Sets $? as a zero for success, otherwise sets an error code
\end{verbatim}
\textbf{EXAMPLE}
\begin{verbatim}
   IMAGE_LOCATION=$(download_extract http://someurl/file.vmdk.gz /root dl-extract.log)
\end{verbatim}
\newpage
\subsection{cernvm-preconditions/image\_url}
\textsl{[ cernvm-preconditions ]}
\textsl{[ Functions ]}

\label{ch:robo5}
\label{ch:cernvm_preconditions_image_url}
\index{unsorted!image\_url}\index{Functions!image\_url}
\textbf{NAME}
\begin{verbatim}
   image_url
\end{verbatim}
\textbf{DESCRIPTION}
\begin{verbatim}
   Precondition Test - Verify that the download page exists and that there is a 
   valid download url for the CernVM image specified, returns the url to 
   download the image
\end{verbatim}
\textbf{ARGUMENTS}
\begin{verbatim}
   $1 - The CernVM download page url
   $2 - The image version
   $3 - The hypervisor of the image
   $4 - The architecture of the image
   $5 - The type of image
\end{verbatim}
\textbf{RETURN VALUE}
\begin{verbatim}
   imageurl - The url to download the image
\end{verbatim}
\textbf{RESULT}
\begin{verbatim}
   exitstatus - Sets $? as a zero for success, otherwise sets an error code
\end{verbatim}
\textbf{EXAMPLE}
\begin{verbatim}
   IMAGE_URL=$(image_url http://downloadpage.com 2.4.0 kvm x86 desktop)
\end{verbatim}
\newpage
\subsection{cernvm-preconditions/validate\_def\_settings}
\textsl{[ cernvm-preconditions ]}
\textsl{[ Functions ]}

\label{ch:robo6}
\label{ch:cernvm_preconditions_validate_def_settings}
\index{unsorted!validate\_def\_settings}\index{Functions!validate\_def\_settings}
\textbf{NAME}
\begin{verbatim}
   validate_def_settings
\end{verbatim}
\textbf{DESCRIPTION}
\begin{verbatim}
   Precondition Test - Verify that the mandatory configuration settings for the
   virtual machine XML definition file have been provided and are valid
\end{verbatim}
\textbf{ARGUMENTS}
\begin{verbatim}
   $1 - The virtual machine XML definition file
\end{verbatim}
\textbf{RESULT}
\begin{verbatim}
   exitstatus - Sets $? as a zero for success, otherwise sets an error code
\end{verbatim}
\textbf{EXAMPLE}
\begin{verbatim}
   validate_def_settings ./vm-definition.xml
\end{verbatim}
\newpage
\subsection{cernvm-preconditions/validate\_def\_xml}
\textsl{[ cernvm-preconditions ]}
\textsl{[ Functions ]}

\label{ch:robo7}
\label{ch:cernvm_preconditions_validate_def_xml}
\index{unsorted!validate\_def\_xml}\index{Functions!validate\_def\_xml}
\textbf{NAME}
\begin{verbatim}
   validate_def_xml
\end{verbatim}
\textbf{DESCRIPTION}
\begin{verbatim}
   Precondition Test - Verify that the XML definition file provided is valid
\end{verbatim}
\textbf{ARGUMENTS}
\begin{verbatim}
   $1 - The virtual machine XML definition file
\end{verbatim}
\textbf{RESULT}
\begin{verbatim}
   exitstatus - Sets $? as a zero for success, otherwise sets an error code
\end{verbatim}
\textbf{EXAMPLE}
\begin{verbatim}
   validate_def_xml ./vm-definition.xml
\end{verbatim}
\newpage
\subsection{cernvm-preconditions/validate\_net\_settings}
\textsl{[ cernvm-preconditions ]}
\textsl{[ Functions ]}

\label{ch:robo8}
\label{ch:cernvm_preconditions_validate_net_settings}
\index{unsorted!validate\_net\_settings}\index{Functions!validate\_net\_settings}
\textbf{NAME}
\begin{verbatim}
   validate_net_settings
\end{verbatim}
\textbf{DESCRIPTION}
\begin{verbatim}
   Precondition Test - Verify that the mandatory configuration settings for the
   network XML definition file have been provided and are valid
\end{verbatim}
\textbf{ARGUMENTS}
\begin{verbatim}
   $1 - The network XML definition file
\end{verbatim}
\textbf{RESULT}
\begin{verbatim}
   exitstatus - Sets $? as a zero for success, otherwise sets an error code
\end{verbatim}
\textbf{EXAMPLE}
\begin{verbatim}
   validate_net_settings ./network-definition.xml
\end{verbatim}
\newpage
\subsection{cernvm-preconditions/verify\_autologin\_ssh}
\textsl{[ cernvm-preconditions ]}
\textsl{[ Functions ]}

\label{ch:robo9}
\label{ch:cernvm_preconditions_verify_autologin_ssh}
\index{unsorted!verify\_autologin\_ssh}\index{Functions!verify\_autologin\_ssh}
\textbf{NAME}
\begin{verbatim}
   verify_autologin_ssh
\end{verbatim}
\textbf{DESCRIPTION}
\begin{verbatim}
   Precondition Test - Enable automatic SSH login to the machine for the user
   specified using keys instead of passwords, and verify that it is possible 
   to login automatically
\end{verbatim}
\textbf{ARGUMENTS}
\begin{verbatim}
   $1 - The IP address of the machine to login via ssh
   $2 - The username to login with
   $3 - The password to login with
\end{verbatim}
\textbf{RESULT}
\begin{verbatim}
   exitstatus - Sets $? as a zero for success, otherwise sets an error code
\end{verbatim}
\textbf{EXAMPLE}
\begin{verbatim}
   verify_autologin_ssh 192.168.1.125 root password
\end{verbatim}
\textbf{TODO}
\begin{verbatim}
   Implement support to only remove the offending key line from known_hosts
   instead of deleting the entire file
\end{verbatim}
\newpage
\subsection{cernvm-preconditions/verify\_exists}
\textsl{[ cernvm-preconditions ]}
\textsl{[ Functions ]}

\label{ch:robo10}
\label{ch:cernvm_preconditions_verify_exists}
\index{unsorted!verify\_exists}\index{Functions!verify\_exists}
\textbf{NAME}
\begin{verbatim}
   verify_exists
\end{verbatim}
\textbf{DESCRIPTION}
\begin{verbatim}
   Precondition Test - Verify that a file/folder exists
\end{verbatim}
\textbf{ARGUMENTS}
\begin{verbatim}
   $1 - The location and name of the file
\end{verbatim}
\textbf{RESULT}
\begin{verbatim}
   exitstatus - Sets $? as a zero for success, otherwise sets an error code
\end{verbatim}
\textbf{EXAMPLE}
\begin{verbatim}
   verify_exists /root/file.tar.gz
\end{verbatim}
\newpage
\subsection{cernvm-preconditions/verify\_hash}
\textsl{[ cernvm-preconditions ]}
\textsl{[ Functions ]}

\label{ch:robo11}
\label{ch:cernvm_preconditions_verify_hash}
\index{unsorted!verify\_hash}\index{Functions!verify\_hash}
\textbf{NAME}
\begin{verbatim}
   verify_hash
\end{verbatim}
\textbf{DESCRIPTION}
\begin{verbatim}
   Precondition Test - Verify the hash of a file
\end{verbatim}
\textbf{ARGUMENTS}
\begin{verbatim}
   $1 - The location and name of the file
\end{verbatim}
\textbf{RESULT}
\begin{verbatim}
   exitstatus - Sets $? as a zero for success, otherwise sets an error code
\end{verbatim}
\textbf{EXAMPLE}
\begin{verbatim}
   verify_hash /root/file.tar.gz
\end{verbatim}
\textbf{TODO}
\begin{verbatim}
   Implement the verify_hash function later as it is not important at the moment
\end{verbatim}
\newpage
\subsection{cernvm-preconditions/verify\_hypervisor}
\textsl{[ cernvm-preconditions ]}
\textsl{[ Functions ]}

\label{ch:robo12}
\label{ch:cernvm_preconditions_verify_hypervisor}
\index{unsorted!verify\_hypervisor}\index{Functions!verify\_hypervisor}
\textbf{NAME}
\begin{verbatim}
   verify_hypervisor
\end{verbatim}
\textbf{DESCRIPTION}
\begin{verbatim}
   Precondition Test - Verify that the hypervisor for the current virtual machine
   tested is accessible, set the hypervisor URI as a global variable
\end{verbatim}
\textbf{ARGUMENTS}
\begin{verbatim}
   $1 - The virtual machine XML definition file
\end{verbatim}
\textbf{RESULT}
\begin{verbatim}
   exitstatus - Sets $? as a zero for success, otherwise sets an error code
\end{verbatim}
\textbf{EXAMPLE}
\begin{verbatim}
   verify_hypervisor vm-definition.xml
\end{verbatim}
\newpage
\subsection{cernvm-preconditions/verify\_ssh\_login}
\textsl{[ cernvm-preconditions ]}
\textsl{[ Functions ]}

\label{ch:robo13}
\label{ch:cernvm_preconditions_verify_ssh_login}
\index{unsorted!verify\_ssh\_login}\index{Functions!verify\_ssh\_login}
\textbf{NAME}
\begin{verbatim}
   verify_ssh_login
\end{verbatim}
\textbf{DESCRIPTION}
\begin{verbatim}
   Precondition Test - Verify that user is able to login via ssh
\end{verbatim}
\textbf{ARGUMENTS}
\begin{verbatim}
   $1 - The IP address of the machine to login via ssh
   $2 - The username to login with
\end{verbatim}
\textbf{RESULT}
\begin{verbatim}
   exitstatus - Sets $? as a zero for success, otherwise sets an error code
\end{verbatim}
\textbf{EXAMPLE}
\begin{verbatim}
   verify_ssh_login 192.168.1.125 root
\end{verbatim}
\newpage
\subsection{cernvm-preconditions/verify\_virsh\_uri}
\textsl{[ cernvm-preconditions ]}
\textsl{[ Functions ]}

\label{ch:robo14}
\label{ch:cernvm_preconditions_verify_virsh_uri}
\index{unsorted!verify\_virsh\_uri}\index{Functions!verify\_virsh\_uri}
\textbf{NAME}
\begin{verbatim}
   verify_virsh_uri
\end{verbatim}
\textbf{DESCRIPTION}
\begin{verbatim}
   Precondition Test - Verify that the URI virsh is connected to matches the
                       URI for the current hypervisor 
\end{verbatim}
\textbf{NOTES}
\begin{verbatim}
   This precondition test is useful for catching a potential libvirt or hypervisor error
   that is not caught by the scripts or virsh, for example if virsh fails to connect properly
   to the URI specified, the URI that is returned by virsh will be the default and not match
   the URI for the current hypervisor being tested
\end{verbatim}
\textbf{ARGUMENTS}
\begin{verbatim}
   $1 - The URI of the hypervisor
\end{verbatim}
\textbf{RESULT}
\begin{verbatim}
   exitstatus - Sets $? as a zero for success, otherwise sets an error code
\end{verbatim}
\textbf{EXAMPLE}
\begin{verbatim}
   verify_virsh_uri vmwarews:///session
\end{verbatim}
\newpage
\subsection{cernvm-preconditions/verify\_vm\_net}
\textsl{[ cernvm-preconditions ]}
\textsl{[ Functions ]}

\label{ch:robo15}
\label{ch:cernvm_preconditions_verify_vm_net}
\index{unsorted!verify\_vm\_net}\index{Functions!verify\_vm\_net}
\textbf{NAME}
\begin{verbatim}
   verify_vm_net
\end{verbatim}
\textbf{DESCRIPTION}
\begin{verbatim}
   Precondition Test - Verify that virtual machine NAT network is active and 
   set to autostart
\end{verbatim}
\textbf{ARGUMENTS}
\begin{verbatim}
   $1 - The virtual machine network name
\end{verbatim}
\textbf{RESULT}
\begin{verbatim}
   exitstatus - Sets $? as a zero for success, otherwise sets an error code
\end{verbatim}
\textbf{EXAMPLE}
\begin{verbatim}
   verify_vm_net default
\end{verbatim}
\newpage
\section{test-suite/cernvm-testcases}
\textsl{[ Generics ]}

\label{ch:robo33}
\label{ch:test_suite_cernvm_testcases}
\index{unsorted!cernvm-testcases}\index{Generics!cernvm-testcases}
\textbf{NAME}
\begin{verbatim}
   cernvm-testcases
\end{verbatim}
\textbf{DESCRIPTION}
\begin{verbatim}
   This script contains each of the CernVM Release Testing
   test cases and provides a simple interface to execute each test
   and returns either a success or failure, (0 or 1) which can be 
   used to generate a TAP report.

   More complex test cases can be created by combining other test cases
   as prerequisites for the test case
\end{verbatim}
\textbf{NOTES}
\begin{verbatim}
   Nearly all of the test cases require the root account on the CernVM image as 
   many of the files and commands can only be accessed by an account with root 
   privileges
\end{verbatim}
\textbf{TODO}
\begin{verbatim}
   MAKE MANY OF THE TEST CASES HAVE OTHER TEST CASES AS
   PREREQUISITES AND THEN IF THEY FAIL REPORT THAT THE TEST CASE
   FAILED BECAUSE A PREREQUISITE FAILED, AND WHY THAT PREREQUISITE
   FAILED. THIS IS MUCH BETTER THAN HAVING A TEST CASE FAIL DUE
   TO ANOTHER DEPENDENCY AND MAKES THE TEST CASES ORDER-INDEPENDENT
   IE. FOR check_time(), CALL check_ssh() AND VERIFY THAT SSH IS
   FIRST POSSIBLE, THIS GIVES MORE EXPLANATION TO FAILURES RATHER
   THAN A FAILURE FOR THE NTPD TIME BEING INCORRECT, WHEN IN REALITY
   check_time() COULDN'T SSH TO THE MACHINE
   *** THIS IS ESSENTIALLY TAPPER'S YAML STRUCTURE ANYWAYS...
\end{verbatim}
\newpage
\subsection{cernvm-testcases/check\_boot\_error}
\textsl{[ cernvm-testcases ]}
\textsl{[ Functions ]}

\label{ch:robo16}
\label{ch:cernvm_testcases_check_boot_error}
\index{unsorted!check\_boot\_error}\index{Functions!check\_boot\_error}
\textbf{NAME}
\begin{verbatim}
   check_boot_error
\end{verbatim}
\textbf{DESCRIPTION}
\begin{verbatim}
   CernVM Test Case - Check for error messages at boot
\end{verbatim}
\textbf{ARGUMENTS}
\begin{verbatim}
   $1 - The IP address of the machine to login via ssh
   $2 - The name of the boot errors log file
\end{verbatim}
\textbf{RESULT}
\begin{verbatim}
   exitstatus - Sets $? as a zero for success, otherwise sets an error code
\end{verbatim}
\textbf{EXAMPLE}
\begin{verbatim}
   check_boot_error 192.168.1.125 boot-error.log
\end{verbatim}
\newpage
\subsection{cernvm-testcases/check\_ssh}
\textsl{[ cernvm-testcases ]}
\textsl{[ Functions ]}

\label{ch:robo17}
\label{ch:cernvm_testcases_check_ssh}
\index{unsorted!check\_ssh}\index{Functions!check\_ssh}
\textbf{NAME}
\begin{verbatim}
   check_ssh
\end{verbatim}
\textbf{DESCRIPTION}
\begin{verbatim}
   CernVM Test Case - Check login via ssh
\end{verbatim}
\textbf{ARGUMENTS}
\begin{verbatim}
   $1 - The IP address of the machine to login via ssh
   $2 - The username to login with
\end{verbatim}
\textbf{RESULT}
\begin{verbatim}
   exitstatus - Sets $? as a zero for success, otherwise sets an error code
\end{verbatim}
\textbf{EXAMPLE}
\begin{verbatim}
   check_ssh 192.168.1.125 root
\end{verbatim}
\newpage
\subsection{cernvm-testcases/check\_time}
\textsl{[ cernvm-testcases ]}
\textsl{[ Functions ]}

\label{ch:robo18}
\label{ch:cernvm_testcases_check_time}
\index{unsorted!check\_time}\index{Functions!check\_time}
\textbf{NAME}
\begin{verbatim}
   check_time
\end{verbatim}
\textbf{DESCRIPTION}
\begin{verbatim}
   CernVM Test Case - Check for correct time / running ntpd
\end{verbatim}
\textbf{ARGUMENTS}
\begin{verbatim}
   $1 - The IP address of the machine to login via ssh
\end{verbatim}
\textbf{RESULT}
\begin{verbatim}
   exitstatus - Sets $? as a zero for success, otherwise sets an error code
\end{verbatim}
\textbf{EXAMPLE}
\begin{verbatim}
   check_time 192.168.1.125
\end{verbatim}
\newpage
\subsection{cernvm-testcases/check\_web\_restart}
\textsl{[ cernvm-testcases ]}
\textsl{[ Functions ]}

\label{ch:robo19}
\label{ch:cernvm_testcases_check_web_restart}
\index{unsorted!check\_web\_restart}\index{Functions!check\_web\_restart}
\textbf{NAME}
\begin{verbatim}
   check_web_restart
\end{verbatim}
\textbf{DESCRIPTION}
\begin{verbatim}
   CernVM Test Case - Restart through the web interface and check that there
                      are no error messages at boot
\end{verbatim}
\textbf{ARGUMENTS}
\begin{verbatim}
    $1 - The hostname or ip address for the web interface
    $2 - The name of the web reboot logfile
    $3 - The name of the boot error logfile
\end{verbatim}
\textbf{RESULT}
\begin{verbatim}
   exitstatus - Sets $? as a zero for success, otherwise sets an error code
\end{verbatim}
\textbf{EXAMPLE}
\begin{verbatim}
   check_web_restart 192.168.1.125 web-reboot.log boot-error.log
\end{verbatim}
\newpage
\section{test-suite/general-interface}
\textsl{[ Generics ]}

\label{ch:robo34}
\label{ch:test_suite_general_interface}
\index{unsorted!general-interface}\index{Generics!general-interface}
\textbf{NAME}
\begin{verbatim}
   general-interface
\end{verbatim}
\textbf{DESCRIPTION}
\begin{verbatim}
   This script contains general interface functions that interface with 
   the host system and provide generic functionality such as checking the
   host architecture, getting the host operating system, checking if a file
   exists, etc.

   These functions can be utilized to create precondition tests and test 
   cases which require generic functionality that is not part of the
   virt or web interface functions
\end{verbatim}
\newpage
\subsection{general-interface/extract\_file}
\textsl{[ general-interface ]}
\textsl{[ Functions ]}

\label{ch:robo20}
\label{ch:general_interface_extract_file}
\index{unsorted!extract\_file}\index{Functions!extract\_file}
\textbf{NAME}
\begin{verbatim}
   extract_file
\end{verbatim}
\textbf{DESCRIPTION}
\begin{verbatim}
   Extracts a file based on extension within the directory it is located in
\end{verbatim}
\textbf{ARGUMENTS}
\begin{verbatim}
   $1 - The location and name of the file
\end{verbatim}
\textbf{RESULT}
\begin{verbatim}
   exitstatus - Sets $? as a zero for success, otherwise sets an error code
\end{verbatim}
\textbf{EXAMPLE}
\begin{verbatim}
   extract_file /root/file.tar.gz
\end{verbatim}
\newpage
\subsection{general-interface/file\_exists}
\textsl{[ general-interface ]}
\textsl{[ Functions ]}

\label{ch:robo21}
\label{ch:general_interface_file_exists}
\index{unsorted!file\_exists}\index{Functions!file\_exists}
\textbf{NAME}
\begin{verbatim}
   file_exists
\end{verbatim}
\textbf{DESCRIPTION}
\begin{verbatim}
   Simple function that checks if a file/folder exists
\end{verbatim}
\textbf{ARGUMENTS}
\begin{verbatim}
   $1 - The location and name of the file
\end{verbatim}
\textbf{RESULT}
\begin{verbatim}
   exitstatus - Sets $? as a zero for success, otherwise sets an error code
\end{verbatim}
\textbf{EXAMPLE}
\begin{verbatim}
   file_exists ./template.xml
\end{verbatim}
\newpage
\subsection{general-interface/filename\_from\_header}
\textsl{[ general-interface ]}
\textsl{[ Functions ]}

\label{ch:robo22}
\label{ch:general_interface_filename_from_header}
\index{unsorted!filename\_from\_header}\index{Functions!filename\_from\_header}
\textbf{NAME}
\begin{verbatim}
   filename_from_header
\end{verbatim}
\textbf{DESCRIPTION}
\begin{verbatim}
   Function that returns the name of a file to be downloaded given a url by
   looking at the "Location:" specified in HTTP header
\end{verbatim}
\textbf{ARGUMENTS}
\begin{verbatim}
   $1 - The download url of the file
\end{verbatim}
\textbf{RETURN VALUE}
\begin{verbatim}
   filename - The name of a file to be downloaded
\end{verbatim}
\textbf{EXAMPLE}
\begin{verbatim}
   FILE_NAME=$(filename_from_header http://someurl/file.tar.gz)
\end{verbatim}
\newpage
\subsection{general-interface/filename\_from\_url}
\textsl{[ general-interface ]}
\textsl{[ Functions ]}

\label{ch:robo23}
\label{ch:general_interface_filename_from_url}
\index{unsorted!filename\_from\_url}\index{Functions!filename\_from\_url}
\textbf{NAME}
\begin{verbatim}
   filename_from_url
\end{verbatim}
\textbf{DESCRIPTION}
\begin{verbatim}
   Function that returns the name of a file to be downloaded given a url
\end{verbatim}
\textbf{ARGUMENTS}
\begin{verbatim}
   $1 - The download url of the file
\end{verbatim}
\textbf{RETURN VALUE}
\begin{verbatim}
   filename - The name of a file to be downloaded
\end{verbatim}
\textbf{EXAMPLE}
\begin{verbatim}
   FILE_NAME=$(filename_from_url http://someurl/file.tar.gz)
\end{verbatim}
\newpage
\subsection{general-interface/find\_file}
\textsl{[ general-interface ]}
\textsl{[ Functions ]}

\label{ch:robo24}
\label{ch:general_interface_find_file}
\index{unsorted!find\_file}\index{Functions!find\_file}
\textbf{NAME}
\begin{verbatim}
   find_file
\end{verbatim}
\textbf{DESCRIPTION}
\begin{verbatim}
   Function that finds a file and returns the name and path of a file given the 
   root directory and the extension of the file
\end{verbatim}
\textbf{ARGUMENTS}
\begin{verbatim}
   $1 - The root directory to search for the file
   $2 - The extension of the file to look for
\end{verbatim}
\textbf{RETURN VALUE}
\begin{verbatim}
   filelocation - The name and path of a file
\end{verbatim}
\textbf{RESULT}
\begin{verbatim}
   exitstatus - Sets $? as a zero for success, otherwise sets an error code
\end{verbatim}
\textbf{EXAMPLE}
\begin{verbatim}
   FILE_LOCATION=$(find_file /usr/share/images vmdk)
\end{verbatim}
\newpage
\subsection{general-interface/get\_hash}
\textsl{[ general-interface ]}
\textsl{[ Functions ]}

\label{ch:robo25}
\label{ch:general_interface_get_hash}
\index{unsorted!get\_hash}\index{Functions!get\_hash}
\textbf{NAME}
\begin{verbatim}
   get_hash
\end{verbatim}
\textbf{DESCRIPTION}
\begin{verbatim}
   Simple function that returns the hash of a file
\end{verbatim}
\textbf{ARGUMENTS}
\begin{verbatim}
   $1 - The location and name of the file
   $2 - The type of hash, currently supported hashes are: 
        crc32, md5, sha, sha1, sha224, sha256, sha384, sha512
\end{verbatim}
\textbf{RETURN VALUE}
\begin{verbatim}
   hash - The hash of the file
\end{verbatim}
\textbf{EXAMPLE}
\begin{verbatim}
   HASH=$(get_hash /root/file.tar.gz md5)
\end{verbatim}
\newpage
\subsection{general-interface/get\_ip\_address}
\textsl{[ general-interface ]}
\textsl{[ Functions ]}

\label{ch:robo26}
\label{ch:general_interface_get_ip_address}
\index{unsorted!get\_ip\_address}\index{Functions!get\_ip\_address}
\textbf{NAME}
\begin{verbatim}
   get_ip_address
\end{verbatim}
\textbf{DESCRIPTION}
\begin{verbatim}
   Simple function that returns the ip address of a virtual machine
\end{verbatim}
\textbf{ARGUMENTS}
\begin{verbatim}
   $1 - The current hypervisor for the virtual machine network
   $2 - The MAC address of the virtual machine network connection
   $3 - The network XML definition file, not applicable for VMware
\end{verbatim}
\textbf{RETURN VALUE}
\begin{verbatim}
   ipaddress - The ip address defined in the xml network definition file
\end{verbatim}
\textbf{RESULT}
\begin{verbatim}
   exitstatus - Sets $? as a zero for success, otherwise sets an error code
\end{verbatim}
\textbf{EXAMPLE}
\begin{verbatim}
   IP_ADDRESS=$(get_ip_address kvm 52:54:00:7B:30:75 ./network-definition.xml)
   IP_ADDRESS=$(get_ip_address vmware 00:50:56:C0:00:01)
\end{verbatim}
\newpage
\subsection{general-interface/get\_mac\_address}
\textsl{[ general-interface ]}
\textsl{[ Functions ]}

\label{ch:robo27}
\label{ch:general_interface_get_mac_address}
\index{unsorted!get\_mac\_address}\index{Functions!get\_mac\_address}
\textbf{NAME}
\begin{verbatim}
   get_mac_address
\end{verbatim}
\textbf{DESCRIPTION}
\begin{verbatim}
   Simple function that returns the mac address from the virtual machine definition file
\end{verbatim}
\textbf{ARGUMENTS}
\begin{verbatim}
   $1 - The virtual machine XML definition file
\end{verbatim}
\textbf{RETURN VALUE}
\begin{verbatim}
   macaddress - The mac address defined in the xml definition file
\end{verbatim}
\textbf{RESULT}
\begin{verbatim}
   exitstatus - Sets $? as a zero for success, otherwise sets an error code
\end{verbatim}
\textbf{EXAMPLE}
\begin{verbatim}
   MAC_ADDRESS=$(get_mac_address ./kvm-definition.xml)
\end{verbatim}
\newpage
\subsection{general-interface/get\_net\_name}
\textsl{[ general-interface ]}
\textsl{[ Functions ]}

\label{ch:robo28}
\label{ch:general_interface_get_net_name}
\index{unsorted!get\_net\_name}\index{Functions!get\_net\_name}
\textbf{NAME}
\begin{verbatim}
   get_net_name
\end{verbatim}
\textbf{DESCRIPTION}
\begin{verbatim}
   Simple function that returns the network name for a virtual machine
\end{verbatim}
\textbf{ARGUMENTS}
\begin{verbatim}
   $1 - The current hypervisor for the virtual machine network
   $2 - The network XML definition file, not applicable for VMware
\end{verbatim}
\textbf{RETURN VALUE}
\begin{verbatim}
   networkname - The network name defined in the xml network definition file
\end{verbatim}
\textbf{RESULT}
\begin{verbatim}
   exitstatus - Sets $? as a zero for success, otherwise sets an error code
\end{verbatim}
\textbf{EXAMPLE}
\begin{verbatim}
   NET_NAME=$(get_net_name kvm ./network-definition.xml)
   NET_NAME=$(get_net_name vmware)
\end{verbatim}
\newpage
\subsection{general-interface/get\_os\_name}
\textsl{[ general-interface ]}
\textsl{[ Functions ]}

\label{ch:robo29}
\label{ch:general_interface_get_os_name}
\index{unsorted!get\_os\_name}\index{Functions!get\_os\_name}
\textbf{NAME}
\begin{verbatim}
   get_os_name
\end{verbatim}
\textbf{DESCRIPTION}
\begin{verbatim}
   Simple function that returns the specific name or version of the OS
\end{verbatim}
\textbf{ARGUMENTS}
\begin{verbatim}
   $1 - The type of OS
\end{verbatim}
\textbf{RETURN VALUE}
\begin{verbatim}
   osname - The name of the OS
\end{verbatim}
\textbf{EXAMPLE}
\begin{verbatim}
   OSNAME=$(get_os_name "linux")
\end{verbatim}
\newpage
\subsection{general-interface/get\_os\_type}
\textsl{[ general-interface ]}
\textsl{[ Functions ]}

\label{ch:robo30}
\label{ch:general_interface_get_os_type}
\index{unsorted!get\_os\_type}\index{Functions!get\_os\_type}
\textbf{NAME}
\begin{verbatim}
   get_os_type
\end{verbatim}
\textbf{DESCRIPTION}
\begin{verbatim}
   Simple function that returns the type of OS such as linux or osx
\end{verbatim}
\textbf{RETURN VALUE}
\begin{verbatim}
   ostype - The type of OS, either linux,osx, or windows
\end{verbatim}
\textbf{EXAMPLE}
\begin{verbatim}
   OSTYPE=$(get_os_type)
\end{verbatim}
\newpage
\subsection{general-interface/ssh\_autologin}
\textsl{[ general-interface ]}
\textsl{[ Functions ]}

\label{ch:robo31}
\label{ch:general_interface_ssh_autologin}
\index{unsorted!ssh\_autologin}\index{Functions!ssh\_autologin}
\textbf{NAME}
\begin{verbatim}
   ssh_autologin
\end{verbatim}
\textbf{DESCRIPTION}
\begin{verbatim}
   A function which configures automatic SSH login using keys instead of passwords
\end{verbatim}
\textbf{ARGUMENTS}
\begin{verbatim}
   $1 - The IP address of the machine to login via ssh
   $2 - The username to login with
   $3 - The password to login with
\end{verbatim}
\textbf{RESULT}
\begin{verbatim}
   exitstatus - Sets $? as a zero for success, otherwise sets an error code
\end{verbatim}
\textbf{EXAMPLE}
\begin{verbatim}
   ssh_autologin 192.168.1.125 root password
\end{verbatim}
\newpage
\section{test-suite/testsuite-trace}
\textsl{[ Generics ]}

\label{ch:robo35}
\label{ch:test_suite_testsuite_trace}
\index{unsorted!testsuite-trace}\index{Generics!testsuite-trace}
\textbf{NAME}
\begin{verbatim}
   testsuite-trace
\end{verbatim}
\textbf{DESCRIPTION}
\begin{verbatim}
   This script contains several trace and debugging functions that
   provide an easy and effective way to log the execution of CernVM
   test cases, precondition tests, and any other function which
   enables trace and debugging support.

   These functions are well suited for debugging precondition tests
   and CernVM test cases, to enable debugging of a function simply
   call the function using "call", to enable more verbose debugging
   set the trace verbosity level.
\end{verbatim}
\newpage
\subsection{testsuite-trace/add\_trace\_close}
\textsl{[ testsuite-trace ]}
\textsl{[ Functions ]}

\label{ch:robo38}
\label{ch:testsuite_trace_add_trace_close}
\index{unsorted!add\_trace\_close}\index{Functions!add\_trace\_close}
\textbf{NAME}
\begin{verbatim}
   add_trace_close
\end{verbatim}
\textbf{DESCRIPTION}
\begin{verbatim}
   Adds the closing marker for the trace of the function call
\end{verbatim}
\textbf{ARGUMENTS}
\begin{verbatim}
   $1 - The name of the function that was called
\end{verbatim}
\textbf{RESULT}
\begin{verbatim}
   exitstatus - Sets $? as a zero for success, otherwise sets an error code
\end{verbatim}
\textbf{EXAMPLE}
\begin{verbatim}
   add_trace_close connect_virsh
\end{verbatim}
\newpage
\subsection{testsuite-trace/add\_trace\_output}
\textsl{[ testsuite-trace ]}
\textsl{[ Functions ]}

\label{ch:robo39}
\label{ch:testsuite_trace_add_trace_output}
\index{unsorted!add\_trace\_output}\index{Functions!add\_trace\_output}
\textbf{NAME}
\begin{verbatim}
   add_trace_output
\end{verbatim}
\textbf{DESCRIPTION}
\begin{verbatim}
   Adds the data specified to the trace output, which will be logged to file
\end{verbatim}
\textbf{ARGUMENTS}
\begin{verbatim}
   $@ - The data to add to the trace output
\end{verbatim}
\textbf{RESULT}
\begin{verbatim}
   TRACE_OUTPUT - Adds the data speficied to TRACE_OUTPUT global variable
   INDEX        - Sets the global variable index as the next index element of TRACE_OUTPUT
\end{verbatim}
\textbf{EXAMPLE}
\begin{verbatim}
   add_trace_output "${TRACE_TEMPLATE[@]}"
\end{verbatim}
\newpage
\subsection{testsuite-trace/add\_trace\_results}
\textsl{[ testsuite-trace ]}
\textsl{[ Functions ]}

\label{ch:robo40}
\label{ch:testsuite_trace_add_trace_results}
\index{unsorted!add\_trace\_results}\index{Functions!add\_trace\_results}
\textbf{NAME}
\begin{verbatim}
   add_trace_results
\end{verbatim}
\textbf{DESCRIPTION}
\begin{verbatim}
   Adds the results from a traced function call back to the TRACE_OUTPUT heading for the function 
\end{verbatim}
\textbf{ARGUMENTS}
\begin{verbatim}
   $1 - The RETURN VALUE from the function called
   $2 - The RESULT from the function called
   $3 - The index of the heading in TRACE_OUTPUT for the function called
   $4 - The length of the heading for the function called 
\end{verbatim}
\textbf{RESULT}
\begin{verbatim}
   exitstatus - Sets $? as a zero for success, otherwise sets an error code
\end{verbatim}
\textbf{EXAMPLE}
\begin{verbatim}
   add_trace_results $FUNCTION_RETURN_VALUE $FUNCTION_RESULT 0 12
\end{verbatim}
\newpage
\subsection{testsuite-trace/add\_trace\_template}
\textsl{[ testsuite-trace ]}
\textsl{[ Functions ]}

\label{ch:robo41}
\label{ch:testsuite_trace_add_trace_template}
\index{unsorted!add\_trace\_template}\index{Functions!add\_trace\_template}
\textbf{NAME}
\begin{verbatim}
   add_trace_template
\end{verbatim}
\textbf{DESCRIPTION}
\begin{verbatim}
   Adds the contents of the global TRACE_TEMPLATE to TRACE_OUTPUT, this allows the
   TRACE_TEMPLATE file to be dynamically re-generated if "call" is used recursively
\end{verbatim}
\textbf{RESULT}
\begin{verbatim}
   TRACE_OUTPUT - Adds the contents of TRACE_TEMPLATE to the TRACE_OUTPUT global variable
   INDEX        - Sets the global variable INDEX as the next index element of TRACE_OUTPUT
\end{verbatim}
\textbf{EXAMPLE}
\begin{verbatim}
   add_trace_template
\end{verbatim}
\newpage
\subsection{testsuite-trace/call}
\textsl{[ testsuite-trace ]}
\textsl{[ Functions ]}

\label{ch:robo42}
\label{ch:testsuite_trace_call}
\index{unsorted!call}\index{Functions!call}
\textbf{NAME}
\begin{verbatim}
   call
\end{verbatim}
\textbf{DESCRIPTION}
\begin{verbatim}
   The core of the trace functionality, to register the function with trace
   and enable debug and logging support prefix the name of the function and
   its arguments with the trace function "call". This executes the function
   in the trace debugging environment and logs the results based on the 
   trace verbosity level specified, the trace function "call" is transparent
   and returns the same exitstatus and return values as the function called.
\end{verbatim}
\textbf{ARGUMENTS}
\begin{verbatim}
   $1 - The name of the function to call with trace/debugging support
   $@ - The arguments to the function that is being called
\end{verbatim}
\textbf{RETURN VALUE}
\begin{verbatim}
   value - The value returned by the function called if applicable
\end{verbatim}
\textbf{RESULT}
\begin{verbatim}
   exitstatus - Sets the exitstatus as what is returned by the function
                called, $? as a zero for success, otherwise sets an error code
\end{verbatim}
\textbf{EXAMPLE}
\begin{verbatim}
   call download_extract http://cern.ch/cernvm-img.tar.gz /usr/share/image logfile.log
\end{verbatim}
\newpage
\subsection{testsuite-trace/create\_trace\_log}
\textsl{[ testsuite-trace ]}
\textsl{[ Functions ]}

\label{ch:robo43}
\label{ch:testsuite_trace_create_trace_log}
\index{unsorted!create\_trace\_log}\index{Functions!create\_trace\_log}
\textbf{NAME}
\begin{verbatim}
   create_trace_log
\end{verbatim}
\textbf{DESCRIPTION}
\begin{verbatim}
   Creates a new trace log file with a title, the date, and basic info
\end{verbatim}
\textbf{ARGUMENTS}
\begin{verbatim}
   $1 - The name of the trace log file to create
\end{verbatim}
\textbf{RESULT}
\begin{verbatim}
   exitstatus - Sets $? as a zero for success, otherwise sets an error code
\end{verbatim}
\textbf{EXAMPLE}
\begin{verbatim}
   create_trace_log cernvm-trace.log
\end{verbatim}
\newpage
\subsection{testsuite-trace/error\_msg}
\textsl{[ testsuite-trace ]}
\textsl{[ Functions ]}

\label{ch:robo44}
\label{ch:testsuite_trace_error_msg}
\index{unsorted!error\_msg}\index{Functions!error\_msg}
\textbf{NAME}
\begin{verbatim}
   error_msg
\end{verbatim}
\textbf{DESCRIPTION}
\begin{verbatim}
   Message to report in the trace log file if a function has an error
\end{verbatim}
\textbf{ARGUMENTS}
\begin{verbatim}
   $1 - The error message to report
\end{verbatim}
\textbf{RESULT}
\begin{verbatim}
   exitstatus - Sets $? as a zero for success, otherwise sets an error code
\end{verbatim}
\textbf{EXAMPLE}
\begin{verbatim}
   error_msg "The function had an unknown error"
\end{verbatim}
\newpage
\subsection{testsuite-trace/generate\_trace\_template}
\textsl{[ testsuite-trace ]}
\textsl{[ Functions ]}

\label{ch:robo45}
\label{ch:testsuite_trace_generate_trace_template}
\index{unsorted!generate\_trace\_template}\index{Functions!generate\_trace\_template}
\textbf{NAME}
\begin{verbatim}
   generate_trace_template
\end{verbatim}
\textbf{DESCRIPTION}
\begin{verbatim}
   Generate a template for the trace log file output
\end{verbatim}
\textbf{ARGUMENTS}
\begin{verbatim}
   $1 - The name of the function currently being traced
   $@ - The arguments to the function that is being called
\end{verbatim}
\textbf{RESULT}
\begin{verbatim}
   TRACE_TEMPLATE - Generates a trace template for current function
\end{verbatim}
\textbf{EXAMPLE}
\begin{verbatim}
   generate_trace_template create_net ./network-definition.xml default
\end{verbatim}
\newpage
\subsection{testsuite-trace/generic\_msg}
\textsl{[ testsuite-trace ]}
\textsl{[ Functions ]}

\label{ch:robo46}
\label{ch:testsuite_trace_generic_msg}
\index{unsorted!generic\_msg}\index{Functions!generic\_msg}
\textbf{NAME}
\begin{verbatim}
   generic_msg
\end{verbatim}
\textbf{DESCRIPTION}
\begin{verbatim}
   A generic message to report in the trace log file
\end{verbatim}
\textbf{ARGUMENTS}
\begin{verbatim}
   $1 - A generic message to report
\end{verbatim}
\textbf{RESULT}
\begin{verbatim}
   exitstatus - Sets $? as a zero for success, otherwise sets an error code
\end{verbatim}
\textbf{EXAMPLE}
\begin{verbatim}
   generic_msg "Waiting for the CernVM image to start"
\end{verbatim}
\newpage
\subsection{testsuite-trace/log\_trace\_output}
\textsl{[ testsuite-trace ]}
\textsl{[ Functions ]}

\label{ch:robo47}
\label{ch:testsuite_trace_log_trace_output}
\index{unsorted!log\_trace\_output}\index{Functions!log\_trace\_output}
\textbf{NAME}
\begin{verbatim}
   log_trace_output
\end{verbatim}
\textbf{DESCRIPTION}
\begin{verbatim}
   Logs the trace output to the file specified
\end{verbatim}
\textbf{ARGUMENTS}
\begin{verbatim}
   $1 - The name of the logfile to store the trace output
\end{verbatim}
\textbf{RESULT}
\begin{verbatim}
   exitstatus - Sets $? as a zero for success, otherwise sets an error code
\end{verbatim}
\textbf{EXAMPLE}
\begin{verbatim}
   log_trace_output $TRACE_LOGFILE
\end{verbatim}
\newpage
\subsection{testsuite-trace/success\_msg}
\textsl{[ testsuite-trace ]}
\textsl{[ Functions ]}

\label{ch:robo48}
\label{ch:testsuite_trace_success_msg}
\index{unsorted!success\_msg}\index{Functions!success\_msg}
\textbf{NAME}
\begin{verbatim}
   success_msg
\end{verbatim}
\textbf{DESCRIPTION}
\begin{verbatim}
   Message to report in the trace log file if a function executes successfully
\end{verbatim}
\textbf{ARGUMENTS}
\begin{verbatim}
   $1 - The success message to report
\end{verbatim}
\textbf{RESULT}
\begin{verbatim}
   exitstatus - Sets $? as a zero for success, otherwise sets an error code
\end{verbatim}
\textbf{EXAMPLE}
\begin{verbatim}
   success_msg "The function executed successfully"
\end{verbatim}
\newpage
\section{test-suite/virt-interface}
\textsl{[ Generics ]}

\label{ch:robo36}
\label{ch:test_suite_virt_interface}
\index{unsorted!virt-interface}\index{Generics!virt-interface}
\textbf{NAME}
\begin{verbatim}
   virt-interface
\end{verbatim}
\textbf{DESCRIPTION}
\begin{verbatim}
   This script contains several virtualization functions that
   interface with libvirsh and return a success or failure, 
   which can be used to generate a TAP report.

   These functions are well suited for precondition tests to ensure
   that virtual machines can be created and controlled before 
   executing any more tests.
\end{verbatim}
\newpage
\subsection{virt-interface/connect\_virsh}
\textsl{[ virt-interface ]}
\textsl{[ Functions ]}

\label{ch:robo49}
\label{ch:virt_interface_connect_virsh}
\index{unsorted!connect\_virsh}\index{Functions!connect\_virsh}
\textbf{NAME}
\begin{verbatim}
   connect_virsh
\end{verbatim}
\textbf{DESCRIPTION}
\begin{verbatim}
   Connect to virsh for the current virtual machine hypervisor URI and display URI
\end{verbatim}
\textbf{ARGUMENTS}
\begin{verbatim}
   $1 - The URI of the hypervisor
\end{verbatim}
\textbf{RETURN VALUE}
\begin{verbatim}
   URI - The actual URI of the hypervisor that virsh is connected to
\end{verbatim}
\textbf{RESULT}
\begin{verbatim}
   exitstatus - Sets $? as a zero for success, otherwise sets an error code
\end{verbatim}
\textbf{EXAMPLE}
\begin{verbatim}
   connect_virsh qemu:///system
\end{verbatim}
\newpage
\subsection{virt-interface/create\_vm}
\textsl{[ virt-interface ]}
\textsl{[ Functions ]}

\label{ch:robo50}
\label{ch:virt_interface_create_vm}
\index{unsorted!create\_vm}\index{Functions!create\_vm}
\textbf{NAME}
\begin{verbatim}
   create_vm
\end{verbatim}
\textbf{DESCRIPTION}
\begin{verbatim}
   Create a virtual machine from an xml file, verify it has been created 
\end{verbatim}
\textbf{ARGUMENTS}
\begin{verbatim}
   $1 - The path to the virtual machine definition file
   $2 - The name of the virtual machine
\end{verbatim}
\textbf{RESULT}
\begin{verbatim}
   exitstatus - Sets $? as a zero for success, otherwise sets an error code
\end{verbatim}
\textbf{EXAMPLE}
\begin{verbatim}
   create_vm ./vm-definition.xml virt-machine
\end{verbatim}
\newpage
\subsection{virt-interface/create\_vm\_net}
\textsl{[ virt-interface ]}
\textsl{[ Functions ]}

\label{ch:robo51}
\label{ch:virt_interface_create_vm_net}
\index{unsorted!create\_vm\_net}\index{Functions!create\_vm\_net}
\textbf{NAME}
\begin{verbatim}
   create_vm_net
\end{verbatim}
\textbf{DESCRIPTION}
\begin{verbatim}
   Create a virtual machine network from an xml file, verify it has been created 
\end{verbatim}
\textbf{ARGUMENTS}
\begin{verbatim}
   $1 - The path to the network definition file
   $2 - The virtual machine network name
\end{verbatim}
\textbf{RESULT}
\begin{verbatim}
   exitstatus - Sets $? as a zero for success, otherwise sets an error code
\end{verbatim}
\textbf{EXAMPLE}
\begin{verbatim}
   create_vm_net ./network-definition.xml default
\end{verbatim}
\newpage
\subsection{virt-interface/destroy\_vm}
\textsl{[ virt-interface ]}
\textsl{[ Functions ]}

\label{ch:robo52}
\label{ch:virt_interface_destroy_vm}
\index{unsorted!destroy\_vm}\index{Functions!destroy\_vm}
\textbf{NAME}
\begin{verbatim}
   destroy_vm
\end{verbatim}
\textbf{DESCRIPTION}
\begin{verbatim}
   Destroy a virtual machine, verify it has been removed from virsh
\end{verbatim}
\textbf{ARGUMENTS}
\begin{verbatim}
   $1 - The name of the virtual machine
\end{verbatim}
\textbf{RESULT}
\begin{verbatim}
   exitstatus - Sets $? as a zero for success, otherwise sets an error code
\end{verbatim}
\textbf{EXAMPLE}
\begin{verbatim}
   destroy_vm virt-machine
\end{verbatim}
\textbf{NOTES}
\begin{verbatim}
   The files will still exist, the virtual machine is simply no longer 
   accesible until it is re-created 
\end{verbatim}
\newpage
\subsection{virt-interface/destroy\_vm\_net}
\textsl{[ virt-interface ]}
\textsl{[ Functions ]}

\label{ch:robo53}
\label{ch:virt_interface_destroy_vm_net}
\index{unsorted!destroy\_vm\_net}\index{Functions!destroy\_vm\_net}
\textbf{NAME}
\begin{verbatim}
   destroy_vm_net
\end{verbatim}
\textbf{DESCRIPTION}
\begin{verbatim}
   Destroy a virtual machine network, verify it has been removed from virsh
\end{verbatim}
\textbf{ARGUMENTS}
\begin{verbatim}
   $1 - The virtual machine network name
\end{verbatim}
\textbf{RESULT}
\begin{verbatim}
   exitstatus - Sets $? as a zero for success, otherwise sets an error code
\end{verbatim}
\textbf{EXAMPLE}
\begin{verbatim}
   destroy_vm_net default
\end{verbatim}
\textbf{NOTES}
\begin{verbatim}
   The network definition files will still exist, network is simply no longer 
   accesible until it is re-created 
\end{verbatim}
\newpage
\subsection{virt-interface/has\_console\_support}
\textsl{[ virt-interface ]}
\textsl{[ Functions ]}

\label{ch:robo54}
\label{ch:virt_interface_has_console_support}
\index{unsorted!has\_console\_support}\index{Functions!has\_console\_support}
\textbf{NAME}
\begin{verbatim}
   has_console_support
\end{verbatim}
\textbf{DESCRIPTION}
\begin{verbatim}
   Verify that the virtual machine has console support requires that the virtual
   machine has been first started
\end{verbatim}
\textbf{ARGUMENTS}
\begin{verbatim}
   $1 - The name of the virtual machine
\end{verbatim}
\textbf{RESULT}
\begin{verbatim}
   exitstatus - Sets $? as a zero for success, otherwise sets an error code
\end{verbatim}
\textbf{EXAMPLE}
\begin{verbatim}
   has_console_support virt-machine
\end{verbatim}
\textbf{WARNINGS}
\begin{verbatim}
   Support for this function has been deprecated and its use is strongly
   discouraged as console support is unnecessary and only supported for KVM
\end{verbatim}
\newpage
\subsection{virt-interface/start\_vm}
\textsl{[ virt-interface ]}
\textsl{[ Functions ]}

\label{ch:robo55}
\label{ch:virt_interface_start_vm}
\index{unsorted!start\_vm}\index{Functions!start\_vm}
\textbf{NAME}
\begin{verbatim}
   start_vm
\end{verbatim}
\textbf{DESCRIPTION}
\begin{verbatim}
   Start the virtual machine and verify it started
\end{verbatim}
\textbf{ARGUMENTS}
\begin{verbatim}
   $1 - The path to the virtual machine definition file
   $2 - The name of the virtual machine
\end{verbatim}
\textbf{RESULT}
\begin{verbatim}
   exitstatus - Sets $? as a zero for success, otherwise sets an error code
\end{verbatim}
\textbf{EXAMPLE}
\begin{verbatim}
   start_vm virt-machine
\end{verbatim}
\newpage
\subsection{virt-interface/stop\_vm}
\textsl{[ virt-interface ]}
\textsl{[ Functions ]}

\label{ch:robo56}
\label{ch:virt_interface_stop_vm}
\index{unsorted!stop\_vm}\index{Functions!stop\_vm}
\textbf{NAME}
\begin{verbatim}
   stop_vm
\end{verbatim}
\textbf{DESCRIPTION}
\begin{verbatim}
   Stop the virtual machine and verify it has shutdown
\end{verbatim}
\textbf{ARGUMENTS}
\begin{verbatim}
   $1 - The name of the virtual machine
\end{verbatim}
\textbf{RESULT}
\begin{verbatim}
   exitstatus - Sets $? as a zero for success, otherwise sets an error code
\end{verbatim}
\textbf{EXAMPLE}
\begin{verbatim}
   stop_vm virt-machine
\end{verbatim}
\newpage
\subsection{virt-interface/vm\_net\_active}
\textsl{[ virt-interface ]}
\textsl{[ Functions ]}

\label{ch:robo57}
\label{ch:virt_interface_vm_net_active}
\index{unsorted!vm\_net\_active}\index{Functions!vm\_net\_active}
\textbf{NAME}
\begin{verbatim}
   vm_net_active
\end{verbatim}
\textbf{DESCRIPTION}
\begin{verbatim}
   Set the default network as active and verify it is active
\end{verbatim}
\textbf{ARGUMENTS}
\begin{verbatim}
   $1 - The virtual machine network name
\end{verbatim}
\textbf{RESULT}
\begin{verbatim}
   exitstatus - Sets $? as a zero for success, otherwise sets an error code
\end{verbatim}
\textbf{EXAMPLE}
\begin{verbatim}
   vm_net_active default
\end{verbatim}
\newpage
\subsection{virt-interface/vm\_net\_autostart}
\textsl{[ virt-interface ]}
\textsl{[ Functions ]}

\label{ch:robo58}
\label{ch:virt_interface_vm_net_autostart}
\index{unsorted!vm\_net\_autostart}\index{Functions!vm\_net\_autostart}
\textbf{NAME}
\begin{verbatim}
   vm_net_autostart
\end{verbatim}
\textbf{DESCRIPTION}
\begin{verbatim}
   Set the default network to autostart and verify that it is set to autostart
\end{verbatim}
\textbf{ARGUMENTS}
\begin{verbatim}
   $1 - The virtual machine network name
\end{verbatim}
\textbf{RESULT}
\begin{verbatim}
   exitstatus - Sets $? as a zero for success, otherwise sets an error code
\end{verbatim}
\textbf{EXAMPLE}
\begin{verbatim}
   vm_net_autostart default
\end{verbatim}
\newpage
\section{test-suite/web-interface}
\textsl{[ Generics ]}

\label{ch:robo37}
\label{ch:test_suite_web_interface}
\index{unsorted!web-interface}\index{Generics!web-interface}
\textbf{NAME}
\begin{verbatim}
   web-interface
\end{verbatim}
\textbf{DESCRIPTION}
\begin{verbatim}
   This script contains several  functions that provide an interface to
   the CernVM virtual machine web interface and return a success or failure, 
   which can be used to generate a TAP report.

   These functions can be utilized to create test cases in cernvm-testcases or 
   can be executed individually as precondition tests
\end{verbatim}
\newpage
\subsection{web-interface/generate\_header}
\textsl{[ web-interface ]}
\textsl{[ Functions ]}

\label{ch:robo59}
\label{ch:web_interface_generate_header}
\index{unsorted!generate\_header}\index{Functions!generate\_header}
\textbf{NAME}
\begin{verbatim}
   generate_header
\end{verbatim}
\textbf{DESCRIPTION}
\begin{verbatim}
   Generate an http header using the template header and any additional 
   header values defined
\end{verbatim}
\textbf{EXAMPLE}
\begin{verbatim}
   generate_header
\end{verbatim}
\textbf{NOTES}
\begin{verbatim}
   This function should only be called within the scope of a web-interface
   function after the TEMPLATE_HEADER has been generated and the 
   ADDITIONAL_HEADER information unique to the function has been set
\end{verbatim}
\newpage
\subsection{web-interface/generate\_template\_header}
\textsl{[ web-interface ]}
\textsl{[ Functions ]}

\label{ch:robo60}
\label{ch:web_interface_generate_template_header}
\index{unsorted!generate\_template\_header}\index{Functions!generate\_template\_header}
\textbf{NAME}
\begin{verbatim}
   generate_template_header
\end{verbatim}
\textbf{DESCRIPTION}
\begin{verbatim}
   Generate a HTTP template header for the current hypervisor which is a basis
   to generate headers for different web-interface functions
\end{verbatim}
\textbf{ARGUMENTS}
\begin{verbatim}
   $1 - The hostname or ip address for the web interface
\end{verbatim}
\textbf{RESULT}
\begin{verbatim}
   TEMPLATE_HEADER - Exports the generated HTTP template header globally
\end{verbatim}
\textbf{EXAMPLE}
\begin{verbatim}
   generate_template_header 192.168.1.125
\end{verbatim}
\textbf{TODO}
\begin{verbatim}
   PERHAPS GENERATE DIFFERENT USER-AGENT BASED ON HOST OS
\end{verbatim}
\newpage
\subsection{web-interface/web\_apply\_settings}
\textsl{[ web-interface ]}
\textsl{[ Functions ]}

\label{ch:robo61}
\label{ch:web_interface_web_apply_settings}
\index{unsorted!web\_apply\_settings}\index{Functions!web\_apply\_settings}
\textbf{NAME}
\begin{verbatim}
   web_apply_settings
\end{verbatim}
\textbf{DESCRIPTION}
\begin{verbatim}
   Apply settings configured for the CernVM image using the CernVM web interface, 
   which then reboots the CernVM image once completed
\end{verbatim}
\textbf{ARGUMENTS}
\begin{verbatim}
   $1 - The hostname or ip address for the web interface
   $2 - The name of the logfile
\end{verbatim}
\textbf{RESULT}
\begin{verbatim}
   exitstatus - Sets $? as a zero for success, otherwise sets an error code
\end{verbatim}
\textbf{EXAMPLE}
\begin{verbatim}
   web_apply_settings 192.168.1.125 apply-settings.log
\end{verbatim}
\newpage
\subsection{web-interface/web\_check\_interface}
\textsl{[ web-interface ]}
\textsl{[ Functions ]}

\label{ch:robo62}
\label{ch:web_interface_web_check_interface}
\index{unsorted!web\_check\_interface}\index{Functions!web\_check\_interface}
\textbf{NAME}
\begin{verbatim}
   web_check_interface
\end{verbatim}
\textbf{DESCRIPTION}
\begin{verbatim}
   Verify that virtual machine has web interface support
\end{verbatim}
\textbf{ARGUMENTS}
\begin{verbatim}
   $1 - The hostname or ip address for the web interface
   $2 - The name of the logfile
\end{verbatim}
\textbf{RESULT}
\begin{verbatim}
   exitstatus - Sets $? as a zero for success, otherwise sets an error code
\end{verbatim}
\textbf{EXAMPLE}
\begin{verbatim}
   web_check_interface 192.168.1.125 check-interface.log
\end{verbatim}
\newpage
\subsection{web-interface/web\_check\_login}
\textsl{[ web-interface ]}
\textsl{[ Functions ]}

\label{ch:robo63}
\label{ch:web_interface_web_check_login}
\index{unsorted!web\_check\_login}\index{Functions!web\_check\_login}
\textbf{NAME}
\begin{verbatim}
   web_check_login
\end{verbatim}
\textbf{DESCRIPTION}
\begin{verbatim}
   Verify that it is possible to login on web interface
\end{verbatim}
\textbf{ARGUMENTS}
\begin{verbatim}
   $1 - The hostname or ip address for the web interface
   $2 - The web interface username, usually admin
   $3 - The web interface password, by default password
   $4 - The name of the logfile
\end{verbatim}
\textbf{RESULT}
\begin{verbatim}
   exitstatus - Sets $? as a zero for success, otherwise sets an error code
\end{verbatim}
\textbf{EXAMPLE}
\begin{verbatim}
   web_check_login 192.168.1.125 admin password check-login.log
\end{verbatim}
\newpage
\subsection{web-interface/web\_config\_desktop}
\textsl{[ web-interface ]}
\textsl{[ Functions ]}

\label{ch:robo64}
\label{ch:web_interface_web_config_desktop}
\index{unsorted!web\_config\_desktop}\index{Functions!web\_config\_desktop}
\textbf{NAME}
\begin{verbatim}
   web_config_desktop
\end{verbatim}
\textbf{DESCRIPTION}
\begin{verbatim}
   Configure the CernVM image desktop settings using the CernVM web interface
\end{verbatim}
\textbf{ARGUMENTS}
\begin{verbatim}
   $1 - The hostname or ip address for the web interface
   $2 - Configure startx on boot, accepted values are on or off
   $3 - The CernVM image desktop resolution
   $4 - The CernVM image keyboard locale
   $5 - The name of the logfile
\end{verbatim}
\textbf{RESULT}
\begin{verbatim}
   exitstatus - Sets $? as a zero for success, otherwise sets an error code
\end{verbatim}
\textbf{EXAMPLE}
\begin{verbatim}
   web_config_desktop 192.168.1.125 on 1024x768 us config-desktop.log
\end{verbatim}
\newpage
\subsection{web-interface/web\_config\_group}
\textsl{[ web-interface ]}
\textsl{[ Functions ]}

\label{ch:robo65}
\label{ch:web_interface_web_config_group}
\index{unsorted!web\_config\_group}\index{Functions!web\_config\_group}
\textbf{NAME}
\begin{verbatim}
   web_config_group
\end{verbatim}
\textbf{DESCRIPTION}
\begin{verbatim}
   Configure the CernVM image appliance group settings using the CernVM web interface
\end{verbatim}
\textbf{ARGUMENTS}
\begin{verbatim}
   $1 - The hostname or ip address for the web interface
   $2 - The appliance primary group, all capitals, only one group may be specified
   $3 - The name of the logfile
\end{verbatim}
\textbf{RESULT}
\begin{verbatim}
   exitstatus - Sets $? as a zero for success, otherwise sets an error code
\end{verbatim}
\textbf{EXAMPLE}
\begin{verbatim}
   web_config_group 192.168.1.125 ALICE config-group.log
\end{verbatim}
\textbf{TODO}
\begin{verbatim}
   ENABLE AN ARRAY / LIST OF APPLIANCE GROUPS
\end{verbatim}
\newpage
\subsection{web-interface/web\_config\_password}
\textsl{[ web-interface ]}
\textsl{[ Functions ]}

\label{ch:robo66}
\label{ch:web_interface_web_config_password}
\index{unsorted!web\_config\_password}\index{Functions!web\_config\_password}
\textbf{NAME}
\begin{verbatim}
   web_config_password
\end{verbatim}
\textbf{DESCRIPTION}
\begin{verbatim}
   Configure the web interface administrator password using the CernVM web interface
\end{verbatim}
\textbf{ARGUMENTS}
\begin{verbatim}
   $1 - The hostname or ip address for the web interface
   $2 - The new web interface administration password
   $3 - The name of the logfile
\end{verbatim}
\textbf{RESULT}
\begin{verbatim}
   exitstatus - Sets $? as a zero for success, otherwise sets an error code
\end{verbatim}
\textbf{EXAMPLE}
\begin{verbatim}
   web_config_password 192.168.1.125 newpassword config-password.log
\end{verbatim}
\newpage
\subsection{web-interface/web\_config\_proxy}
\textsl{[ web-interface ]}
\textsl{[ Functions ]}

\label{ch:robo67}
\label{ch:web_interface_web_config_proxy}
\index{unsorted!web\_config\_proxy}\index{Functions!web\_config\_proxy}
\textbf{NAME}
\begin{verbatim}
   web_config_proxy
\end{verbatim}
\textbf{DESCRIPTION}
\begin{verbatim}
   Configure the proxy settings using the CernVM web interface
\end{verbatim}
\textbf{ARGUMENTS}
\begin{verbatim}
   $1 - The hostname or ip address for the web interface
   $2 - The name of the logfile
\end{verbatim}
\textbf{RESULT}
\begin{verbatim}
   exitstatus - Sets $? as a zero for success, otherwise sets an error code
\end{verbatim}
\textbf{EXAMPLE}
\begin{verbatim}
   web_config_proxy 192.168.1.125 config-proxy.log
\end{verbatim}
\textbf{TODO}
\begin{verbatim}
   ADD SUPPORT TO ACTUALLY SPECIFY PROXY SETTINGS
\end{verbatim}
\newpage
\subsection{web-interface/web\_create\_user}
\textsl{[ web-interface ]}
\textsl{[ Functions ]}

\label{ch:robo68}
\label{ch:web_interface_web_create_user}
\index{unsorted!web\_create\_user}\index{Functions!web\_create\_user}
\textbf{NAME}
\begin{verbatim}
   web_create_user
\end{verbatim}
\textbf{DESCRIPTION}
\begin{verbatim}
   Create a new user using the CernVM web interface
\end{verbatim}
\textbf{ARGUMENTS}
\begin{verbatim}
   $1 - The hostname or ip address for the web interface
   $2 - The name of the new user to create
   $3 - The password for the new user
   $4 - The group for the new user, lowercase
   $5 - The name of the logfile
\end{verbatim}
\textbf{RESULT}
\begin{verbatim}
   exitstatus - Sets $? as a zero for success, otherwise sets an error code
\end{verbatim}
\textbf{EXAMPLE}
\begin{verbatim}
   web_create_user 192.168.1.125 newuser password alice create-user.log
\end{verbatim}
\textbf{TODO}
\begin{verbatim}
   Add documentation to developer-manual which lists available user groups,
   as well, perhaps account for invalid user group and return error
\end{verbatim}
\newpage
\subsection{web-interface/web\_restart}
\textsl{[ web-interface ]}
\textsl{[ Functions ]}

\label{ch:robo69}
\label{ch:web_interface_web_restart}
\index{unsorted!web\_restart}\index{Functions!web\_restart}
\textbf{NAME}
\begin{verbatim}
   web_restart
\end{verbatim}
\textbf{DESCRIPTION}
\begin{verbatim}
   Restart through the web interface
\end{verbatim}
\textbf{ARGUMENTS}
\begin{verbatim}
   $1 - The hostname or ip address for the web interface
   $2 - The name of the web reboot logfile
\end{verbatim}
\textbf{RESULT}
\begin{verbatim}
   exitstatus - Sets $? as a zero for success, otherwise sets an error code
\end{verbatim}
\textbf{EXAMPLE}
\begin{verbatim}
   web_restart 192.168.1.125 web-restart.log
\end{verbatim}
\newpage
\subsection{web-interface/web\_root\_password}
\textsl{[ web-interface ]}
\textsl{[ Functions ]}

\label{ch:robo70}
\label{ch:web_interface_web_root_password}
\index{unsorted!web\_root\_password}\index{Functions!web\_root\_password}
\textbf{NAME}
\begin{verbatim}
   web_root_password
\end{verbatim}
\textbf{DESCRIPTION}
\begin{verbatim}
   Set the root password using the CernVM web interface
\end{verbatim}
\textbf{ARGUMENTS}
\begin{verbatim}
   $1 - The hostname or ip address for the web interface
   $2 - The new root password
   $3 - The current web interface administration password
   $4 - The name of the logfile
\end{verbatim}
\textbf{RESULT}
\begin{verbatim}
   exitstatus - Sets $? as a zero for success, otherwise sets an error code
\end{verbatim}
\textbf{EXAMPLE}
\begin{verbatim}
 web_root_password 192.168.1.125 newpass currentpass root-password.log
\end{verbatim}

