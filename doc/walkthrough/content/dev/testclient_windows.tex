\section{Windows 7 Test Client Setup}
\subsection{Configuring the system}
\label{sec:windows7config}
\begin{enumerate}
\item	Now that the Windows 7 installation has completed the first step in configuring the system is to remove unnecessary prompts,
		which is necessary for setting up a test client. Begin by going to the control panel from \verb|Windows Logo -> Control Panel|
		and set the ``View by:'' option to ``Small icons'' as this will make it easier to find configuration options.
		
\item	Next, begin by disabling User Account Control (UAC)\footnote{UAC is the annoying prompt which asks you to authenticate certain
		actions and must be removed for services such as VNC, SSH, and running tests to work} from the control panel select ``User Accounts''.
		Then select the option ``Change user account control settings'' and slide the bar down to the bottom, this turns UAC off, then click
		``OK'' to apply the changes.
		
\item	Next, disable automatic updates to prevent updates conflicting with testing and to provide control over when to apply Windows 7 updates,
		go to the control panel and select ``Windows Update'', then select the option "Change settings" and then from the drop down select the 
		option ``Never check updates'' then click ``OK'' to apply the changes.
		
\item	Next, disable Windows Defender, which will conflict with testing and cause other issues, begin by going to the control panel and 
		selecting ``Windows Defender'', then click ``Tools'', \emph{if for some reason you cannot access the ``Tools'' option for
		Windows Defender then ignore this step and continue}. Otherwise, click ``Options'', under ``Administrator options'', select or clear 
		the ``Use Windows Defender'' check box and click ``Save'' to apply the changes. You may be prompted for an administrator password or 	
		confirmation, simply type the password or provide confirmation.
		
\item	Next, disable the Windows Firewall, which will conflict with testing and cause issues such as blocking virtual machine connections 
		and testing services\footnote{This may seem like an unwise security decision, but this guide assumes that the test client has been
		setup in a work environment, which should already have an industry grade firewall that is far superior to Windows Firewall}.
		Begin by navigating to the control panel and select ``Windows Firewall'', then click the option ``Turn Windows Firewall on
		or off'' and for each available firewall configuration, ensure that the option ``Notify me when Windows Firewall blocks a new
		program'' is unchecked and that the option ``Turn off Windows Firewall (not recommended)'' is selected, then click ``OK'' to
		apply the changes.
		
\item	Next, disable Action Center prompts which conflict with testing, begin by going to the control panel and selecting ``Action Center''
		then select the option ``Change Action Center settings''. This will bring up the option to turn messages on or off, {\bf uncheck all of 
		the ``Security messages'' and ``Maintencance messages''} such as ``Windows Update'' and ``Virus Backup'', then click ``OK'' to apply
		the changes.
		
\item	Next, disable error reporting and customer experience improvement messages by again going to the control panel and selecting 
		``Action Center'' then select the option ``Change Action Center settings''. This will bring up the previous menu from the last
		previous step, this time under the section ``Related settings'' do the following.
\begin{itemize}	
\item[a.]	Click ``Customer Experience Improvement Program settings'' and ensure that it is set to ``No, I don't want to participate in
			the program'' and then click ``Save Changes'', you may be prompted for an administrator password or confirmation, simply type
			the password or provide confirmation.
			
\item[b.]	Next, click ``Problem reporting settings'' and ensure that it is set to ``Never check for solutions (not recommended)'' and
			then click ``Change report settings for all users'' and ensure that it is set to ``Never check for solutions (not recommended)''
			as well and click ``OK''. You may be prompted for an administrator password or confirmation, simply type the password or provide
			confirmation.
\end{itemize}	

\item	Now that Windows 7 has been configured to remove unnecessary prompts, the next steps involve performance enhancements and installing
		and configuring necessary software, begin by going to the control panel and select ``Power Options'' then select the option 
		``High Performance''. Next click the option ``Change plan settings'' and ensure that the option ``Put the computer to sleep'' is
		set to `Never'' then click ``Change advanced power settings'' and ensure that the following options are set.
\begin{itemize}
\item	Disable require password on wakeup
\item	Set hard disk to never turn off
\item	Ensure that the options ``Enable wakeup timers'' is enabled
\item	Ensure that the option ``hibernate'' is enabled
\item	Ensure that computer never hibernates
\item	Allow hybrid sleep
\end{itemize}

\item	Next, disable search indexing\footnote{Which does have an impact on disk I/O and is important to virtualized testing where you have
		several virtual machines running using the same physical disk}, begin my clicking \verb|Start -> Computer| and then right click 
		on the 	``Local Disk (C:)'' drive and select ``Properties'' and then uncheck the option ``Allow files on this drive to have contents
		indexed\ldots{}'' and click ``Apply'' to apply the changes, it may take several minutes for the indexes to be removed.
		
\item	Next, from the control panel select ``Administrative Tools'' and click ``Services'' to launch the Windows services administration tool,
		then disable the following unnecessary services by double clicking on each entry, setting the ``Startup type:'' to ``Disabled'' and
		clicking ``Apply''.
\begin{itemize}
\item	BitLocker Drive Encryption Service
\item	Bluetooth Support Service
\item	Remote Registry
\item	Tablet PC Input Service
\item	Windows Biometric Services
\item	Windows Defender
\item	All Windows Media services
\item	Windows Search
\end{itemize}

\item	Next, since this is a test client and Windows Aero and other Windows aesthetic effects are not needed, disable graphics effects and
		enhancements. Begin by going to the control panel and select ``Performance Information and Tools'' and select the option ``Adjust
		visual effects'', then select the option ``Adjust for best performance'' and click ``Apply'' to apply the changes.
		
\item	Next, since some of the necessary drivers for graphics support and other devices may have not been installed automatically begin
		by updating the system, from the control panel select ``Windows Update'', then click the button ``Check for updates'' to check
		for the latest updates, including driver updates. Select from both the regular and recommended updates available any drivers
		listed as they are specific to the system, also if you wish to install updates for the system now, select the updates from the
		list {\bf EXCEPT for update KB971033 which is known to cause issues even with genuine copies of Windows}. Then click ``Apply'' 
		to apply the updates and restart the system after the updates have completed, if you wish to install the latest Windows 7 Service
		Pack you will have to repeat the update procedure after the system restarts.
		
\item	Now that the necessary drivers for the system have been installed, and perhaps the updates as well, the next step is to disable
		Windows 7 from automatically selecting the drivers for the system, which can conflict with the drivers installed by the
		virtualization hypervisors. Begin by going to the control panel and select ``System'' then select the option ``Advanced
		system settings'' and in then click the ``Hardware'' tab. Now, click the ``Device Installation Settings'' button and
		set only the following two options ``No, let me choose what to do'' and ``Never install driver software from Windows
		Update'' then click ``Save Changes''. You may be prompted for an administrator password or confirmation, simply 
		type the password or provide confirmation.
		
		
\item	Next, since not all versions of Windows 7 include remote desktop support, TightVNC Server will be used instead, begin
		by navigating to control panel and select ``System'' then the option ``Remote settings'' ensure that the option
		``Allow Remote Assistance connections to this computer'' is disabled and that the option ``Don't allow connections
		to this computer'' then click ``Apply''.
		
\item	Now, download the latest version of TightVNC for Windows from the following location:\url{http://www.tightvnc.com/download.php}
		and ensure that you download the ``Self-installing package''. Next, execute the installer and simply click ``Next'' and agree
		to the license agreement, then at the ``Choose Components'' stage of the installation select only ``TightVNC Server'' from
		the list and click ``Next'' until you get to the ``Select Additional Tasks'' stage of the installer and ensure that the
		option ``Set passwords for the service before finishing the installation'' is disabled.
		
\item	Finally, configure Windows 7 to login automatically on start, begin by click the Windows logo and typing ``run'' and then press
		enter in the search box, next in the ``Run'' dialog box, type in \emph{\bf control userpasswords2} and press enter. This will
		display the ``User Accounts'' window, uncheck the option ``Users must enter a user name and password to use this computer''
		and click ``OK''. You will then be prompted to enter the current password and confirm it, after doing so, you will no longer 
		be prompted to enter your password to login on the system.

\item	Now that the system has finally been configured for testing\footnote{Hey, it's Windows, you didn't expect this to be a cakewalk
		did you?} reboot the system and ensure that the following work.
\begin{itemize}
\item	It automatically boots up into the full desktop environment without having to login
\item	You have VNC access to the machine and can control the system using VNC	
\end{itemize}
\end{enumerate}


%\newpage
%\subsection{Installing and configuring the hypervisors}
%\subsection{Configuring the CernVM Image}
%\subsection{Installing and configuring Cygwin}
%TODO: ADD INSTRUCTIONS FOR SETTING UP CYGWIN ENVIRONMENT, WHICH IS NECESSARY TO RUN TEST SUITE
%		THE CYGWIN ENVIRONMENT REQUIRES THE FOLLOWING TO RUN TEST SUITES
%		-> BASH
%		-> ALL UNIX UTILITIES (GREP, WGET, CURL, SSH, AWK, SED, NETCAT, TEST, FIND, ...)
%		-> PERL, THE ISSUE WILL BE TO USE STRAWBERRY PERL (WINDOWS ENVIRONMENT) OR PERL 5
%			RUNNING IN THE CYGWIN ENVIRONMENT
%
%TODO:	ADD INSTRUCTIONS FOR SETTING UP SSH ON WINDOWS (THROUGH CYGWIN)
%\subsection{Setting up the Tapper Test Suite}