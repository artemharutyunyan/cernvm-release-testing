\section{Ubuntu 10.04 Based Server Setup}
\subsection{Installing the system}
\label{sec:ubuntuinstall}
\flushleft
\begin{enumerate}
\item 	First, begin by loading the Ubuntu 10.04 LTS CD and select the ``Install Ubuntu 10.04 LTS'' option at the initial boot menu.

\item 	Next, when prompted for the location and time zone, select a region according to where you currently reside and simply configure 
		the time zone according to your local time zone, such as CEST.

\item	When prompted for the keyboard layout, leave it as the default ``Suggested Option: USA'' and simple click ``Forward'' to proceed
		with the installation.
		
\item	Now, when you are at the next stage of the installation called ``Prepare disk space'', if there are no other operating systems 		
		installed on the computer, \emph{such as Windows}, select the option ``Erase and use the entire disk'', {\bf this will erase
		all the data on the hard drive}. Otherwise, if there are other operating systems installed on the system select the option 
		``Install them side by side, choosing between them each startup''.
		 
\item 	When you are at the next stage of the installer called ``Who are you?'' enter a name and username that is simple and 
		relevant such as \emph{cernvm} and set the password to something you will not forget and is fairly complex with numbers and
		letters. But, most importantly {\bf keep the username and password entered consistent across all systems created as part of 
		the infrastructure} as it makes administration and everything else much easier.
		
		Then, for the next option, enter a relevant name for the machine based on the hardware or operating system it is running; the
		hostname should be relevant and unique to better identify the system. A good naming convention should refer to the hardware or 
		operating system and call it a server to differentiate from the test clients, for example a hostname such as 
		\emph{cernvm-ubuntu-server} could be used, {\bf whatever convention you use make sure it is consistent}. Then select the option 
		``Log in automatically'' to have the system login to the desktop environment without entering a password. 

\item	Now, you should be at the last stage of the installation titled ``Ready to install'', simply click the ``Install''
		button to install Ubuntu 10.04 LTS.

\item	After that the system should soon complete the installation, after the installation has finished ensure that you reboot the 
		system and remove the CD so that the system does not load the CD again when it starts.
\end{enumerate}


\newpage
\subsection{Configuring the system}
\label{sec:ubuntuconfig}
\begin{enumerate}
\item	After the system has booted remove the follow unnecessary startup applications by selecting from the menu  
		\verb|System -> Preferences -> Startup Applications|
\begin{itemize}
\item	bluetooth
\item	check for new hardware drivers
\item	evolution alarm
\item	Gnome Login Sound
\item	print queue
\item	pulseaudio
\item	ubuntu one
\item	update notifier
\item	visual assistance/aid
\item	any others you think are unnecessary based on your own discretion
\end{itemize}

\item Next enable and configure remote desktop from the menu \verb|System -> Preferences -> Remote Desktop| and ensure
that the following options are configured
\begin{itemize}
\item	Enable the option ``Allow others to view your desktop''
\item	Enable the option ``Allow other users to control your desktop''
\item	Disable the option ``You must confirm access to this machine''
\end{itemize}

\item 	Next, enable the following repositories so that Ubuntu has access to an even larger amount of software packages, from the
		menu select \verb|System -> Administration -> Software Sources| then click the ``Other Software'' tab and enable the items
		listed. Then click ``Close'' and if prompted click the ``Reload'' button. 

\item 	Next, configure the screen saver from the menu \verb|System -> Preferences -> Screensaver| and ensure that the following options
		are configured
\begin{itemize}
\item	Disable the option ``Lock screen when screensaver active''
\end{itemize}

\item	Next, open the terminal and enable SSH support by installing openSSH using the following commands

\lstset{}
\begin{lstlisting}
# You will be prompted to enter your password
$ sudo apt-get update
$ sudo apt-get install openssh-server
\end{lstlisting}

\item 	The following instructions involve enabling headless support so that you can remote desktop to the machine without having a 
		monitor connected to the computer.
\begin{itemize}
\item[a.] Edit the xorg.conf file and put the following in it

\lstset{language=bash,caption=Configuring Xorg for Headless Support}
\begin{lstlisting}
$ sudo gedit /etc/X11/xorg.conf

Section "Device"
Identifier "VNC Device"
Driver "vesa"
EndSection

Section "Screen"
Identifier "VNC Screen"
Device "VNC Device"
Monitor "VNC Monitor"
SubSection "Display"
Modes "1280x1024"
EndSubSection
EndSection

Section "Monitor"
Identifier "VNC Monitor"
HorizSync 30-70
VertRefresh 50-75
EndSection
\end{lstlisting}
	
\item[b.] Then edit grub and set the option ``nomodeset'', and proceed to update grub and reboot

\lstset{language=bash,caption=Configuring Grub for Headless Support}
\begin{lstlisting}
$ sudo gedit /etc/default/grub

GRUB_CMDLINE_LINUX="nomodeset"

$ sudo update-grub"
\end{lstlisting}
\end{itemize}

\item	Now, reboot the machine, and ensure that the following work
\begin{itemize}
\item	It automatically boots up into the full desktop environment without having to login
\item	You have access to the machine using SSH and can login
\item	You have VNC access to the machine and can control the system using VNC	
\end{itemize}

\item	Finally, update the system from the menu \verb|System -> Administration -> Update Manager| and after it has 
		completed the updates reboot the system
\end{enumerate}


\newpage
\subsection{Installing the Tapper Server}
\begin{enumerate}
\item 	Next, execute the following commands to install necessary dependencies.

\lstset{language=bash,caption= Install Dependencies}
\begin{lstlisting}
$ sudo apt-get update
$ sudo apt-get install make
$ sudo apt-get install subversion
\end{lstlisting}

\item 	Now, download the latest copy of the Tapper-Deployment, which is an installer for \tapper from the \cernvmreleasetesting 
		Google Code Project page

\lstset{language=bash,caption= Download Tapper-Deployment}
\begin{lstlisting}
$ svn checkout http://cernvm-release-testing.googlecode.com/svn/trunk/\
installer/tapper-deployment
\end{lstlisting}

\item 	Now edit the Makefile in the Tapper-Deployment installer folder and configure variable TAPPER\_SERVER which 
		is the hostname of the machine that is currently installing the starter-kit. For now disregard the TESTMACHINE 
		variables, you should have something similar to this in the Makefile.

\lstset{language=bash,caption= Makefile Configuration}
\begin{lstlisting}
# initial machine names
TAPPER_SERVER=cernvm-server
TESTMACHINE1=johnconnor
TESTMACHINE2=sarahconnor
TESTMACHINE3=bullock
\end{lstlisting}

\item	After you have configured the Makefile in the installer folder, install Tapper-Deployment by executing the
		following command, for any prompts during the installation leave them as default and press enter. {\bf During
		the installation, you will be prompted for the mysql password, DO NOT ENTER a password here UNLESS you already have an
		existing MySQL installation/database with a password set for the ``root'' account}. Finally, if you have any errors or
		other issues during the installation, please contact us with a summary of the problem and send us a copy of the installation 
		log ``install.log''. 

\lstset{language=bash,caption= Install Tapper-Deployment}
\begin{lstlisting}
$ cd installer/
$ sudo make localsetup 2>&1 | tee install.log
\end{lstlisting}
\end{enumerate}


\newpage
\subsection{Setting up Tapper Web Interface and Database}
\begin{enumerate}
\item 	Next you need to set a password for the root account of the mysql database\footnote{This will eventually be implemented in the 
		makefile}

\lstset{language=bash,caption= Set MySQL Root Password}
\begin{lstlisting}
# Example: mysqladmin -u root password abc123
$ sudo mysqladmin -u root password <newpassword>
\end{lstlisting}

\item 	Now that the installion has completed and the security issue has been dealt with, ensure that you can access the tapper web
		interface and that it is working by viewing it in your browser using the url, \url{http://localhost/tapper}\footnote{This can be accessed
		locally and remotely from other systems using the server hostname or IP address}

\item 	Next, install PHPMyAdmin so that it's easy to administrate and configure the Tapper databases, when prompted to select the
		``Web server to reconfigure automatically'' select apache2 by pressing the space bar and press enter. \emph{If you are 
		prompted to ``configure databases for phpmyadmin with dbconfig-common'' select {\bf NO} }.

\lstset{language=bash,caption= Install PHPMyAdmin}
\begin{lstlisting}
$ sudo apt-get update

# When prompted for the server to reconfigure automatically select apache2
# when prompted to configure the database with dbconfig-common select NO
$ sudo apt-get install phpmyadmin
\end{lstlisting}

\item	Verify that PHPMyAdmin has been installed and configured correctly and that you can access the tapper web
		interface by viewing it in your browser using the url, \url{http://localhost/phpmyadmin}. Login to the PHPMyAdmin
		web interface using the \emph{username root and the MySQL root password you set earlier using mysqladmin}.

\item 	Now, add all of the configured test machines created in the ``Test Client Platform Setup'' section to the
		\tapper database and set the test clients as active, then add the hardware specifications for each test client
		to the database. This example is just using a single generic test machine, you will have to repeat these commands
		for each test client and change the hostname \emph{cernvm-host} and the values for mem, core,vendor, and has\_ecc 
		as needed; the vendor can be AMD or Intel.
		
\lstset{language=bash,caption= Adding Test Clients to Tapper Database}
\begin{lstlisting}
# Add the hostname of the test client to database
$ tapper-testrun newhost --name cernvm-host --active

# Add the hardware specifications for the test client
$ mysql testrundb -utapper -ptapper
$ insert into host_feature(host_id, entry, value)  values \
((select id from host where name = 'cernvm-host'),     'mem',  4096);
$ insert into host_feature(host_id, entry, value)  values \
((select id from host where name = 'cernvm-host'),   'cores',     4);
$ insert into host_feature(host_id, entry, value)  values \
((select id from host where name = 'cernvm-host'),  'vendor', 'AMD');
$ insert into host_feature(host_id, entry, value)  values \
((select id from host where name = 'cernvm-host'), 'has_ecc',     0);
\end{lstlisting}


\item 	Next, send a sample test report to the tapper server, to ensure that the web interface, MCP, database, and reports
		framework are all working by viewing the tapper reports in your browser at the following url, 
		\url{http://localhost/tapper/reports} You should now see a report from whatever the ``Tapper-Machine-Name'' in 
		demo\_report.tap was set as. \emph{For the example demo\_report.tap provided below it would be cernvm-server}.

\lstset{language=bash,caption= Send a Report from the \tapper~Server to Itself}
\begin{lstlisting}
# Save the following in a file named demo_report.tap
$ vi demo_report.tap

	1..2
	# Tapper-Suite-Name: Tapper-Deployment
	# Tapper-Suite-Version: 1.001
	# Tapper-Machine-Name: cernvm-server
	ok - Hello test world
	ok - Just another description

# Send the report to the tapper server using netcat
$ cat demo_report.tap | netcat -q7 -w1 cernvm-server 7357
\end{lstlisting}

\item 	Finally, ssh login to one of the test machine that was set up earlier, \emph{in our examples, cernvm-host} and send another 
		sample test report to the tapper server, to ensure that the web interface, MCP, database, and reports framework are all working 
		by viewing the tapper reports in your browser at the following url: \url{http://localhost/tapper/reports}. You should now see a 
		report from whatever the ``Tapper-Machine-Name'' in demo\_report.tap was set as. \emph{For the example demo\_report.tap provided 
		below it would be cernvm-testclient}.

\lstset{language=bash,caption= Send a Report to the \tapper~Server from a Test Client}
\begin{lstlisting}
# Save the following in a file named demo_report.tap
$ vi demo_report.tap

	1..2
	# Tapper-Suite-Name: Tapper-Deployment
	# Tapper-Suite-Version: 1.001
	# Tapper-Machine-Name: cernvm-testclient
	ok - Hello test world
	ok - Just another description

# Send the report to the tapper server using netcat
$ cat demo_report.tap | netcat -q7 -w1 cernvm-server 7357
\end{lstlisting}

\item	Now that it has been verified that the tapper server, including the web interface, MCP, database, and reports framework 
		are all working; return to the sections titled ``Setting up the Tapper Test Suite''	for each of the test client, as 
		there are unique instructions for each operating system.
\end{enumerate}