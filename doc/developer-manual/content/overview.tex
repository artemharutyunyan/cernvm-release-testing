\chapter{Overview}
\label{sec:overview}

CernVM currently supports images for VirtualBox, VMware, Xen, KVM and Microsoft Hyper-V hypervisors, each new release of a CernVM image needs to be 
thoroughly tested on each supported platform and hypervisor. The \cernvmreleasetesting\ project is designed to meet this requirement by providing an 
automated testing environment for CernVM images, which will install and configure CernVM images, run the set of tests and report the results on a web
interface.

The intent of this document is to provided a reference manual on the \cernvmtestframework\ for developers, which should provide enough information 
about the design that developers should easily be able to add new \cernvmreleasetesting\ test cases. This developer manual is intended for individuals
who have already set up and configured the core components of a \releasetesting\ infrastructure for CernVM image testing, such as the \amdtapper\
web server and test clients, including hypervisors.If you already have a \cernvmreleasetesting\ infrastructure set up and wish to further expand
and develop the code base, then this guide for you.

All the code needed to begin development of the \cernvmtestframework\ for CernVM image testing is located at the \cernvmreleasetesting\ Google 
Code project page\cite{GCreleasetesting} including this document and all other documentation. 

While this document is not intended to be a replacement for the \amdtapper\ reference manual, the following is a brief description of the
\releasetesting\ infrastructure including an introduction to the core component, \amdtapper\cite{tapper}. Figure~\ref{fig:architecture} 
consists of a diagram outlining the \indexed{\tapper~Architecture}, which consists of test clients and a server, the server is what 
controls the test clients, gathers results, and then displays the results through a web interface.

\begin{figure}[hbp]
	\begin{center}
		\includegraphics[scale=0.25]{img/tapper_architecture_overview.png}
	\end{center}
	\caption{Overview of the \tapper architecture}
	\label{fig:architecture}
\end{figure}

The \cernvmtestframework\ was initially intended to only facilitate the role of ``Test Suites'', which would execute tests and submit a report
file in the form of a ``Test Anything Protocol'' (TAP) file to the ``Test Reports Framework'', which is essentially the web server that displays
the results of tests. But has since been expanded to comprise the role of the ``Test Automation Framework'', which deploys, installs, and configures
the \cernvm images before testing. The most important concept to take away from the diagram is that the \cernvmtestframework\ includes both the
``Test Automation Framework'' and ``Test Suites'', even though it is referred to as a ``Test Suite''.


%The components of the \cernvmtestframework\ are listed in Table~\ref{tbl:components}. 

%\begin{table}
%  \begin{center}

%    \begin{tabularx}{\linewidth}{l|X}
%      {\bf\centering \releasetesting component name} & {\bf\centering Description} \\\hline
%        \texttt{Agent} & Communicates with service insances and requests a job to execute. Upon receiving the job downloads the input 
%                         files, executes the job, uploads job output files and reports that the job execution is finished. \\
%        \texttt{Generic Job and Storage Manager} & Distributes jobs from the internal queue to Agents for execution, provides space for storing 
%                                                   the job output   \\
%        \texttt{AliEn Job Manager} & Retrieves jobs from the central task queue of \indexed{AliEn} Grid \cite{alien} and sends them to 
%                                     Agents for execution \\
%        \texttt{AliEn Storage Manager} & Uploads the output of AliEn jobs executed by Agents and finalizes the job status in AliEn Grid.  \\
%        \texttt{PanDA Storage Manager} & Uploads the output of \indexed{PanDA} \cite{panda} jobs executed by Agents and finalizes the job status in PanDA Grid  \\
%    \end{tabularx}  
%  \end{center}

%  \caption{List of \releasetesting and \amdtapper components}

%  \label{tbl:components}
%\end{table}
