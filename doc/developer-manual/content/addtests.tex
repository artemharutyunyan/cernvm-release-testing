\chapter{Adding \cernvm Test Cases}
\section{Adding Test Cases Overview}
\label{sec:addtestsoverview}

The \cernvm Test Cases are simply the tests which fulfil the role of the ``\cernvm Test Cases'', these tests are just scripted
implementations of the \cernvm Test Cases which would have otherwise been executed manually. The \cernvm Test Cases
are an essential component of the main ``cernvm-tests.sh'' script and require both cernvm-preconditions tests as well as
interface functions to execute the test cases. The cernvm-precondition functions are essential to executing test cases as the 
precondition tests are what initially set up and prepare the host environment for executing the actual test cases. In addition,
the precondition tests can be used by the \cernvm Test Cases to ensure that the minimum dependencies are met before
executing the test case, which allows for modular, or out-of-order testing. This enables a single test to be executed without
having any prerequisites for a specific precondition test or test case to have been executed before.

Therefore, the \cernvm Test Cases differ fom precondition tests because they are merely tests which automate the otherwise
manual process of executing the test cases and checking the \cernvm image. Unlike precondition tests, test cases are not required to
pass in order for another \cernvm test case to be executed accurately, any dependencies required for a test case to execute
accurately should be instead satisfied by a precondition test. 

Thus, unlike precondition tests, the test cases should only contain the functionality essential to executing the \cernvm Test Cases,
anything that could be seen as a prerequisite to executing a test case, such as starting or restarting a virtual machine should
not be part of the test case code. \emph{This is important as separating prerequisites from actual test cases enables modular,
or out of order testing}; writing prerequisites to test cases as precondition tests allows a single test to be executed independently
or the results or execution of another test case.


\section{Example, Adding a New Test Case}
\label{sec:addnewtest}

The following will be an example of adding a \cernvm Test Case which verifies that it is possible to login to the \cernvm image
using SSH, there will be code samples provided, as well as a detailed explanation of the entire procedure to add a new test case
and intergrate it with the ``cernvm-tests.sh'' script.

\begin{enumerate}

\item	The first step to take before adding a single line of code is to sit down and analyze any prerequisites for executing the
			test case, any condition that must be met before the actual test case can be executed can be considered a prerequisite
			and thus would be a precondition test.  First start by creating a list of any procedures that would have to be manually
			executed before actually executing the test case, such as starting the virtual machine and verifying that it has network
			access.
			
			For the test case used in this example, which verifies that it is possible to login to the \cernvm image using SSH, the
			list of prerequisites could be listed as something similar to the following.
			
			\begin{itemize}
			\item	Download and extract the \cernvm image
			\item	Create the virtual machine
			\item	Configure the virtual machine for testing, such as the amount of memory
			\item	Start the virtual machine
			\item	Configure the \cernvm image through the web interface, add a new user, configure experiment group
			\end{itemize}
			
\item	Next, refer to the API reference for a comprehensive list of the precondition tests and interface functions  available 
			and look for any precondition tests or functions that satisfy the prerequisites needed to execute the test case. Specifically,
			refer to the section titled, ``test-suite/cernvm-preconditions''~\ref{ch:test_suite_cernvm_preconditions}, which has every 
			precondition test function documented, including a description of what the precondition test does and the arguments and
			return values.
			
			For the test case used in this example, all of the prerequisites listed in the previous step can be executed
			by existing precondition tests and interface functions. Thus, there is a high probability that many of the prerequisites for
			each test case have been already satisfied by a precondition test. The following is a list of the precondition tests and
			interface functions that satisfy all of the prerequisites listed in the previous step.
			
			\begin{itemize}
				\item	download\_extract()
				\item	create\_def()
				\item	verify\_hypervisor()
				\item	create\_vm()
				\item	start\_vm()
				\item	web\_check\_interface()
				\item	web\_check\_login()
				\item	configure\_image\_web()
			\end{itemize}
			
\item	Now that the functions necessary to meet all of the prerequisites for the \cernvm test case have been determined, the next
			step involves determining the variables and values to pass to the precondition tests and other interface functions. After determining 
			the arguments to the functions, determine which configuration options should be specified in the configuration file to be passed as
			arguments to the functions, as almost all of the variables required by the functions should be specified in the configuration file and
			not set in the cernvm-tests.sh script itself. For a complete list of the available configuration options refer to section~\ref{sec:configfile},
			remember that it is not essential to specify every optional configuration setting as all of the configuration options which are not mandatory 
			have default values.
			
			For the test case used in this is example the KVM \cernvm image will be used to verify SSH access, therefore only the two
			mandatory configuration options ``CVM\_TS\_REPORT\_SERVER'' and ``CVM\_VM\_IMAGE\_VERSION'' would have to 
			be specified in the provided kvm configuration file \verb|content/cernvm-kvm.cfg|. This is because all of the arguments
			required by the functions listed in the previous step have suitable default values specified in the cernvm-tests.sh script
			if the optional configuration setting is not specified. For example, the following code snippet from \emph{cernvm-tests.sh}
			demonstrates the suiteable default values for the variables required by the configure\_image\_web() function, which
			configures the \cernvm image through the web interface.
			
\lstset{language=bash,caption=Suiteable Default Values for the Web Interface}
\begin{lstlisting}
######### Optional Web Interface Settings ##########
ADMIN_USERNAME="${CVM_WEB_ADMIN_USERNAME:-admin}"
ADMIN_DEFAULT_PASS="${CVM_WEB_ADMIN_DEFAULT_PASS:-password}"
ADMIN_PASS="${CVM_WEB_ADMIN_PASS:-VM4l1f3}"

# CernVM image user settings, specify the settings for new account
USER_NAME="${CVM_WEB_USER_NAME:-alice}"
USER_PASS="${CVM_WEB_USER_PASS:-VM4l1f3}"
USER_GROUP="${CVM_WEB_USER_GROUP:-alice}"

# CernVM image root account settings, specify password for root account
ROOT_PASS="${CVM_WEB_ROOT_PASS:-VM4l1f3}"

# CernVM image desktop settings
STARTXONBOOT="${CVM_WEB_STARTXONBOOT:-on}"
RESOLUTION="${CVM_WEB_RESOLUTION:-1024x768}"
KEYBOARD_LOCALE="${CVM_WEB_KEYBOARD_LOCALE:-us}"

# CernVM image primary group (experiment) settings
EXPERIMENT_GROUP="${CVM_WEB_EXPERIMENT_GROUP:-ALICE}"
\end{lstlisting}
			
\item	Now that the functions, variables, and configuration options necessary to meet all of the prerequisites for 
			the \cernvm test cases have been determined. The next step involves adding the precondition tests and 
			other interface functions to the main ``cernvm-tests.sh'' script in a form that can be handled by 
			tapper-autoreport. While the cernvm-tests.sh script is simply a script that has an incremental list of the 
			precondition tests and test cases to be executed, the precondition tests and test cases still need to be
			integrated with tapper-autoreport so that the results of the tests can be added to the Test Anything Protocol
			(TAP) report, which is submitted to the \tapper Server.
			
			Because many of the test cases share similar prerequisites, such as the virtual machine first being created and
			started, it is best to place the ordered list of precondition tests within the cervnvm-tests.sh script before calling
			the test case function. By placing the calls to precondition tests directly in the cernvm-tests.sh script instead
			of calling them from a test case function, test case dependencies can be avoided as the prerequisites for most
			of the \cernvm test cases are met before any test case functions are called.
			
\item	After adding the ordered list of precondition tests to the cernvm-tests.sh script, the next step is to integrate
			the precondition tests with tapper-autoreport so that the results of the tests can be added to the Test Anything 
			Protocol (TAP) report and submitted to the \tapper Server. The easiest method of doing this is to catch the exit
			status of the functions called, which is provided internally by bash using the variable {\bf \$?} and pass the exit
			status to the tapper-autoreport function {\bf ok}. The tapper-autoreport function, {\bf ``ok''} takes two arguments,
			the return code and report message; the most practical method of specifying the report message is to use
			the precondition test's function description from the API reference and use an incremental counter for each test.
			
			The following code snippet from \emph{cernvm-tests.sh} is an example of implementing the precondition tests
			before calling any test case functions and integrating the results of the precondition tests with the auto-tapper
			reporting facilities. All of the variables used in the function calls are either set based on the options provided in
			the configuration file or use default values if the optional configuration options are not specified.

\lstset{language=bash,caption=Adding Precondition Tests and Integrating with Tapper-AutoReport}
\begin{lstlisting}

######### Optional CernVM Image Settings ##########
IMAGE_RELEASE_ID="${CVM_VM_IMAGE_RELEASE_ID}"
NAME="${CVM_VM_NAME:-cernvm-${HYPERVISOR}-${IMAGE_VERSION}}"

######### Optional Web Interface Settings ##########
ADMIN_USERNAME="${CVM_WEB_ADMIN_USERNAME:-admin}"
ADMIN_DEFAULT_PASS="${CVM_WEB_ADMIN_DEFAULT_PASS:-password}"
ADMIN_PASS="${CVM_WEB_ADMIN_PASS:-VM4l1f3}"

######### CernVM Image Settings ##########
VM_XML_DEFINITION="" # Leave blank, the virtual machine definition file

. . .

# Precondition Test 14 - Verify that virtual machine can be started
start_vm ${VM_XML_DEFINITION} $NAME
ok $? "Precondition Test 14 - Verify that virtual machine $VMNAME \
has been started"

. . .

# Precondition Test 17 - Verify that virtual machine has web interface
#                                     support
web_check_interface ${IP_ADDRESS} web_interface.log
ok $? "Precondition Test 17 - Verify that virtual machine $VMNAME has web \
interface support"


# Precondition Test 18 - Verify that it is possible to login on
#                                     web interface
web_check_login ${IP_ADDRESS} $ADMIN_USERNAME $ADMIN_DEFAULT_PASS \
web_interface_login.log
ok $? "Precondition Test 18 - Verify that it is possible to login on\
web interface"


# Precondition Test 19 - Setup and configure the initial CernVM image 
#                                     through the web interface
configure_image_web ${IP_ADDRESS} $ADMIN_USERNAME $ADMIN_DEFAULT_PASS \
web_config_image.log
ok $? "Precondition Test 19 - Setup and configure the initial CernVM image \
through the web interface"
\end{lstlisting}

\item	Finally, the last step involves adding the new test case manually, again review the API reference~\ref{ch:robo32}
			and look for any functions that may already provide the necessary functionality for the test case. In the event that
			the functionality needed has not already been implemented, create new functions in one of the appropriate interface
			files. Then, implement a function for the new test case in the ``cernvm-testscases'' file and call the test case function
			from ``cernvm-tests.sh'' and integrate the results with tapper-autoreport.
						
			In some cases, there is a precondition test which already provides the functionality needed to implement the test case
			and simply needs to be called with arguments specific to the test case. This is the scenario for the test case used
			in this example, a precondition test already exists to verify that the root account has SSH access, to  create the new
			test case the arguments for the precondition test simply need to be changed.

\lstset{language=bash,caption=Adding a New Test Case and Integrating with Tapper-AutoReport}
\begin{lstlisting}

### To implement the test case call the existing precondition test 
### and verify that a specific user, instead of the default root
### account has SSH access
			 
# Add the test case function to cernvm-tests.sh that verifies SSH access
check_ssh()
{
    verify_ssh_login $1 $2

    return $?
}

### Finally add the test case to cernvm-tests.sh and 
### Integrate the results of function with tapper-autoreport

# CernVM Test Case 1 - Check login via ssh as user created through
#                                    the web interface
check_ssh ${IP_ADDRESS} $USER_NAME
ok $? "CernVM Test Case 1 - Check login via ssh as user created \
through web interface"
\end{lstlisting}
			

\end{enumerate}

