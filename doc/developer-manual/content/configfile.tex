\chapter{Test Suite Configuration File}
\label{sec:configfile}

The configuration file is essential to setting up the initial \cernvm test suite for testing, while most of the default 
settings provided in the configuration file are sufficient for most \cernvm image testing environments, there are still
some mandatory settings which {\bf must be configured before testing can begin}. In addition to the mandatory settings
that must be specified before tests can be executed, there are also optional configuration settings which provide settings
that can override the default settings normally taken when the default configuration file is used, these include options
to override the default virtual machine settings specified in the template files.

Each group of settings starts with {\bf CVM}, which is short for \cernvm~, but then has a unique prefix depending on the 
category of setting, there are four categories of options for both the mandatory and optional settings that can be provided. 
The four setting prefixes are {\bf TS, VM, WEB, TC} these denote options that are specific to a category of configuration options. 
The following is brief summary of each configuration setting prefix, and what category of configurations each prefix applies to.

\begin{description}
\item[{\bf TS}]		Options which have this prefix are associated with configuration settings specific to the
					\cernvmreleasetesting Test Suite

\item[{\bf VM}]		Options which have this prefix are associated with configuration settings specific to the
					\cernvm image hypervisor settings, such as the setting for the virtual machine memory
					
\item[{\bf WEB}]	Options which have this prefix are associated with configuration settings specific to the
					\cernvm web interface and configuring the virtual machine through the web interface
					
\item[{\bf TC}]		Options which have this prefix are associated with configuration settings specific to the
					\cernvm Test Cases
\end{description}


\section{Mandatory Settings}
\label{sec:mandatorysettings}

In most testing scenarios only the mandatory configuration settings need to be specified such as the hypervisor and 
the download page, but optional settings are also provided to override internal default settings used by the 
\cernvmtestframework\. The following is a list of the mandatory settings that must be configured in order for the
tests to work, ensure that you enter valid values, \emph{in lower-case}, for the settings indicated.




\begin{description}
\item {\bf	 CVM\_TS\_SUITENAME}
		\begin{itemize}
		\item[-]	Must ALWAYS be set, only define once at the top of the configuration file, 
	  		  	usually the default suite name given in the test suite configuration file is fine
		\end{itemize}
	  		  
\item {\bf	 CVM\_TS\_SUITEVERSION}
		\begin{itemize}
		\item[-]	Must ALWAYS be set, only define once at the top of the configuration file,
	  		  	reflects the release version number of the test suite framework, the default 
      		  	suite version given in the test suite configuration should only be changed if 
     		  	you make modifications to the test suite framework which differentiate it from
      		  	the version released on Google Code.
		\end{itemize}

\item {\bf CVM\_TS\_REPORT\_SERVER}
		\begin{itemize}
		\item[-] Must ALWAYS be set, only define once at the top of the configuration file,
      			this is the ip address or hostname of the Tapper report server which the reports
      			from the test results are sent to
		\end{itemize}
		
\item {\bf CVM\_TS\_DOWNLOAD\_PAGE}
		\begin{itemize}
		\item[-]	Must ALWAYS be set, normally the default url provided in the configuration file is
				accurate, but in the event that the internal \cernvm image release page is relocated
				then this url must be changed.
		\end{itemize}

\item {\bf CVM\_PRECONDITION\_TEST\_LIST}
		\begin{itemize}
		\item[-]	Must ALWAYS be set, it is a space-separated list of the \pretest to be executed,
						should be left as the default as the precondition tests ensure that all of the 
						prerequisites for the \testcase are met. For a description of each test and the
						numerical value associated, please refer to ~\ref{sec:cernvmpretests}.
		\end{itemize}

\item {\bf CVM\_TEST\_CASE\_LIST}
		\begin{itemize}
		\item[-]	Must ALWAYS be set, it is a space-separated list of the \testcase to be executed,
						should be left as the default unless there is a specific list of test cases to
						execute. For a description of each test case and the numerical value associated,
						please refer to ~\ref{sec:cernvmtestcases}.
		\end{itemize}

\item {\bf CVM\_VM\_HYPERVISOR}
		\begin{itemize}
		\item[-]	Must ALWAYS be set, MUST be the first setting before the rest of the mandatory
	  			and optional settings specific to the hypervisor are set
	  	\item[-]	Valid values (case sensitive) are {\bf kvm, vbox, vmware, xen}
		\end{itemize}

\item {\bf CVM\_VM\_TEMPLATE}
		\begin{itemize}
		\item[-]	Must ALWAYS be set, normally the default template provided in the configuration file
				should not be changed, only change this to use a custom template file for the \cernvm 
				image
		\item[-]	The custom template file \emph{must be placed within the templates folder}
		\end{itemize}

\item {\bf CVM\_VM\_NET\_TEMPLATE}
		\begin{itemize}
		\item[-]	Must ALWAYS be set, normally the default network template provided in the configuration file
				should not be changed, only change this to use a custom network template file for the \cernvm 
				image
		\item[-]	The custom network template file \emph{must be placed within the templates folder}
		\item[-] 	The network template file, {\bf only applies to kvm, virtualbox, xen}
		\end{itemize}

\item {\bf CVM\_VM\_IMAGE\_VERSION}
		\begin{itemize}
		\item[-]	Must ALWAYS be set,  MUST be defined for the HYPERVISOR entry in the configuration
				file, specifies the version of the CernVM image to use from the release page
		\end{itemize}
	
\item {\bf CVM\_VM\_IMAGE\_TYPE}
		\begin{itemize}
		\item[-]	Must ALWAYS be set,  MUST be defined for the HYPERVISOR entry in the configuration
				file, specifies the type of CernVM image, such as desktop, basic, head node, etc
		\item[-]	Valid image types supported, (case sensitive) are {\bf basic and desktop}
		\end{itemize}
		
\item {\bf CVM\_VM\_ARCH}
		\begin{itemize}
		\item[-]	Must ALWAYS be set, MUST be defined for the HYPERVISOR entry in the configuration
				file, specifies the architecture of the \cernvm image
		\item[-]	Valid architectures (case sensitive) are {\bf x86 and x86\_64}
		\end{itemize}

% perhaps make these optional...
%export VM_TEMPLATE="cernvm-vbox.xml"
%export VM_NET_TEMPLATE="vbox-network.xml"
%export VM_IMAGE_VERSION="2.4.0"		

\end{description}




\section{Optional Settings}
\label{sec:optionalsettings}

In most testing scenarios only the mandatory configuration settings need to be specified such as the hypervisor and 
the download page, but optional settings are also provided to override internal default settings used by the 
\cernvmtestframework\. The following is a list of the optional settings that may be specified to override the default
settings, the optional settings must be configured for each of the HYPERVISOR settings defined in the configuration file.
The optional settings are separated primarily into four categories, host settings, virtual machine settings, web interface
settings, and test case settings. 

Again, only the mandatory settings are required to be specified in order for the tests to work, the optional settings 
can be ignored completely and the test suite scripts should still execute correctly. Therefore, optional settings should
only be specified by advanced users as improper optional settings can cause precondition tests to return failures,
\emph{it is only recommended that you start configuring optional settings after verifying the results of the scripts 
using only the mandatory settings}.


\paragraph*{Optional Host Settings}
\begin{description}
\item {\bf CVM\_TS\_IMAGES\_DIR}
		\begin{itemize}
		\item[-]	The root directory for the location of the \cernvm images and all
				configuration files and settings, by default /usr/share/images on
				Linux/OS X systems and \verb|C:\users\default\application data\images|
				on Windows systems
		\end{itemize}

\item {\bf CVM\_TS\_OSTYPE}
		\begin{itemize}
		\item[-]	The type of the host operating system, such as linux,  which is automatically 
					determined by the scripts unless specified
		\item[-]	\emph{The valid values,\bf{case sensitive}, are linux, osx, and windows}
		\end{itemize}

\item {\bf CVM\_TS\_OSNAME}
		\begin{itemize}
		\item[-]	The name and version of the host operating system which is automatically determined by
					the scripts unless specified, such as "Red Hat 6" or OS X 10.6.8
		\end{itemize}
		
\item {\bf CVM\_TS\_HOSTNAME}
		\begin{itemize}
		\item[-]	The hostname of the system, determined automatically by the script,
				only set this if you wish to override the default hostname of the
				system
		\end{itemize}
\end{description}


\paragraph*{Optional Virtual Machine Settings}~\newline

The following are the optional virtual machine settings which can be specified to override
the default settings used by the \cernvmtestframework\, these default virtual machine settings
used by the framework are based on the virtual machine XML template definition files defined
in the templates directory.
 
\begin{description}
\item {\bf CVM\_VM\_IMAGE\_RELEASE\_ID}
		\begin{itemize}
		\item[-]	Overrides the default configuration setting, which uses the most recent
				release id of a \cernvm image, with the specific release id of the 
				\cernvm image to download
		\item[-]	The release id is used to help better identifying the image, as for each new
				release added that is the same version, the release id will have incremented
				to indicate that it is a newer release of the same image version
		\end{itemize}
		
\item {\bf CVM\_VM\_NAME}
		\begin{itemize}
		\item[-]	Overrides the default name of the virtual machine set by the 
				virtual machine template XML defintion file
		\item[-]	It is recommended that this setting is specified if testing multiple
				versions of the same \cernvm image, for example a name such as
				``cernvm-vbox-2.4.0'' would help differentiate between other versions
		\end{itemize}
		
\item {\bf CVM\_VM\_CPUS}
		\begin{itemize}
		\item[-]	Overrides the default number of cpus, which is one cpu by default,
				set by the virtual machine template XML defintion file
		\item[-]	Valid values are from {\bf 1 - 4}, but the number specified cannot
				exceed the actual number of cores/cpus on the host system
		\end{itemize}
	
\item {\bf CVM\_VM\_MEMORY}
		\begin{itemize}
		\item[-]	Overrides the default default amount of memory set by the virtual machine
				template XML defintion file
		\item[-]	It is recommended that you specify this value if thrashing occurs on the
				\cernvm image when executing tests due to a lack of memory
		\item[-]	Valid values are in kilobytes and must be based on an amount of memory in
				kilobytes that is a multiple of a base value of 2. For example, to increase
				the memory of a system to 1024 MB, set the value as {\bf 1048576}, which is the
				amount of memory in kilobytes
		\end{itemize}
		
\item {\bf CVM\_VM\_VIDEO\_MEMORY}
		\begin{itemize}
		\item[-]	Overrides the default amount of video memory set by the virtual machine
				template XML defintion file
		\item[-]	It is recommended that you specify this value if display errors occurs on the
				\cernvm image before or when executing tests due to a lack of video memory
		\item[-]	Valid values are in kilobytes and must be based on an amount of video memory in
				kilobytes that is a multiple of a base value of 2. For example, to increase
				the video memory of a system to 64 MB, set the value as {\bf 65536}, which is the
				amount of video memory in kilobytes
		\end{itemize}
		
\item {\bf CVM\_VM\_NET\_NAME}
		\begin{itemize}
		\item[-]	Overrides the default virtual network name set by the virtual machine
				template XML defintion file
		\item[-]	This is the one optional setting {\bf you should never configure}, unless you have
	  			manually created a different virtual network for the hypervisor
		\end{itemize}

\item {\bf CVM\_VM\_MAC\_ADDRESS}
		\begin{itemize}
		\item[-]	Overrides the default MAC address defined in the XML definition file if specified,
					otherwise the default MAC address set by the virtual machine template XML 
					defintion file will be used
		\item[-]	Valid values are valid mac addresses, which are six sets of hexadecimal characters
					separated by colons, for example 00:21:A9:FE:33:2B
		\end{itemize}
\end{description}


\paragraph*{Optional Web Interface Settings}~\newline

The following are the optional web interface settings which can be specified to override
the default settings used by the \cernvmtestframework\, such as the \cernvm image desktop
resolution and the primary experiment group.

\begin{description}
\item {\bf CVM\_WEB\_ADMIN\_USERNAME}
		\begin{itemize}
		\item[-]	Overrides the default web inteface administration account user name,
				which is ``admin'' by default. This optional settings should not 
				have to be modified unless the \cernvm web interface defaults change
		\end{itemize}
		
\item {\bf CVM\_WEB\_ADMIN\_DEFAULT\_PASS}
		\begin{itemize}
		\item[-]	Overrides the default web interface administration account password,
				which is ``password'' by default. This optional settings should not 
				have to be modified unless the \cernvm web interface defaults change
		\end{itemize}
		
\item {\bf CVM\_WEB\_ADMIN\_PASS}
		\begin{itemize}
		\item[-]	Overrides the web inteface administration account password set by
				the test suite scripts with a user defined web interface
				administration password
		\item[-]	The password specified {\bf must be six characters or longer}
		\end{itemize}

\item {\bf CVM\_WEB\_USER\_NAME}
		\begin{itemize}
		\item[-]	Overrides the default account name ``alice'' of the new user created 
				by the test suite scripts through the web interface
		\item[-]	The user name specified should only contain alphabetical characters
		\end{itemize}
		
\item {\bf CVM\_WEB\_USER\_PASS}
		\begin{itemize}
		\item[-]	Overrides the default password ``VM4l1f3'' of the new user created 
				by the test suite scripts through the web interface
		\item[-]	The password specified {\bf must be six characters or longer}
		\end{itemize}

%TODO ADD A LIST OF THE VALID GROUPS FOR THE NEW USER
\item {\bf CVM\_WEB\_USER\_GROUP}
		\begin{itemize}
		\item[-]	Overrides the default group ``alice'' for the new user created 
				by the test suite scripts through the web interface
		\item[-]	The group specified must be a valid group available
				through the web interface, such as ``alice''
		\end{itemize}

\item {\bf CVM\_WEB\_ROOT\_PASS}
		\begin{itemize}
		\item[-]	Overrides the default password ``VM4l1f3'' of the root account
				on the \cernvm image set by the test suite scripts through the
				web interface
		\item[-]	The password specified {\bf must be six characters or longer}
		\end{itemize}
		
\item {\bf CVM\_WEB\_STARTXONBOOT}
		\begin{itemize}
		\item[-]	Overrides the default \cernvm desktop setting set by the test 
				suite scripts through the web interface, which configures X to
				start on boot
		\item[-]	The valid values, (lower-case) are either ``on'' to start X on boot,
				\emph{which is the default}, or ``off'' to not start X on boot		
		\end{itemize}
				
%TODO GET A LIST OF THE VALID RESOLUTIONS ACCEPTED BY THE WEB INTERFACE
\item {\bf CVM\_WEB\_RESOLUTION}
		\begin{itemize}
		\item[-]	Overrides the default \cernvm desktop resolution, {\bf 1024x768} set by
				the test suite scripts through the web interface
		\item[-]	The valid values are valid resolutions up to a {\bf maximum resolution
				of 1680x1050}
		\end{itemize}

%TODO GET A LIST OF THE VALID KEYBOARD LOCALES
\item {\bf CVM\_WEB\_KEYBOARD\_LOCALE}
		\begin{itemize}
		\item[-]	Overrides the default \cernvm desktop keyboard locale, which is ``us'' by
				default, set by the test suite scripts through the web interface
		\item[-]	The valid values are valid locale settings
		\end{itemize}

\item {\bf CVM\_WEB\_EXPERIMENT\_GROUP}
		\begin{itemize}
		\item[-]	Overrides the default \cernvm primary experiment group, which is ``ALICE''
				by default, set by the test suite scripts through the web interface
		\item[-]	The valid values are one of following group names, {\bf the group name
				specified must be in UPPERCASE}: ALICE, ATLAS, CMS, LHCB, LCD, NA61, HONE,
				HEPSOFT, BOSS, GEANT4	
		\end{itemize}
\end{description}


\paragraph*{Optional Test Case Settings}~\newline

The following are the optional test case settings which can be specified to override
the default settings used by the \cernvmtestframework\ for executing the \cernvmreleasetesting
test cases.

\begin{description}
\item {\bf CVM\_TC\_USER\_NAME}
		\begin{itemize}
		\item[-]	Overrides the default account name ``bob'' of the new user created 
				through the web interface as part of a \cernvmreleasetesting test case
		\item[-]	The user name specified should only contain alphabetical characters
		\end{itemize}
		
\item {\bf CVM\_TC\_USER\_PASS}
		\begin{itemize}
		\item[-]	Overrides the default password ``R00tM3'' of the new user created 
				through the web interface as part of a \cernvmreleasetesting test case
		\item[-]	The password specified {\bf must be six characters or longer}
		\end{itemize}
		
\item {\bf CVM\_TC\_EXPERIMENT\_GROUP}
		\begin{itemize}
		\item[-]	Overrides the default primary experiment group, ``LHCB'', to migrate to 
						as part of a \cernvmreleasetesting test case
		\item[-]	The valid values are one of following group names, {\bf the group name
				specified must be in UPPERCASE}: ALICE, ATLAS, CMS, LHCB, LCD, NA61, HONE,
				HEPSOFT, BOSS, GEANT4
		\end{itemize}
		
\item {\bf CVM\_TC\_USER\_GROUP}
		\begin{itemize}
		\item[-]	Overrides the default group ``lhcb'' to change to for the primary 
						user as part of a \cernvmreleasetesting test case
		\item[-]	The group specified must be a valid group available
				through the web interface, such as ``lhcb''
		\end{itemize}
\end{description}