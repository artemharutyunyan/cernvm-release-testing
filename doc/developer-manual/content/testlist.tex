\chapter{CernVM Precondition Tests and Test Cases}
\label{sec:cernvmtests}

The \cernvm precondition tests and test cases are the core of the entire testing infrastructure, these are
the tests and prerequisite tests for test cases which make up all of the testing which is done for the
virtual machine image. {\bf It is essential that you are familiar with these tests and understand that the}
\pretest~ {\bf are tests which ensures that the prerequisites for the actual tests are met.}
Essentially, the \testcase are the actual tests which check for errors and validate the \cernvm image, the
\pretest are tests which ensure that the prerequisites required for the actual \testcase are met, such as
verifying that the virtual machine can even be started on the test system.


\section{CernVM Precondition Tests}
\label{sec:cernvmpretests}

As stated in the introduction, the \pretest are tests which ensure that the prerequisites required for the
actual \testcase are met, such as verifying that the virtual machine can even be started on the test system.
{\bf Although precondition tests are not the actual test cases themselves, they are required to all succeed
in order for the \testcase to be executed accurately}. It is highly recommended that you only execute
the \testcase after all of the \pretest have all completed successfully, as executing the test cases when
a precondition test has failed may result in inaccurate results.

\begin{description}
\item {\bf	 Precondition Test 0}
		\begin{itemize}
		\item[-] 	Verify that the download page exists and that there is a valid download url for the CernVM 
						image specified, returns the url to download the image
		\end{itemize}

\item {\bf	 Precondition Test 1}
		\begin{itemize}
		\item[-]	Download and extract the CernVM image, returns the location of the extracted cernvm image file
		\end{itemize}

\item {\bf	 Precondition Test 2}
		\begin{itemize}
		\item[-]	Create an XML definition file for the virtual machine based on the template XML definition file and
						settings defined and return the location of the final xml defintion file
		\end{itemize}
		
\item {\bf	 Precondition Test 3}
		\begin{itemize}
		\item[-]	Verify that the virtual machine XML definition file exists
		\end{itemize}

\item {\bf	 Precondition Test 4}
		\begin{itemize}
		\item[-]	Verify that the XML definition file provided is valid
		\end{itemize}
		
\item {\bf	 Precondition Test 5}
		\begin{itemize}
		\item[-]	Verify that the mandatory configuration settings for the virtual machine XML definition file have
						been provided and are valid
		\end{itemize}

\item {\bf	 Precondition Test 6}
		\begin{itemize}
		\item[-]	Verify that the hypervisor for the current virtual machine tested is accessible, set the URI as a 
						global variable
		\end{itemize}

\item {\bf	 Precondition Test 7}
		\begin{itemize}
		\item[-]	Create an XML definition file for the virtual machine network based on the template network XML
						definition file and settings defined and return the location of the created xml defintion file
		\end{itemize}
		
\item {\bf	 Precondition Test 8}
		\begin{itemize}
		\item[-]	Verify that the network XML definition file exists
		\end{itemize}

\item {\bf	 Precondition Test 9}
		\begin{itemize}
		\item[-]	Verify that the network XML definition file provided is valid
		\end{itemize}
		
\item {\bf	 Precondition Test 10}
		\begin{itemize}
		\item[-]	Verify that the mandatory configuration settings for the network XML definition file have
						been provided and are valid
		\end{itemize}
		
\item {\bf	 Precondition Test 11}
		\begin{itemize}
		\item[-]	Verify that the virtual machine network has been created from an xml file
		\end{itemize}

\item {\bf	 Precondition Test 12}
		\begin{itemize}
		\item[-]	Verify that virtual machine NAT network is active and  set to autostart, only supported
						for KVM currently
		\end{itemize}
		
\item {\bf	 Precondition Test 13}
		\begin{itemize}
		\item[-]	Set a new UUID for the virtual machine hdd, this test is specific to VirtualBox and is
						a fix for a known UUID conflict issue
		\end{itemize}

\item {\bf	 Precondition Test 14}
		\begin{itemize}
		\item[-]	Verify that virtual machine domain has been created from an xml file
		\end{itemize}
		
\item {\bf	 Precondition Test 15}
		\begin{itemize}
		\item[-]	Verify that virtual machine can be started
		\end{itemize}
		
\item {\bf	 Precondition Test 16}
		\begin{itemize}
		\item[-]	Verify that virtual machine has been stopped
		\end{itemize}

\item {\bf	 Precondition Test 17}
		\begin{itemize}
		\item[-]	Verify that virtual machine has web interface support
		\end{itemize}
		
\item {\bf	 Precondition Test 18}
		\begin{itemize}
		\item[-]	Verify that it is possible to login on web interface
		\end{itemize}

\item {\bf	 Precondition Test 19}
		\begin{itemize}
		\item[-]	Setup and configure the initial CernVM image through the web interface
		\end{itemize}
		
\item {\bf	 Precondition Test 20}
		\begin{itemize}
		\item[-]	Verify that it is possible to login on web interface using the new web
						interface administrator password
		\end{itemize}
		
\item {\bf	 Precondition Test 21}
		\begin{itemize}
		\item[-]	Enable automatic SSH login to the machine for the user specified using
						keys instead of passwords, and verify that it is possible to login automatically
		\end{itemize}

\item {\bf	 Precondition Test 22}
		\begin{itemize}
		\item[-]	Set the root password using the CernVM web interface
		\end{itemize}
		
\item {\bf	 Precondition Test 23}
		\begin{itemize}
		\item[-]	Enable automatic SSH login to the machine for the root user using keys
						instead of passwords, and verify that it is possible to login automatically
		\end{itemize}
\end{description}


\section{CernVM Test Cases}
\label{sec:cernvmtestcases}

As stated in the introduction the \testcase are the actual test cases for testing \cernvm
images, the following is a list of the available \cernvm test cases, for more detailed
information about the functions that make up each test, refer to the API reference at 
the end of this document.

\begin{description}
\item {\bf	 CernVM Test Case 0}
		\begin{itemize}
		\item[-] 	Check login via ssh as user created through web interface
		\end{itemize}

\item {\bf	 CernVM Test Case 1}
		\begin{itemize}
		\item[-] 	Check login via ssh as root
		\end{itemize}

\item {\bf	 CernVM Test Case 2}
		\begin{itemize}
		\item[-] 	No error messages at boot
		\end{itemize}

\item {\bf	 CernVM Test Case 3}
		\begin{itemize}
		\item[-] 	Check for correct time / running ntpd
		\end{itemize}

\item {\bf	 CernVM Test Case 4}
		\begin{itemize}
		\item[-] 	Create a new user using the CernVM web interface
		\end{itemize}

\item {\bf	 CernVM Test Case 5}
		\begin{itemize}
		\item[-] 	 Verify that the user is created and can be accessed from ssh login
		\end{itemize}

\item {\bf	 CernVM Test Case 6}
		\begin{itemize}
		\item[-] 	Restart through the web interface and check that there are no error
						messages at boot
		\end{itemize}

\item {\bf	 CernVM Test Case 7}
		\begin{itemize}
		\item[-] 	Shutdown the system and disconnect the network, then start the image, 
						it should take longer to boot but the system should not hang on startup
		\end{itemize}

\item {\bf	 CernVM Test Case 8}
		\begin{itemize}
		\item[-] 	Check that cernvmfs automount scripts works correctly and is able to 
						mount any experiment group to /cvmfs/
		\end{itemize}
		
\item {\bf	 CernVM Test Case 9}
		\begin{itemize}
		\item[-] 	Check the cvmfs cache list, verify that the cache list is  available after
						restarting the cvmfs daemon
		\end{itemize}

\item {\bf	 CernVM Test Case 10}
		\begin{itemize}
		\item[-] 	Migrate to another experiment such as LHCB using the web interface 
						and make sure the relative tests are loaded 
		\end{itemize}

\item {\bf	 CernVM Test Case 11}
		\begin{itemize}
		\item[-] 	Check that cernvmfs automount scripts works correctly and is able to
						mount the new experiment group to /cvmfs/
		\end{itemize}

\item {\bf	 CernVM Test Case 12}
		\begin{itemize}
		\item[-] 	Check the cvmfs cache list for the new experiment group, verify that the
						cache list is available after restarting the cvmfs daemon 
		\end{itemize}

\item {\bf	 CernVM Test Case 13}
		\begin{itemize}
		\item[-] 	Change the group of the primary user
		\end{itemize}
\end{description}