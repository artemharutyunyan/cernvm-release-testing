\section{Debian Based Test Client Setup}
\subsection{Installing the system}
\label{sec:debianinstall}
\flushleft
\begin{enumerate}
\item 	First, begin by loading the Debian Squeeze CD and select the first `install' option at the initial boot menu.

\item	For the first few installation options, such as default language and keyboard layout simply leave the default values
		and press enter unless you have different requirements than the default options.

\item	When prompted to enter the hostname for the computer, enter a relevant hostname for the machine based on the hardware or 
		operating system it is running; the hostname should be relevant and unique as \tapper will use this hostname to identify the 
		system. A good naming convention should refer to the hardware or operating system and call it a host to differentiate from the 
		virtual machine that will be running as a guest\footnote{This will make more sense later when you have to create and configure 
		CernVM images}, for example a hostname such as \emph{cernvm-debian6-host} could be used, {\bf whatever convention you use make 
		sure it is consistent}.
the following
\item	Next, you will be prompted to enter a domain name, delete the preset domain name ``local'' and leave it blank, \emph{unless
		your network environment has specific domain name requirements}.

\item	Next, when prompted to enter the root password, set the root password to something you will not forget and is
		fairly difficult with numbers and letters. {\bf Again, whatever password you use,  make sure it's consistent as you will 	
		frequently use this account}.

\item 	Next, when prompted to enter a username for the new user to be created on the system, set the username as something simple and 
		relevant such as \emph{cernvm} and the password to something you will not forget and is fairly complex with numbers and letters. 
		But, most importantly {\bf keep this username and password consistent across all systems created as part of the infrastructure} 
		as it makes administration and everything else much easier.

\item 	When prompted for the timezone simply configure the timezone according to your local time zone, such as CEST

\item	Now, when prompted by the installer to configure the partitioning layout, if there are other operating systems installed on the 
		system select the ``Guided - use entire disk'' option, and if available select the option ``Use Remaining Free Space''. 
		Otherwise, if there are no other operating system installed on the hard drive select the ``Manual'' option, \emph{beware that 
		doing so will risk erasing everything on the hard drive if you create a new partition table}. Using the manual option, create 
		two primary partitions, with the first taking up the size of the hard drive minus twice the size of the amount of RAM installed, 
		and the second primary partition as a SWAP file using the remaining free space. The following is an example of what the partiton 
		layout would look like for a 40.0 GB hard drive with 2GB of ram.

\lstset{language=bash,caption=Manual Partition Layout Example}
\begin{lstlisting}
#1	PRIMARY	36.0 GB		B f	 EXT4	/
#2	PRIMARY	4.0  GB		  f	 SWAP	SWAP
\end{lstlisting}

\item 	When prompted to scan another CD or DVD, select {\bf No} unless the system does not have access to the internet, as all the 
		packages will be installed using a network mirror.

\item	When prompted to use a network mirror, select {\bf Yes} and select a mirror near your location and in a country with a reputable 
		connection/internet infrastructure such as (USA, Netherlands, Germany, France).

\item 	When prompted to choose software to install, select the following
\begin{itemize}
\item 	Graphical desktop environment
\item 	SSH Server
\item	Standard system utilities
\end{itemize}

\item 	When prompted to install the GRUB boot loader to the master boot record, select {\bf Yes}

\item	After that the system should soon complete the installation, after the installation has finished ensure that you reboot the 
		system and remove the CD so that the system does not load the CD again when it starts.
\end{enumerate}




\subsection{Configuring the system}
\label{sec:debianconfig}
\begin{enumerate}
\item	After the system has booted remove the follow unnecessary startup applications by selecting from the menu  
		\verb|System -> Preferences -> Startup Applications|
\begin{itemize}
\item	bluetooth
\item	evolution alarm
\item	Gnome Login Sound
\item	print queue
\item	screensaver
\item	update notifier
\item	visual assistance/aid
\item	volume control
\item	any others you think are unnecessary based on your own discretion
\end{itemize}

\item Remove the follow unnecessary services by selecting from the menu \verb|System -> Administration -> Services|
\begin{itemize}
\item	alsa utils
\item	bluetooth
\item	CUPS
\item	exim4
\item	any others you think are unnecessary based on your own discretion
\end{itemize}

\item Next enable and configure remote desktop from the menu \verb|System -> Preferences -> Remote Desktop| and ensure
that the following options are configured
\begin{itemize}
\item	Enable the option ``Allow others to view your desktop''
\item	Enable the option ``Allow other users to control your desktop''
\item	Disable the option ``You must confirm access to this machine''
\end{itemize}

\item Next configure the system to login automatically at boot from the menu select \verb|System -> Administration -> Login Screen|
and then set it to login to the user account you created previously (such as cernvm) automatically.

\item Next, remove cd-rom support from sources.list, which is used by Debian for updates \footnote{And is a nuisance for any new user
as it forces you to find the CD and put it in the computer for the update to continue}, execute the following command with root privileges
and comment out any lines that start with ``deb cdrom'' by using a \#

\lstset{language=bash,caption=Removing CD-ROM Requirement for Updates}
\begin{lstlisting}
$ su -c "gedit /etc/apt/sources.list"
\end{lstlisting}

\item	Again, continue to edit /etc/apt/sources.list still with root privileges and ensure that each line ends with ``main 
		contrib non-free'', then save the file and do the following command with root privileges.

\lstset{language=bash,caption=Updating the System}
\begin{lstlisting}
$ su -c "apt-get update"
\end{lstlisting}

\item Next, configure the screen saver from the menu \verb|System -> Preferences -> Screensaver| and ensure that the following options are 
configured
\begin{itemize}
\item	Disable the option ``Lock screen when screensaver active''
\end{itemize}

\item 	The following instructions involve enabling headless support so that you can remote desktop to the machine without having a 
		monitor connected to the computer
\begin{itemize}
\item[a.] Edit the xorg.conf file and put the following in it

\lstset{language=bash,caption=Configuring Xorg for Headless Support}
\begin{lstlisting}
$ su -c "gedit /etc/X11/xorg.conf"

Section "Device"
Identifier "VNC Device"
Driver "vesa"
EndSection

Section "Screen"
Identifier "VNC Screen"
Device "VNC Device"
Monitor "VNC Monitor"
SubSection "Display"
Modes "1280x1024"
EndSubSection
EndSection

Section "Monitor"
Identifier "VNC Monitor"
HorizSync 30-70
VertRefresh 50-75
EndSection
\end{lstlisting}
	
\item[b.] Then edit grub and set the option ``nomodeset'', and proceed to update grub and reboot

\lstset{language=bash,caption=Configuring Grub for Headless Support}
\begin{lstlisting}
$ su -c "gedit /etc/default/grub"

GRUB_CMDLINE_LINUX="nomodeset"

$ su -c "update-grub"
\end{lstlisting}
\end{itemize}

\item	Now, reboot the machine, and ensure that the following work
\begin{itemize}
\item	It automatically boots up into the full desktop environment without having to login
\item	You have access to the machine using SSH and can login on the root account
\item	You have VNC access to the machine and can control the system using VNC	
\end{itemize}

\item	Finally, update the system from the menu \verb|System -> Administration -> Update Manager| and after it has 
		completed the updates reboot the system
\end{enumerate}




\subsection{Installing libvirt and virsh}
\label{sec:debianvirsh}
\begin{enumerate}
\item	The virtualization API libvirt and the command line tool virsh~\cite{libvirt} are the essential components required 
		for setting up a test client and must be installed and properly configured before any testing can begin. Ensure that
		you follow the proceeding directions carefully and validate that virsh is working properly before proceeding to 
		install and configure the various hypervisors.
		
\item	First, begin by reviewing the release news listed on the libvirt website, \url{http://libvirt.org/news.htm} and read 
		through the release notes for the latest version released to make sure that there are no regressions or deprecated 
		support for the platforms you wish to support. If you intend to set up an entire infrastructure and support all of the
		\cernvm virtualization platforms, which would include \emph{Xen, KVM, VirtualBox, and VMware}, then you must download
		a version later than \emph{0.8.7} as there was no support for VMware prior to that release.

\item	Next, download the latest release that is a {\bf tar.gz} file from the libvirt release server, 
		\url{http://libvirt.org/sources/} based on the latest release which does not have any regressions or deprecations for
		the virtualization platforms you wish to support~\footnote{This shouldn't be an issue but just in case there is a 
		newer version in which Xen support is deprecated, then you would need to use the last release which has Xen support}.
		As of this date, the latest release of libvirt is version 0.9.2, this is the release that will be used for the
		following instructions and examples.
		
\item	Next, install the following dependencies which are required to install the libvirt files from the source files
			that were downloaded, \emph{from now on execute all commands as root}.

\lstset{language=bash,caption=Install Dependencies}
\begin{lstlisting}
$ su
$ apt-get install libxen-dev gnutls-dev libnuma-dev \
libdevmapper-dev python-dev libnl-dev libxml2 \
libxml2-dev libgnutls-dev 

# Install GCC
$ apt-get install gcc make build-essential
\end{lstlisting}

\item	Next, install software for managing and viewing virtual machines.

\lstset{language=bash,caption=Install Virtual Machine Management Software}
\begin{lstlisting}
# Change to root account, enter password if prompted
$ su
$ apt-get install virt-manager virtinst virt-viewer
\end{lstlisting}
		
\item	Next extract the files and execute configure with the following options, then finally compile and install libvirt.

\lstset{language=bash,caption=Compile and Install libvirt}
\begin{lstlisting}
# Extract and execute configure
$ tar -xvvzf libvirt-*.tar.gz
$ cd libvirt-*/
$ ./configure --prefix=/usr --disable-silent-rules --disable-shared \
--enable-static --enable-dependency-tracking --with-xen \
--with-xen-inotify --with-qemu --with-vmware --with-libssh2 \
--with-vbox --with-test --with-remote --with-libvirtd --with-numactl \
--with-network --with-storage-dir

# Compile and install libvirt
$ make 
$ make install
\end{lstlisting}

\item 	Finally ensure that virsh installed correctly and is running by connecting to the test hypervisor and ensuring 
			that the test virtual machine, named ``test'' is running.

\lstset{language=bash,caption=Verify virsh was Installed Properly}
\begin{lstlisting}
# Change to root account, enter password if prompted
$ su

# Verify virsh is working, test should be running
$ virsh -c test:///default list --all
\end{lstlisting}
\end{enumerate}




\subsection{Installing and configuring KVM}
\label{sec:debiankvm}
\begin{enumerate}
\item	The first step is to install the KVM hypervisor, start by installing KVM and the other additional packages such
		as virt-manager, which is a graphical management tool and virt-install, which is a command line interface (CLI)
		virtual machine creation/installation/configuration tool using the following commands with root privileges. If
		you receive a message that a package is already installed then simply continue.

\lstset{language=bash,caption=Installing KVM and Other Related Programs}
\begin{lstlisting}
$  su
$ apt-get install qemu-kvm
\end{lstlisting}

\item	Next, verify that KVM has been installed properly and that virsh can connect to the KVM hypervisor using the
		following commands, if you are able to connect to the virsh console without any errors then virsh is able
		to connect to the KVM hypervisor.

\lstset{language=bash,caption=Verify that virsh can Access KVM}
\begin{lstlisting}
$ su
$ virsh -c qemu:///session
\end{lstlisting}

\item 	Next, to ensure that KVM is properly configured and installed, follow this guide provided on the CernVM website
			\url{http://cernvm.cern.ch/portal/kvm} {\bf except, do not create a kvm definition file as the xml template file
			is provided by the test suite scripts} and verify that you are able to connect to the libvirtd kvm system session.
		
\lstset{language=bash,caption=Verify that KVM is Properly Configured}
\begin{lstlisting}
$ su
$ virsh -c qemu:///system 
\end{lstlisting}

\item	Finally, ensure that you are able to connect to the QEMU/KVM hypervisor using the virtual machine manager, as it 
			is necessary to view the status of the CernVM images and must be installed to troubleshoot and view the CernVM 
			images. Simply launch the virtual machine manager application from 
			\verb|Applications -> System Tools -> Virtual Machine Manager| and{\bf if you are prompted to install libvirt, 
			select ``No'' } as a custom version of libvirt was installed previously.
\end{enumerate}




\subsection{Installing and configuring VirtualBox}
\label{sec:debianvbox}
\begin{enumerate}
\item	First, begin by downloading and installing a version of VirtualBox supported by libvirt from the VirtualBox download 
		page, it is best to download the latest version within the series \emph{that has been available for at least a month 
		prior to the release of the version of libvirt installed}. VirtualBox can be downloaded from the following location,  
		\url{http://www.virtualbox.org/wiki/Downloads} ensure that you select the appropriate Debian based distribution, 
		version and architecture for your system. The following instructions for this section of the guide uses VirtualBox 
		4.0.12 for Debian Squeeze, AMD64.
		
\item	Before installing VirtualBox, install the dependencies for executing VirtualBox as well as the dependencies 
			required to build the VirtualBox kernel modules.

\lstset{language=bash,caption=Install VirtualBox Dependencies}
\begin{lstlisting}
$ su
$ apt-get install libqt4-network libqt4-opengl libqtcore4 libqtgui4
$ apt-get install linux-headers-$(uname -r)
\end{lstlisting}
		 	
\item	Next, after downloading the latest version of VirtualBox for your distribution and installing the dependencies 
			install VirtualBox as the root account using the following command.
\item	Next, after downloading the latest version of VirtualBox for your distribution install VirtualBox as the root
		account using the following command.
\begin{lstlisting}
# Enter the root password when prompted
$ su
$ dpkg -i virtualbox-*.deb
\end{lstlisting}	

\item	Next, in order to use VirtualBox and have full access to the drivers needed for USB support, ensure that the root
			account belongs to the group ``vboxusers'' using the following command.
		
\lstset{language=bash,caption=Add root to VirtualBox Group}
\begin{lstlisting}
$ su -c "usermod -a -G vboxusers root"
\end{lstlisting}	
		
\item	Due to an issue with VirtualBox\footnote{The issues is that VirtualBox looks for virtual machine configuration files (*.vbox)
		in the ``VirtualBox VMs'' folder of the user that launched VirtualBox. The issue is worsened by the fact that there can
		only be one ``VirtualBox VMs'' folder which causes conflicts with multiple users.}, in order for it to work with virsh 
		the virtual machine(s) must be created and configured as the root account, otherwise when you try to connect or start a 
		VirtualBox virtual machine with virsh you will get an ``unknown error'', which is obviously very vague and difficult to 
		resolve. {\bf Therefore ALWAYS start VirtualBox as the root account using the following procedure}.

\lstset{language=bash,caption=Always Start VirtualBox as Root}
\begin{lstlisting}
# Switch to the root account, enter root password
$ su

# Start VirtualBox as root
$ virtualbox
\end{lstlisting}

\item	Finally, verify that VirtualBox has been installed properly and that virsh can connect to the VirtualBox hypervisor, 
		verify that the VirtualBox module, \emph{vboxdrv} has been loaded and that you are able to connect to the virsh console 
		without any errors.

\lstset{language=bash,caption=Verify that virsh can Access VirtualBox}
\begin{lstlisting}
$ su

# Verify that the vboxdrv module is loaded
$ lsmod | grep -i vboxdrv

# Verify that virsh can connect to virtualbox
$ virsh -c vbox:///session
\end{lstlisting}
\end{enumerate}




\subsection{Installing and configuring VMware}
\label{sec:debianvmware}
\begin{enumerate}
\item	First, begin by downloading the latest version of VMware Workstation or VMware Player  from the VMware product 
			page, \url{http://www.vmware.com/products/}, VMware Player is free, whereas VMware Workstation requires a
			license. So if you decide to use VMware Workstation instead of VMware Player you will have to purchase a license
			for it in order to continue.

\item	Before installing VMware, install the dependencies required to build the VMware kernel modules.

\lstset{language=bash,caption=Install VMware Dependencies}
\begin{lstlisting}
$ su
$ apt-get install linux-headers-$(uname -r)
\end{lstlisting}
		
\item 	Next, to install VMware simply set the bundle file as executable and execute the file as root.
\lstset{language=bash,caption=Install VMware}
\begin{lstlisting}
$ su
$ chmod +x VMware*.bundle
$ ./VMware*.bundle
\end{lstlisting}

\item	Next, launch VMware as root and wait for it to compile the kernel modules, then verify that the following VMware 
			kernel modules are loaded, currently virsh has support to connect to the VMware hypervisor, but there are some minor
			issues such as a lack of support for VMware network configurations, currently only ``bridged'' mode is supported.
		
\begin{itemize}
\item        vmnet
\item        vmblock
\item        vmci
\item        vmmon

\end{itemize}

\lstset{language=bash,caption=Verify VMware Kernel Extensions Loaded}
\begin{lstlisting}
# Launch VMware as root, this will build kernel modules
$ su
$ vmware

# Verify that the kernel extentsions are loaded
$ lsmod | grep -i vm
\end{lstlisting}

\item      Finally, verify that VMware has been configured properly and that virsh supports the current version of VMware 
                installed by connecting to the virsh console for the VMware hypervisor.

\lstset{language=bash,caption=Verify VMware Works with Virsh}
\begin{lstlisting}
# Verify that virsh can connect to vmware
$ virsh --connect vmwarews:///session
\end{lstlisting}

\item	If everything so far has worked, then libvirt, virsh, and the hypervisors have been installed and configured properly,
		if you have any outstanding issues solve them before proceeding further, or go to the section ``Server Platform 		
		Setup''~\ref{sec:serversetup} as the \tapper~server does not require libvirt, virsh, or hypervisor configuration.
		
\end{enumerate}




\subsection{Installing and configuring Xen}
\label{sec:debianxen}
\begin{enumerate}

\item	At the moment Xen is not natively supported on Red Hat based platforms, therefore the instructions provided
			are specific to Debian Squeeze and currently Debian provides the only option for testing CernVM Xen images.
			The following instructions are specific to Debian Squeeze, Xen support on Lenny is much more difficult and
			only support for Xen 3.x, which is much older than the latest version of Xen which is provided by
			default on Debian Squeeze. The following instructions are for Xen 4.0, which is currently the latest version of
			Xen.
			
\item	{\bf Before proceeding any further ensure that Xen is not installed on a system which has any of the other 
			hypervisors installed as it will cause severe instability}. It is best to create a new installation for the Xen
			hypervisor only, with libvirt installed as installing Xen on a system with any other hypervisor installed
			causes severe instability and either makes the system kernel panic or causes other failures. 
			
\item	Now that you have a dedicated system for Xen, begin by installing the Xen package and Xen Linux 
			kernel with dom0 support.

\lstset{language=bash,caption=Installing Xen Package and Kernel}
\begin{lstlisting}
$ su
$ apt-get install xen-linux-system
\end{lstlisting}

\item	Next, install the packages required by Xen for Hardware Virtual Machine (HVM) which is required to run the
			\cernvm image.\footnote{This is because the \cernvm image for Xen currently requires a fully virtualized environment,
			(HVM) rather than a paravirtualized environment}

\lstset{language=bash,caption=Installing Packages for HVM Support}
\begin{lstlisting}
$ su
$ apt-get install xen-qemu-dm-4.0
\end{lstlisting}

\item	Next, install the following tools for Xen, which make managing virtual machine and interfacting with the Xen
			dom0 hypervisor much easier.

\lstset{language=bash,caption=Installing Xen Tools}
\begin{lstlisting}
$ apt-get install xen-watch xen-tools
\end{lstlisting}

\item 	Next, configure GRUB to support booting from the Xen kernel by default, otherwise you may reboot and determine
			the Xen entry from the boot menu and then modify the value of GRUB\_DEFAULT in the file /etc/default/grub. The
			easiest method is provided, which is to simply boot the Xen kernel by default, {\bf please note that you may not be
			able to run or test \cernvm images on any hypervisor other than Xen if you boot the Xen kernel}. If you need to use
			KVM/VMware/VirtualBox then you will have to use a separate system or installation and boot the regular Linux
			kernel instead of the Xen kernel.
			
\lstset{language=bash,caption=Configure Booting the Xen Kernel}
\begin{lstlisting}
# Set Xen kernel to boot by default
$ su
$ mv /etc/grub.d/10_linux /etc/grub.d/50_linux
\end{lstlisting}

\item	Next, before rebooting the system, configure Xen to shutdown the virtual machines when the host system 
			shuts down instead of attempting to save the state of the virtual machine. Edit the file \verb|/etc/default/xendomains|
			and configure the two settings XENDOMAINS\_RESTORE and XENDOMAINS\_SAVE as follows.

\lstset{language=bash,caption=Configure Xen Virtual Machine Shutdown}
\begin{lstlisting}
$ su -c "gedit /etc/default/xendomains"

# Configure the settings in the file as follows
XENDOMAINS_RESTORE=false
XENDOMAINS_SAVE=""
\end{lstlisting}

\item	Next, enable a bridged network to the Xen virtual machine by editting the file \verb|/etc/xen/xend-config.sxp| and
			uncommenting the lines {\bf (network-script network-bridge)} and {\bf (vif-script vif-bridge)}.
	
\item	Next, configure the memory for the Xen dom0 and disable dom0 memory ballooning, the minimal memory 
			set for this guide is 1024, but you may increase or lower the amount, 512 should still be an acceptable amount.
			
\lstset{language=bash,caption=Configure Xen Memory Use}
\begin{lstlisting}
$ su -c "gedit /etc/xen/xend-config.sxp"

# Uncomment and configure the following options
(dom0-min-mem 1024)
(enable-dom0-ballooning no)
\end{lstlisting}
			
\item	In addition to configuring the 	the memory for the Xen dom0 in the xend configuration file, the Xen dom0 minimal
			memory must also be set as a GRUB boot paramater, \emph{otherwise dom0 will use all available memory making it
			impossible to boot virtual machines}. The memory set for this guide is 1024, but you may increase or lower the 
			amount, 512 should still be an acceptable amount.

\lstset{language=bash,caption=Configure Xen Memory Use as GRUB Boot Paramater}
\begin{lstlisting}
# Edit /etc/default/grub
$ su
$ gedit /etc/default/grub
# Add the following line to the grub configuration file
# Xen boot parameters for all Xen boots
GRUB_CMDLINE_XEN="dom0_mem=1024M"
 		
# update grub with new configurations
$ update-grub
\end{lstlisting}

\item	Finally, enable the Xen unix daemon, which is a mandatory requirement for executing tests and using libvirt with Xen. Simply 	
			uncomment the following line, {\bf (xend-unix-server no)} and change the value to {\bf yes}. 

\item	Now that Xen has been properly configured reboot the system and wait for it to boot the Xen kernel with dom0
			support, it may take several minutes for the system to boot Xen. {\bf Please note, that if you intend
			on using any of the other hypervisors you must use a completely separate system or installation}.\footnote{This
			is because the Xen dom0 and kernel conflicts with the other hypervisors}.

\item	After the system has booted, first verify that the Xen kernel has loaded properly and that the Xen kernel has booted
			correctly with Domain-0 (dom0) support, which is required in order for the Xen hypervisor to function. \footnote{Booting
			the Xen kernel does nothing unless the kernel has booted with Xen dom0 support} {\bf If you receive any errors for the 
			follow procedure try waiting 5 minutes for the xen daemon (xend) to start, it can often take several minutes to start}.

\lstset{language=bash,caption=Verify Xen Booted Correctly}
\begin{lstlisting}
# Verify the kernel, it should contain "xen"
$ su
$ uname -r

# Verify that Xen has Domain-0 running
$ xm list
\end{lstlisting}

\item	Finally, if Xen kernel has booted correctly and the system is also running with Xen Domain-0 support, verify that it is
			possible to connect to the Xen hypervisor with virsh. If virsh is able to connect to the Xen hypervisor then the Xen dom0, 
			Domain-0 should be seen running by executing``list --all''.
			
\lstset{language=bash,caption=Verify virsh can Connect to Xen Hypervisor}
\begin{lstlisting}
# Connect to Xen hypervisor
$ su
$ virsh -c xen:///

# Domain-0 must be running for Xen support
$ list --all
\end{lstlisting}
\end{enumerate}




\subsection{Setting up the Tapper Test Suite}
\label{sec:debiantestsuite}
\begin{enumerate}
\item 	{\bf Before proceeding any further ensure that you have all other test clients set up this far, and then proceed
		to follow the instructions for setting up and configuring the \tapper~server in the section ``Server Platform Setup''}		
		~\ref{sec:serversetup}.
		
\item 	Now that the \tapper~server has been installed and configured and the \tapper web interface and database have proven
		to be working, the next step is to verify that the test client can actually send a report to the \tapper~server in
		the form of a TAP file. After sending the TAP report to the server, ensure that the test client is working by viewing 
		the tapper reports in your browser at the following url: \url{http:/<tapper\_server>/tapper/reports}. You should now see a 
		report from the test client, there should be a report from a system named whatever the ``Tapper-Machine-Name'' in 
		demo\_report.tap was set as. \emph{For the example demo\_report.tap provided below it would be cernvm-rhtestclient}.
		\footnote{This is why a consistent hostname convention was emphasized earlier, as reports are often sorted and organized 
		based on hostnames}.
		
\lstset{language=bash,caption=Send a Basic Report to the \tapper~Server}
\begin{lstlisting}
# Save the following in a file named demo_report.tap

	1..2
	# Tapper-Suite-Name: Tapper-Deployment
	# Tapper-Suite-Version: 1.001
	# Tapper-Machine-Name: cernvm-rhtestclient
	ok - Hello World
	ok - Just another description

# Send the report to the tapper server using netcat
$ cat demo_report.tap | nc -w10 cernvm-server 7357
\end{lstlisting}

\item 	Next, download a copy of the CernVM Test Suite and the CernVM Test Cases from the Google Code svn repository
		\cite{GCreleasetesting} and install the the following dependencies.
		
\lstset{language=bash,caption=Install CernVM Test Suite and Dependencies}
\begin{lstlisting}
# Install subversion, required to checkout auto-tapper
$ yum install subversion

# Checkout a copy of cernvm testsuite and cernvm testcases
$ svn checkout http://cernvm-release-testing.googlecode.com/svn/\
trunk/tapper/tapper-autoreport/ cernvm-testsuite

# Install the missing dependencies
$ apt-get install curl
$ apt-get install wget
$ apt-get install uuid
$ apt-get install spawn-fcgi
$ apt-get install expect
$ apt-get install expectk
$ apt-get install nmh
\end{lstlisting}

\item	Now that cernvm-testsuite has been installed, configure the variables in the configuration file for the
			hypervisors you want to test and according to your \tapper infrastructure setup. Sample configuration files are provided
			in the {\bf config} folder, all of the settings provided in the configuration files by default are the {\bf mandatory, minimum 
			configuration options} and in most cases the	defaults should be sufficient for testing, the only mandatory settings that are
			not provided by default {\bf and must be configured manually are CVM\_TS\_REPORT\_SERVER and 
			CVM\_VM\_IMAGE\_VERSION }. Please refer to the developer manual for a more complete detailed list of the mandatory and 
			optional configuration settings.

\begin{description}
\item[CVM\_TS\_SUITENAME]	Must ALWAYS be set, the default suite name in the configuration file should be suitable

	  		  
\item[CVM\_TS\_SUITEVERSION] 	Must ALWAYS be set, reflects the release version number of the test suite framework, the  default
								version given in the configuration should only be changed if you customize/update the scripts

\item[CVM\_TS\_REPORT\_SERVER]	Must ALWAYS be set, this is the ip address or hostname of the Tapper report server which the reports
      							from the test results are sent to
		
\item[CVM\_TS\_DOWNLOAD\_PAGE]	Must ALWAYS be set, normally the default url provided in the configuration file is accurate, but in the
								event that the internal \cernvm image release page is relocated then this url must be changed.

\item[CVM\_VM\_HYPERVISOR]	Must ALWAYS be set, should not have to change the defaul hypervisor in the configuration files, valid values 
							(case sensitive) are {\bf kvm,vbox,vmware}

\item[CVM\_VM\_TEMPLATE]		Must ALWAYS be set, normally the default template provided in the configuration file should not be changed,
								only change this to use a custom template file. The custom template file \emph{must be placed within the
								templates folder}

\item[CVM\_VM\_NET\_TEMPLATE]	Must ALWAYS be set, normally the default network template provided in the configuration file should not be
	 							changed, only change this to use a custom network template file for the \cernvm image,  {\bf only applies to kvm
	 							and virtualbox}. The custom network template file \emph{must be placed within the templates folder}

\item[CVM\_VM\_IMAGE\_VERSION]	Must ALWAYS be set, specifies the version of the CernVM image to use from the release page
	
\item[CVM\_VM\_IMAGE\_TYPE]		Must ALWAYS be set,  specifies the type of CernVM image, valid image types supported, (case sensitive) are 
								{\bf basic and desktop}

\item[CVM\_VM\_ARCH]		Must ALWAYS be set, specifies the architecture of the \cernvm image, valid architectures (case sensitive) are {\bf 
							x86 and x86\_64}
\end{description}	

\item	Finally, now that cernvm-testsuite has been installed and configured on the test client and the test client and \tapper~Server
			have proven to be working, the next step is to verify that cernvm-testsuite works correctly and can actually send a report to 	
			the \tapper~server in the form of a TAP file. To execute the \cernvm Test Cases script, ``cernvm-tests.sh'' simply source
			the configuration file you wish to use and execute the script. Once the script has completed and sent a TAP report to the server, 
			ensure that the test client is working by viewing the tapper reports in your browser at the  following 
			url: \url{http:/<tapper\_server>/tapper/reports}. You should now see a new report from the test client, there should be a 
			report from a system with the same hostname.\footnote{This is why a consistent hostname convention was emphasized earlier, as 
			reports are often sorted and organized based on hostnames}.

\lstset{language=bash,caption=Execute CernVM Test Cases Script}
\begin{lstlisting}
# Simply source the configuration file, and execute the script
$ . ./config/<configuration_file>
$ ./cernvm-tests.sh
\end{lstlisting}
\end{enumerate}